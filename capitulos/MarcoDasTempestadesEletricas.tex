\chapter{MARCO DAS TEMPESTADES ELÉTRICAS NA AMÉRICA DO SUL}
%\chapter{Marco das Tempestades Elétricas na América do Sul }

O Marco das tempestades elétricas descreve os locais e quando estes sistemas ocorrem na América do Sul. Determina-se a sazonalidade, o ciclo diurno, a distribuição espacial de raios e das tempestades elétricas. 


\section{CICLO DIURNO E CICLO ANUAL}

Utilizando a base de dados de tempestades elétricas construída nesta pesquisa, foi estudada a frequência de ocorrências dos sistemas no decorrer das horas do dia, figura \ref{ciclodiurnototal}, e meses do ano, figura \ref{cicloanualtotal}. Deste modo, obtivemos na figura \ref{diurnoanual}, o ciclo diurno e anual das tempestades elétricas por meio da distribuição de probabilidade de ocorrências.
% entre 10$^{\circ}$ Norte -- 40$^{\circ}$ Sul e 30$^{\circ}$  -- 90$^{\circ}$ Oeste.

\begin{figure}[!hb]
  \centering{
  \subfloat[Ciclo diruno.]{{\includegraphics[scale=0.95]{img/ciclos/ciclodiurno19982011total}} \label{ciclodiurnototal}}
  \subfloat[Ciclo anual.]{{\includegraphics[scale=0.95]{img/ciclos/cicloanual19982011total}} \label{cicloanualtotal}}
  }
\caption{Ciclo diurno e anual das tempestades elétricas observadas em hora local. Os valores de probabilidade foram normalizados pelo total dos 154,189 sistemas identificados.}
\label{diurnoanual} 
\end{figure} 

A figura \ref{ciclodiurnototal},  mostra que, entre 14h e 15h as tempestades elétricas são mais prováveis, indicando que o aquecimento da superfície do continente e o aumento da camada limite planetária no decorrer do dia são ingredientes que podem aumentar a probabilidade de ocorrência em até 4,6 vezes em relação aos horários de menor fluxo de calor sensível para a atmosfera. Enquanto o TRMM observou 2312 tempestades elétricas às 9h (hora local), às 14h foram observadas 13,877.

No ciclo anual, conforme mostra a figura \ref{cicloanualtotal}, observa-se que a estação de tempestades elétricas na América do Sul possui dois picos, um em outubro e outro em março, porém contempla os meses de outubro, novembro, dezembro, janeiro, fevereiro e março. A maior probabilidade de ocorrência esteve associada ao mês de outubro, que concentrou 16,961 tempestades elétricas observadas em 14 anos. % Entre janeiro e março as tempestades elétricas tiveram probabilidade de 0.3\% menor do que no início da estação, em outubro.   


O ciclo diurno também foi estudado para cada região de 10 por 10 graus, como mostra a figura \ref{diurno}. 



\begin{figure}[!hb]
\centering{\includegraphics[scale=1.5]{img/ciclos/ciclodiurno10x1019982011localtime}}  
\caption{Ciclo diurno em hora local para as tempestades elétricas observadas em cada região de 10 por 10 graus. Os valores de probabilidade são mostrados em porcentagem e foram normalizados pelo total de 154,189 sistemas observados.}
\label{diurno}
\end{figure}


Mesmo que em uma análise geral mostre a importância do aquecimento superficial do continente para a ocorrência de tempestades elétricas, sistemas noturnos sobre a Colômbia e Venezuela são bastante frequentes. Na figura \ref{diurno}, entre 0$^{\circ}$--10$^{\circ}$ Norte e 80$^{\circ}$--70$^{\circ}$ Oeste, às 0h em hora local, temos o maior valor de probabilidade (0.4\%) de tempestades elétricas noturnas da América do Sul, o que representou um número de 617 sistemas observados em 14 anos, apenas entre 0h e 00:59h.


A circulação de vale e montanha associada com a topografia elevada na Colômbia, principalmente a região do Parque Nacional Natural Paramillo, e o Lago Maracaibo na Venezuela, combinados com a atuação da Zona de Convergência Intertropical (ZCIT), promovem condições para o desenvolvimento de tempestades elétricas noturnas de maneira mais eficiente do que as demais regiões. \sigla{name={ZCIT},description={Zona de Convergência Intertropical}}

No Oceano Pacífico, entre 0$^{\circ}$--10$^{\circ}$ Norte e 90$^{\circ}$--80$^{\circ}$ Sul abrangendo o Parque Nacional da Ilha do Coco na Costa Rica e parte das ilhas Galápagos no Equador, foi a região oceânica com a maior probabilidade de ocorrência de tempestades elétricas. Esta possui um ciclo diurno duplo de tempestades elétricas. Elas ocorrem às 4h, em hora local, e as 14h. A maior probabilidade de ocorrência (0.15\%) foi observada às 4h, que correspondeu à 231 sistemas.

No Pacífico Sul, as tempestades elétricas são mais raras do que as demais regiões devido a atuação permanente da subsidência da Célula de Hadley, que modula a Alta Subtropical do Pacífico Sul, responsável também por regiões como o Deserto do Atacama e parte do semi-árido Argentino.

Na região do Atlântico Subtropical, a probabilidade de tempestades elétricas é maior do que no Atlântico Norte. A passagem de sistemas transientes entre 40$^{\circ}$--30$^{\circ}$ Sul e 50$^{\circ}$--30$^{\circ}$ Oeste e 30$^{\circ}$--20$^{\circ}$ Sul e 40$^{\circ}$--30$^{\circ}$ Oeste, gera maior número de tempestades elétricas oceânicas do que com a atuação da ITCZ no Atlântico Tropical. Observa-se também que nas regiões oceânicas o ciclo diurno das tempestades elétricas indica maior atividade noturna e não às 14-15h igual no continente.

%No Atlântico Tropical Norte, entre 0$^{\circ}$--10$^{\circ}$ Norte e 50$^{\circ}$--30$^{\circ}$ Oeste, o pico de tempestades elétricas deve estar associado com a atividade da ZCIT.

O pico de atividade de tempestades elétricas durante o ciclo diurno, figura \ref{diurno}, ocorreu entre 10$^{\circ}$--0$^{\circ}$ Sul e 70$^{\circ}$--50$^{\circ}$ Oeste e 20$^{\circ}$--10$^{\circ}$ Sul e 60$^{\circ}$--50$^{\circ}$ Oeste. Em cada uma destas três caixas, observou-se a probabilidade de aproximadamente 0.8\% entre as 14h e 15h, mostrando que em toda esta região, o TRMM observou uma média de 3 tempestades elétricas a cada 2 dias para apenas estas duas horas do dia.


Entre 30$^{\circ}$--20$^{\circ}$ Sul e 60$^{\circ}$--50$^{\circ}$ Oeste, na figura \ref{diurno}, região de grande atividade de Sistemas Convectivos de Meso-escala (MCS) conforme descrevem \citeonline{Durkee2009}, encontra-se um máximo durante a tarde e os sistemas noturnos tiveram probabilidade de ocorrência 2.7 vezes menor do que os valores encontrados sobre os vales na Colômbia e Venezuela, mostrando que a ocorrência dos MCS ao Sul da América do Sul com ciclo de vida maior do que 9h ou com formação noturna, não possuem probabilidade de ocorrência que destaca-se em relação as demais regiões continentais, mesmo neste banco de dados composto apenas por tempestades elétricas. Na figura \ref{anual}, há um máximo de atividade em outubro que antecede a estação de tempestades elétricas entre dezembro e março. O máximo é observado em janeiro com 1234 sistemas identificados na região.\sigla{name={MCS},description={Sistemas Convectivos de Meso-escala}}


A tabela \ref{caracEstacao} mostra os meses de duração das estações de tempestades elétricas com base na estudo mostrado na figura \ref{anual}. Os períodos em que a probabilidade de ocorrência de sistemas foram superiores à 0.7 do máximo observado na região, foram considerados como os períodos das estações de tempestades elétricas. %Regiões como as linhas 19 e 20 da tabela \ref{caracEstacao} mostram que houveram apenas 32 tempestades elétricas em 14 anos, portanto 


\begin{figure}[!ht]
\centering{\includegraphics[scale=1.5]{img/ciclos/cicloanual10x1019982011localtime}}  
\caption{Ciclo anual em hora local para as tempestades elétricas observadas em cada região de 10 por 10 graus. Os valores de probabilidade são mostrados em porcentagem e foram normalizados pelo total de 154,189 sistemas observados. As linhas horizontais cortam o valor de 0.7 do máximo de probabilidade, utilizado como limiar para definir o início e fim das estações de tempestades elétricas.}
\label{anual}
\end{figure}



\begin{table}[!ht]
\caption{Principais características do ciclo anual de probabilidade de ocorrência de tempestades elétricas observadas entre 1998-2011, em cada região de 10 por 10 graus.}
\label{caracEstacao}
\centering
\small
\newcommand{\grayline}{\rowcolor[gray]{.88}}
\renewcommand {\tabularxcolumn }[1]{ >{\arraybackslash }m{#1}}
\newcolumntype{W}{>{\centering\arraybackslash}X}
\begin{tabularx}{\textwidth}{p{0.6cm} p{3.5cm} W W W W} %{|p{10cm}|X|X|X|X|X|X|X|X| }
\hline\hline 
\grayline  & Localização & Número de sistemas & Estação (meses) & Duração (meses) & Máximo\\[1.5pt]
\hline
1&0$^{\circ}$--10$^{\circ}$N, 90$^{\circ}$--80$^{\circ}$O& 4159  & Abr--Set &6& Abr\\[1.5pt]\grayline
2&0$^{\circ}$--10$^{\circ}$N, 80$^{\circ}$--70$^{\circ}$O& 14,047 & Mar--Nov &9& Out\\[1.5pt]
3&0$^{\circ}$--10$^{\circ}$N, 70$^{\circ}$--60$^{\circ}$O& 11,787 & Jul--Out &4& Set\\[1.5pt]\grayline
4&0$^{\circ}$--10$^{\circ}$N, 60$^{\circ}$--50$^{\circ}$O&  4868 & Jul--Set &3& Ago\\[1.5pt]
5&0$^{\circ}$--10$^{\circ}$N, 50$^{\circ}$--40$^{\circ}$O& 645 & Mai--Jun, Set--Nov &5& Set\\[1.5pt] \grayline
6&0$^{\circ}$--10$^{\circ}$N, 40$^{\circ}$--30$^{\circ}$O& 821 & Out--Dez &3& Dez\\[1.5pt]

7&10$^{\circ}$--0$^{\circ}$S, 90$^{\circ}$--80$^{\circ}$O& 217 & Mar &1& Mar\\[1.5pt]\grayline
8&10$^{\circ}$--0$^{\circ}$S, 80$^{\circ}$--70$^{\circ}$O& 9721 & Set--Dez &4& Out\\[1.5pt]
9&10$^{\circ}$--0$^{\circ}$S, 70$^{\circ}$--60$^{\circ}$O& 12,168 & Ago--Nov &4& Out\\[1.5pt]\grayline
10&10$^{\circ}$--0$^{\circ}$S, 60$^{\circ}$--50$^{\circ}$O& 12,231 & Set--Dez &4& Out\\[1.5pt]
11&10$^{\circ}$--0$^{\circ}$S, 50$^{\circ}$--40$^{\circ}$O& 7731 & Jan--Abr, Nov--Dez &6&  Jan\\[1.5pt]\grayline
12&10$^{\circ}$--0$^{\circ}$S, 40$^{\circ}$--30$^{\circ}$O& 1349 & Jan--Abr &4&  Mar\\[1.5pt]

13&20$^{\circ}$--10$^{\circ}$S, 90$^{\circ}$--80$^{\circ}$O& 1  &   --0--  & 1 & Mai \\[1.5pt]\grayline
14&20$^{\circ}$--10$^{\circ}$S, 80$^{\circ}$--70$^{\circ}$O& 4254 & Jan--Fev,  Set--Dez  &6& Out\\[1.5pt]
15&20$^{\circ}$--10$^{\circ}$S, 70$^{\circ}$--60$^{\circ}$O& 8585 & Jan--Mar, Set--Dez &7& Out\\[1.5pt]\grayline
16&20$^{\circ}$--10$^{\circ}$S, 60$^{\circ}$--50$^{\circ}$O& 10,414 & Jan--Mar,  Out--Dez &6& Out\\[1.5pt]
17&20$^{\circ}$--10$^{\circ}$S, 50$^{\circ}$--40$^{\circ}$O& 8201 & Jan--Mar, Out--Dez &6&  Jan\\[1.5pt]\grayline
18&20$^{\circ}$--10$^{\circ}$S, 40$^{\circ}$--30$^{\circ}$O& 611 & Jan--Mar &3&  Mar\\[1.5pt]

19&30$^{\circ}$--20$^{\circ}$S, 90$^{\circ}$--80$^{\circ}$O& 32 & Mai--Jun &2&  Mai\\[1.5pt]\grayline
20&30$^{\circ}$--20$^{\circ}$S, 80$^{\circ}$--70$^{\circ}$O& 32 & Fev, Mai, Jul--Ago &4&  Fev, Mai,  Jul\\[1.5pt]
21&30$^{\circ}$--20$^{\circ}$S, 70$^{\circ}$--60$^{\circ}$O& 5558 & Dez--Mar &4& Jan\\[1.5pt]\grayline
22&30$^{\circ}$--20$^{\circ}$S, 60$^{\circ}$--50$^{\circ}$O& 8676 & Dez--Mar &4& Jan\\[1.5pt]
23&30$^{\circ}$--20$^{\circ}$S, 50$^{\circ}$--40$^{\circ}$O& 5996 & Dez--Mar &4& Jan\\[1.5pt]\grayline
24&30$^{\circ}$--20$^{\circ}$S, 40$^{\circ}$--30$^{\circ}$O& 1849 & Fev--Mar, Mai, Dez &4&  Mar\\[1.5pt]

25&40$^{\circ}$--30$^{\circ}$S, 90$^{\circ}$--80$^{\circ}$O& 258 & Jun &1&  Jun \\[1.5pt]\grayline
26&40$^{\circ}$--30$^{\circ}$S, 80$^{\circ}$--70$^{\circ}$O& 370 & Jan--Mar, Mai--Jun &5&  Jan\\[1.5pt]
27&40$^{\circ}$--30$^{\circ}$S, 70$^{\circ}$--60$^{\circ}$O& 7638 & Dez--Jan &2&  Jan\\[1.5pt]\grayline
28&40$^{\circ}$--30$^{\circ}$S, 60$^{\circ}$--50$^{\circ}$O& 5403 & Dez--Mar &4&  Jan\\[1.5pt]
29&40$^{\circ}$--30$^{\circ}$S, 50$^{\circ}$--40$^{\circ}$O& 2966 & Jan--Set &9&  Abr\\[1.5pt]\grayline
30&40$^{\circ}$--30$^{\circ}$S, 40$^{\circ}$--30$^{\circ}$O& 2288 & Abr--Jun  &3& Mai\\[1.5pt]


\hline 
\end{tabularx}
\end{table}

 
%No ciclo anual mostrado na figura \ref{anual}, observa-se uma clara diferença sazonal na ocorrência de tempestades elétricas entre os dois Hemisférios. Sobre o Hemisfério Norte as tempestades ocorrem principalmente entre os meses de abril e agosto, enquanto no Hemisfério Sul entre setembro e março, apesar das características individuais de cada região como por exemplo dois ou três picos de atividade durante a estação.
%O deslocamento da ZCIT durante o verão setentrional, inverte a estação de tempestades elétricas entre 10$^{\circ}$ Sul e 10$^{\circ}$ Norte, diminuindo o número de sistemas no inverno austral.

Na região entre o clima semi-árido na Argentina e parte da Bacia do Prata, entre 40$^{\circ}$--20$^{\circ}$ Sul e 70$^{\circ}$--60$^{\circ}$ Norte, figura \ref{anual}, local das tempestades mais severas e convecção mais profunda da América do Sul como apontam \citeonline{cecil2005, Romatschke2010}, foi encontrada a estação de tempestades elétricas mais curta e bem definida. Entre maio e agosto a probabilidade de ocorrência de sistemas é praticamente 0\%, enquanto a estação de tempestades elétricas se define entre dezembro e janeiro.%, pois é no verão que o jato de baixos níveis, que traz umidade da Amazônia, se intensifica e dispara os processos de eletrificação nesta região.

As estações de tempestades elétricas se configuram conforme o Sistema de Monção da América do Sul (SAMS)   \sigla{name={SAMS},description={Sistema de Monção da América do Sul}}  \cite{Zhou1998,Marengo2012}. Na região central da América do Sul, observa-se que com o aumento da temperatura da superfície entre julho e setembro, o máximo de precipitação começa a se deslocar do Hemisfério Norte para o Hemisfério Sul e desta forma iniciando a estação chuvosa meridional pela região Oeste da Bacia Amazônica \cite{Marengo2012,grimm2003nino,reboita2010regimes}.

Entre 10$^{\circ}$--0$^{\circ}$ Sul e 80$^{\circ}$--60$^{\circ}$ Norte,  na figura \ref{anual}, observa-se que o pico da estação de tempestades elétricas ocorreu em outubro, nos primeiros passos da estação chuvosa da América do Sul. Porém o máximo de precipitação nesta região ocorre depois de 4 ou 5 meses. 

Em \citeonline{Petersen2001}, o estudo realizado referente a estrutura tridimensional da precipitação observada pelo TRMM sobre a região Central da Amazônia, mostrou que a convecção mais profunda ocorre também na transição do período seco para o chuvoso, exatamente quando começa a reversão sazonal do vento em baixos níveis associado ao SAMS conforme apontam \citeonline{Zhou1998}.

Com o início do verão, o máximo de precipitação caminha até a região Centro Oeste e Sudeste do Brasil. Em janeiro, o SAMS se configura mais ativamente com a atuação da Zona de Convergência do Atlântico Sul (SACZ) \sigla{name={SACZ},description={Zona de Convergência do Atlântico Sul}} e intensificação do Jato de Baixos Níveis (JBN). \sigla{name={JBN},description={Jato de Baixos Níveis}} A atuação do JBN, principalmente nas regiões abaixo de 20$^{\circ}$ Sul, ativa a estação chuvosa e de tempestades elétricas em sincronismo. 

Durante abril e maio, o SAMS vai se desconfigurando e o máximo de chuva começa a retornar para o Hemisfério Norte caminhando de Sudeste para o Nordeste do Brasil e subindo pelo lado Leste da Bacia Amazônica. Neste retorno é que ocorrem os máximos de precipitação em toda a região da Bacia Amazônica, porém o máximo de ocorrência de atividade elétrica ocorreu na vinda da estação chuvosa para o Hemisfério Sul.

Na região Nordeste do Brasil, entre 10$^{\circ}$--0$^{\circ}$ Sul e 40$^{\circ}$--30$^{\circ}$ Norte, o máximo de chuva ocorre juntamente com o máximo de ocorrência de tempestades elétricas, depois da atuação da SACZ no continente.






%As probabilidades de ocorrência mostradas nas figuras \ref{anual} e \ref{diurno} descrevem bem comportamento dos ciclos diurno e anual, porém os valores de probabilidades são tendenciosos nas regiões em que o satélite passou mais tempo observando, conforme é descrito em \ref{metodoPass}. Pois as densidades de probabilidade em cada região de 10 por 10 graus foram obtidas apenas considerando as latitudes e longitudes médias dos sistemas e a hora, minuto, segundo, dia, mês e ano da observação.





\section{DENSIDADES ESPACIAIS}


Considerando o método descrito em \ref{metodoPass}, referente ao cálculo da densidade de tempestades elétricas, equação \ref{dete}, e densidade de raios, equação \ref{defl}, nesta subseção será possível avaliar se as regiões aonde ocorrem o maior número de sistemas, correspondem as regiões com maior número de raios.

Na figura \ref{tempestadestotal}, observa-se que as regiões de máxima ocorrência de tempestades elétricas estão situadas sobre a Colômbia e região central da Bacia Amazônica, abrangendo a parte brasileira, colombiana, venezuelana e peruana.



\begin{figure}[!ht]
 \centering{
  \subfloat[Densidade espacial total de tempestades elétricas.]{\includegraphics[scale=0.88]{img/DensidadeTempestades/densEspacial19982011TotalTempestadesPolyfill} \label{tempestadestotal}}
  \subfloat[Densidade espacial total de raios.]{\includegraphics[scale=0.88]{img/TaxaFlash/densEspacial_19982011totalTaxaFlashPolyfill}\label{raiosTotal}}
  }
\caption{Densidade espacial de tempestades elétricas e raios observados entre 1998 e 2011.}
\label{tempesRaios}
\end{figure}


Mesmo que os sistemas com as maiores taxas de raios no tempo observados pelo TRMM, estejam mais concentrados no Sul da América do Sul conforme mostram \citeonline{cecil2005,zipser2006}, as tempestades elétricas são bem mais frequentes à Noroeste da AS.

A ITCZ e os mecanismos de convergência de umidade e liberação de calor latente/sensível na Floresta Amazônica, associados intimamente com a manutenção da Alta da Bolívia, são os principais propulsores de tempestades elétricas da América do Sul. 

As regiões de máxima ocorrência de sistemas correspondem as regiões de máxima precipitação.


 aonde atua a ZCIT e ocorre grande fluxo de calor da floresta 

As regiões com taxa de flashes superior a 36$km^{-2}ano^{-1}$ na Amazônia, possuem acima de 150 tempestades elétricas $km^{-2}ano^{-1}$. Na região da Argentina e Paraguai, as mesmas taxas de flashes são atingidas, mas com uma taxa de sistemas em torno de 60$km^{-2}ano^{-1}$.  

Na figura \ref{DensidadeTempestadesSazonal}, o view time foi acumulado para o período de cada estação do ano (DJF, MAM, JJA, SON), para os 14 anos de dados, e a taxa de tempestades elétricas e de flashes foram calculadas em concentração por $km^{-2}dia^{-1}$ para cada estação. 

Apesar da figura \ref{diurnoanual} mostrar que em outubro, novembro e dezembro temos maior probabilidade de ocorrência de tempestades elétricas sobre o continente, a figura \ref{tempestadesSON} mostra qual a distribuição espacial desses sistemas e qual a taxa de raios que os sistemas produziram. 



\begin{figure}
  \centering{
  \subfloat[DJF]{{\includegraphics[scale=0.88]{img/DensidadeTempestades/densEspacial19982011djfTempestadesPolyfill}} \label{tempestadesDJF}}
  \subfloat[MAM]{{\includegraphics[scale=0.88]{img/DensidadeTempestades/densEspacial19982011mamTempestadesPolyfill}} \label{tempestadesMAM}}

  \subfloat[JJA]{{\includegraphics[scale=0.88]{img/DensidadeTempestades/densEspacial19982011jjaTempestadesPolyfill}} \label{tempestadesJJA}}
  \subfloat[SON]{{\includegraphics[scale=0.88]{img/DensidadeTempestades/densEspacial19982011sonTempestadesPolyfill}} \label{tempestadesSON}}
  
  }    
  \caption{Densidade espacial sazonal das tempestades elétricas}
\label{DensidadeTempestadesSazonal}
\end{figure} 

\begin{figure}
  \centering{
  \subfloat[DJF]{{\includegraphics[scale=0.88]{img/TaxaFlash/densEspacial_19982011djfTaxaFlashPolyfill}} \label{tempestadesDJF}}
  \subfloat[MAM]{{\includegraphics[scale=0.88]{img/TaxaFlash/densEspacial_19982011mamTaxaFlashPolyfill}} \label{tempestadesMAM}}

  \subfloat[JJA]{{\includegraphics[scale=0.88]{img/TaxaFlash/densEspacial_19982011jjaTaxaFlashPolyfill}} \label{tempestadesJJA}}
  \subfloat[SON]{{\includegraphics[scale=0.88]{img/TaxaFlash/densEspacial_19982011sonTaxaFlashPolyfill}} \label{tempestadesSON}}
  }    
  \caption{Densidade espacial sazonal de raios}
\label{TaxaFlash}
\end{figure} 

\begin{figure}
\centering{\includegraphics[scale=0.88]{img/TaxaFlashTempestade/densEspacial19982011totalEficienciaPolyfill}}  
\caption{Eficiencia de tempestade}
\label{eficiencia}
\end{figure}