\chapter{Marco das Tempestades Elétricas na América do Sul }


O estudo da ocorrência das tempestades elétricas no tempo sobre a América do Sul pode ser observado na figura \ref{diurnoanual} em que mostra o ciclo diurno e anual entre 1998 e 2011 da probabilidade de ocorrência desses sistemas em toda região de estudo entre 10$^{\circ}$ Norte -- 40$^{\circ}$ Sul e 30$^{\circ}$  -- 90$^{\circ}$ Oeste.



No gráfico esquerdo da figura \ref{diurnoanual}, pode-se observar a probabilidade de ocorrência das tempestades elétricas no decorrer das horas solares do dia e dos meses do ano. Observa-se no ciclo diurno que entre 14:00h e 15:00h as tempestades são mais prováveis, mostrando que o aquecimento da superfície do continente e o aumento da camada limite planetária no decorrer do dia são ingredientes que podem aumentar a probabilidade de ocorrência em até 4,6 vezes em relação a horários de menor fluxo de calor sensível para a atmosfera. Enquanto o TRMM observou 2312 tempestades elétricas às 9$h$, às 14$h$ foram observadas 10786 tempestades elétricas sobre o continente.

No ciclo anual da probabilidade de ocorrência, gráfico na parte direita da figura \ref{diurnoanual}, observamos que a estação de tempestades elétricas do continente contempla os meses de outubro, novembro, dezembro, janeiro, fevereiro e março. A maior probabilidade de ocorrência de tempestades elétricas da América do Sul esteve associada ao mês de outubro, que concentrou 16955 tempestades elétricas observadas pelo TRMM. % Entre janeiro e março as tempestades elétricas tiveram probabilidade de 0.3\% menor do que no início da estação, em outubro.   

Na figura \ref{diurno} podemos observar a probabilidade de ocorrência durante as horas, agora em UTC, para cada caixa de 10$^{\circ}$ $\times$ 10$^{\circ}$ que cobre toda região de estudo. Note que os valores de probabilidade são mostrados em porcentagem e foram normalizados pelo total dos 154141 sistemas observados.

% Portanto o número de tempestades elétricas ($N_{TE}$) observado para cada valor de probabilidade ($p$) em cada caixa de 10$^{\circ}$ $\times$ 10$^{\circ}$  pode ser obtido conforme a equação \ref{}.

Mesmo que em uma análise geral mostre a importância do aquecimento superficial do continente para a ocorrência de tempestades com raios, as descargas elétricas noturnas, são observadas com maior frequência em sistemas sobre a região da Colômbia e Venezuela. No gráfico de probabilidade por horas do dia, na figura \ref{diurno}, entre 0$^{\circ}$  -- 10$^{\circ}$ Norte e 70$^{\circ}$ -- 80$^{\circ}$ Oeste, às 5h UTC (00h UTC-5 hora local), possui o valor de probabilidade 0,4, o que representa um número de 616 tempestades elétricas em 14 anos. Essa região possui topografia elevada e deve haver vigorosa circulação de Brisa de Vale e Montanha que combinada com a atuação da Zona de Convergência Intertropical (ZCIT), promove condição para o desenvolvimento de tempestades elétricas noturnas de maneira muito mais eficiente do que em outras regiões da América do Sul. 

No Atlântico Tropical entre 0$^{\circ}$  -- 10$^{\circ}$ Norte e 30$^{\circ}$ -- 50$^{\circ}$ Oeste, tempestades elétricas noturnas impulsionadas pela ZCIT, foram observadas. Porém com baixa probabilidade de ocorrência.

Na região extratropical a passagem de sistemas transientes, topografia elevada principalmente na região da Serra de Córdoba na Argentina, observada no gráfico entre 30$^{\circ}$  -- 40$^{\circ}$ Sul e 60$^{\circ}$ -- 70$^{\circ}$ Oeste e a dinâmica de formação de Sistemas Convectivos de Meso-escala, como mostra \cite{Durkee2009}, faz com que tempestades elétricas se desenvolvam à noite ou iniciam-se durante o dia e perduram durante a madrugada.
 
Quando olhamos para o ciclo anual dos sistemas sobre a superfície do continente, como mostra a figura \ref{anual}, observa-se um grande contraste sazonal entre 0$^{\circ}$  -- 10$^{\circ}$ Norte e 0$^{\circ}$ -- 40$^{\circ}$ Sul. Entre 0$^{\circ}$  -- 10$^{\circ}$ Norte. O aquecimento do Hemisférico Norte no verão combinado com o deslocamento da ZCIT para o Norte no Inverno Austral, inverte a estação de tempestades elétricas em relação as latitudes 0$^{\circ}$ -- 40$^{\circ}$ Sul.

Na figura \ref{anual}, entre 20$^{\circ}$  -- 40$^{\circ}$ Sul e 60$^{\circ}$ -- 70$^{\circ}$ Oeste, local das tempestades mais severas e convecção mais profunda da América do Sul como aponta \cite{cecil2005, Romatschke2010}, possui uma estação de tempestade curta e muito bem definida. Entre maio e agosto praticamente não há raios, pois é no verão que o jato de baixos níveis, que traz umidade da Amazônia, se intensifica e dispara os processos de eletrificação nesta região Sul da AS.

Portanto a probabilidade de ocorrência de sistemas observados pelo TRMM, depende da quantidade de tempo em que o satélite ficou observando o continente. Fazendo o acumulado do view time do LIS para os 14 anos, como mostra a figura \ref{vt}, observa-se que algumas regiões ao sul da AS, o satélite permanece 10 dias a mais do que em regiões ao norte.

As probabilidades de ocorrência mostradas nas figuras \ref{anual} e \ref{diurno} descrevem bem comportamento dos ciclos diurno e anual, porém os valores de probabilidades são tendenciosos nas regiões em que o satélite passou mais tempo observando.

Na figura \ref{total}, temos na esquerda o acumulado de todos os pontos do VIRS que definiram a área dos sistemas e na parte direita temos o acumulado de todos os flashes contidos nos sistemas identificados. Ambas as concentrações, seja de área de tempestades elétricas ou de ocorrência de flashes foram normalizados pela área da grade e view time. O view time apresentado em dias na figura \ref{view} foi convertido para ano, e as taxas calculadas em concentração por $km^{-2}ano^{-1}$.

Observa-se (figura \ref{total}) um máximo em área de ocorrência sobre a Colômbia e região central da Floresta Amazônica. Mesmo que os sistemas com maior número de raios observados pelo TRMM estejam mais concentrados no Sul da América do Sul, as tempestades elétricas são observadas com maior frequência à Noroeste da AS. 

As regiões com taxa de flashes superior a 36$km^{-2}ano^{-1}$ na Amazônia, possuem acima de 150 tempestades elétricas $km^{-2}ano^{-1}$. Na região da Argentina e Paraguai, as mesmas taxas de flashes são atingidas, mas com uma taxa de sistemas em torno de 60$km^{-2}ano^{-1}$.  

Na figura \ref{estacao}, o view time foi acumulado para o período de cada estação do ano (DJF, MAM, JJA, SON), para os 14 anos de dados, e a taxa de tempestades elétricas e de flashes foram calculadas em concentração por $km^{-2}dia^{-1}$ para cada estação. 

Apesar da figura \ref{diurnoanual} mostrar que em outubro, novembro e dezembro temos maior probabilidade de ocorrência de tempestades elétricas sobre o continente, a figura \ref{son} mostra qual a distribuição espacial desses sistemas e qual a taxa de raios que os sistemas produziram. 


\begin{figure}
\centering{\includegraphics[scale=0.88]{img/DensidadeTempestades/densidade_espacial_1998-2011_total_virsgrid_polyfill}}  
\caption{Densidade espacial total de tempestades elétricas.}
\label{tempestadesTotal}
\end{figure}


\begin{figure}
  \centering{
  \subfloat[DJF]{{\includegraphics[scale=0.88]{img/DensidadeTempestades/densidade_espacial_1998-2011_djf_virsgrid_polyfill}} \label{tempestadesDJF}}
  \subfloat[MAM]{{\includegraphics[scale=0.88]{img/DensidadeTempestades/densidade_espacial_1998-2011_mam_virsgrid_polyfill}} \label{tempestadesMAM}}

  \subfloat[JJA]{{\includegraphics[scale=0.88]{img/DensidadeTempestades/densidade_espacial_1998-2011_jja_virsgrid_polyfill}} \label{tempestadesJJA}}
  \subfloat[SON]{{\includegraphics[scale=0.88]{img/DensidadeTempestades/densidade_espacial_1998-2011_son_virsgrid_polyfill}} \label{tempestadesSON}}
  
  }    
  \caption{Densidade espacial sazonal das tempestades elétricas}
\label{DensidadeTempestades}
\end{figure} 







\begin{figure}
\centering{\includegraphics[scale=0.88]{img/TaxaFlash/densidade_espacial_1998-2011_total_TaxaFlash_polyfill}}  
\caption{Densidade espacial total de raios}
\label{tempestadesTotal}
\end{figure}


\begin{figure}
  \centering{
  \subfloat[DJF]{{\includegraphics[scale=0.88]{img/TaxaFlash/densidade_espacial_1998-2011_djf_TaxaFlash_polyfill}} \label{tempestadesDJF}}
  \subfloat[MAM]{{\includegraphics[scale=0.88]{img/TaxaFlash/densidade_espacial_1998-2011_mam_TaxaFlash_polyfill}} \label{tempestadesMAM}}

  \subfloat[JJA]{{\includegraphics[scale=0.88]{img/TaxaFlash/densidade_espacial_1998-2011_jja_TaxaFlash_polyfill}} \label{tempestadesJJA}}
  \subfloat[SON]{{\includegraphics[scale=0.88]{img/TaxaFlash/densidade_espacial_1998-2011_son_TaxaFlash_polyfill}} \label{tempestadesSON}}
  
  }    
  \caption{Densidade espacial sazonal de raios}
\label{TaxaFlash}
\end{figure} 






\begin{figure}
\centering{\includegraphics[scale=0.88]{img/TaxaFlashTempestade/densidade_espacial_1998-2011_densidade_total_TaxaFlashTempestades_polyfill}}  
\caption{Densidade espacial total da eficiência das tempestades elétricas.}
\label{tempestadesTotal}
\end{figure}


\begin{figure}
  \centering{
  \subfloat[DJF]{{\includegraphics[scale=0.88]{img/TaxaFlashTempestade/densidade_espacial_1998-2011_densidade_djf_TaxaFlashTempestades_polyfill}} \label{eficienciaDJF}}
  \subfloat[MAM]{{\includegraphics[scale=0.88]{img/TaxaFlashTempestade/densidade_espacial_1998-2011_densidade_mam_TaxaFlashTempestades_polyfill}} \label{eficienciaMAM}}

  \subfloat[JJA]{{\includegraphics[scale=0.88]{img/TaxaFlashTempestade/densidade_espacial_1998-2011_densidade_jja_TaxaFlashTempestades_polyfill}} \label{eficienciaJJA}}
  \subfloat[SON]{{\includegraphics[scale=0.88]{img/TaxaFlashTempestade/densidade_espacial_1998-2011_densidade_son_TaxaFlashTempestades_polyfill}} \label{eficienciaSON}}
  
  }    
  \caption{Densidade espacial sazonal da eficiência das tempestades elétricas}
\label{TaxaFlashTempestade}
\end{figure} 









\begin{figure}
\centering{\includegraphics[scale=1.1]{img/ciclos/ciclodiurnoanual19982011_total}}  
\caption{Ciclo}
\label{diurnoanual}
\end{figure}


\begin{figure}
\centering{\includegraphics[scale=1.3]{img/ciclos/ciclo_diurno_19982011total_nuvem_sempr}}  
\caption{Ciclo}
\label{diurno}
\end{figure}

\begin{figure}
\centering{\includegraphics[scale=1.3]{img/ciclos/ciclo_anual_19982011total_nuvem_sempr}}  
\caption{Ciclo}
\label{anual}
\end{figure}





