\chapter{CONCLUSÃO}

Cria-se uma metodologia para caracterizar as tempestades elétricas observadas na AS a partir das medidas integradas dos sensores LIS, PR e VIRS a bordo do TRMM, durante o período de janeiro de 1998 a dezembro de 2011. As tempestades elétricas foram definidas por pixeis contíguos com $T_b$ $\leq$ 258 K do canal 4 do VIRS com pelo menos um raio do LIS. Para tempestades elétricas consideras enormes, foi utilizado o limiar de $T_b$ $\leq$ 221 K, pois as mesmas faziam parte de sistemas frontais ou ZCAS. A partir deste procedimento foi criado um banco de dados de 157~592 tempestades elétricas do TRMM que são compostos basicamente de: distribuição de $T_b$ e respetivas latitude e longitudes do VIRS, perfis verticais de $Z_c$, classificação convectiva, estratiforme e outros, taxa de precipitação na superfície e respectivas latitudes e longitudes, localização (latitude e longitude) dos eventos, grupos e raios,  e tempo de visada com resolução de 0,25$^{\circ}$ $\times$ 0,25$^{\circ}$.

Com base neste subconjunto de dados de tempestades elétricas do TRMM, foi avaliado o Marco das tempestades elétricas na AS, a partir do estudo do ciclo diurno, ciclo anual, distribuição geográfica de densidades de raios e tempestades elétricas.


Para estudar a severidade das tempestades elétricas foram utilizados dois índices de taxas de raios, uma vez que na literatura os extremos são avaliados em termos da taxa de raios no tempo, mas não avaliam a eficiência da produção de raios. Portanto foi utilizado a taxa de raios no tempo -- FT -- e a taxa de raios no tempo normalizado pela área do sistema -- FTA. O FT é comumente utilizado  para prognosticar a ocorrência de granizo e tornados \cite{williams1999,goodman1988,schultz2011,gatlin2010}. Já o FTA foi pensado com o intuito de analisar a eficiência da tempestade em produzir raios. 

Os processos microfísicos que levam as tempestades elétricas a terem mais ou menos raios, foram investigados utilizando os CFADs \cite{yuter1995}. Entretanto, a América do Sul cobre uma vasta extensão territorial que vai do equador até os sub-trópicos e a altura das isotermas podem variar significativamente e como os processos de eletrificação de nuvens  dependem essencialmente da temperatura \cite{Takahashi1978}, faz-se necessário converter a base de altura dos CFADs para temperaturas. Logo criou-se o diagrama CFTD que proporciona uma compreensão a respeito das mudanças de $Z_c$ em função da temperatura do perfil atmosférico, o que auxiliou na identificação de água super-resfriada, cristais de gelo, agregados, \textit{graupel} e granizo, que são responsáveis pela transferência de cargas dentro das nuvens \cite{Takahashi1978,saunders2008}.


\section{MARCO DAS TEMPESTADES ELÉTRICAS}

No o ciclo diurno das tempestades elétricas observadas sobre a AS, observa-se que 40\% dos sistemas ocorreram entre 13h e 17h (HL), mostrando que o aquecimento diurno e o aumento da camada limite planetária no decorrer do dia, aumenta a probabilidade de ocorrência em 5,4 vezes (às 14h HL) em relação ao horário de menor ocorrência, às 9h HL.

No continente observa-se um predomínio da atividade elétrica entre às 13h e 17h, enquanto que no oceano existe uma variação dependendo da proximidade com o continente, mas em geral, sobre o oceano, observa-se dois máximos: um por volta das 20h e outro entre 4--5h. Quando  próximo do continente o pico das 20h se desloca para às 15h.  

Apesar das tempestades elétricas sobre a AS possuírem um ciclo diurno bem definido, entre 0$^{\circ}$--10$^{\circ}$ Norte e 80$^{\circ}$--70$^{\circ}$ Oeste, observou-se a maior probabilidade ($\simeq$0,4\%) de tempestades elétricas noturnas da América do Sul, entre 0h e 00:59h. A circulação de vale e montanha associada com a topografia elevada na Colômbia, principalmente a região do Parque Nacional Natural Paramillo, e o Lago Maracaibo na Venezuela, e a atuação da Zona de Convergência Intertropical (ZCIT), promovem condições para o desenvolvimento de tempestades elétricas noturnas de maneira mais eficiente do que as demais regiões da AS \cite{burgesser2012}.

Sobre o oceano costeiro, temos que entre 0$^{\circ}$--10$^{\circ}$ Norte e 90$^{\circ}$--80$^{\circ}$ Oeste (Panamá e Sul da Costa Rica,  Oceano Pacífico que engloba o Parque Nacional da Ilha do Coco na Costa Rica, incluindo ilhas Galápagos no Equador) foi observado a maior probabilidade de ocorrência de tempestades elétricas da AS, com pico de ocorrência às 4h e outro às 14h. Possivelmente o aquecimento superficial durante o dia e as trocas de energia na forma de calor entre o oceano e a atmosfera explicam esta distribuição bimodal.

A maior atividade horária de tempestades elétricas ( 0,8\%), ocorreu entre 10$^{\circ}$--0$^{\circ}$ Sul e 70$^{\circ}$--50$^{\circ}$ Oeste e 20$^{\circ}$--10$^{\circ}$ Sul e 60$^{\circ}$--50$^{\circ}$ Oeste, durante às 14h e 16h.

Apesar dos SCMs apresentarem um ciclo diurno com atividade noturna \cite{Velasco1987, Durkee2009, machado1998}, entre 30$^{\circ}$--20$^{\circ}$ Sul e 60$^{\circ}$--50$^{\circ}$ Oeste, as tempestades elétricas apresentaram um máximo por volta das 15h.


%\section{CICLO ANUAL}

No ciclo anual, a estação de tempestades elétricas na América do Sul se configura entre outubro e março e possui dois picos: janeiro, durante o verão austral; e outubro, período de transição entre a estação seca e chuvosa, quando se observa a maior atividade de tempestades elétricas. 

Distribuições bimodais de atividade elétrica, março e outubro, estão restritas a parte Norte da AS entre 0-10N e 80-70 Oeste e entre a Amazônia, definidas pelas regiões 10S-0 e 80-50 Oeste. 

A parte Nordeste da AS, entre 0-10 Norte e 70-50 Oeste, é marcada por um máximo em agosto, verão do hemisfério Norte. No sul da AS,  entre 40$^{\circ}$--20$^{\circ}$ Sul e 70$^{\circ}$--60$^{\circ}$ Oeste, região da Argentina e Bacia do Prata, foi encontrada a estação de tempestades elétricas mais curta (2 meses), e uma estação quase sem raios durante o inverno austral.

A região com a estação de maior duração de tempestades elétricas,  9 meses (março a novembro), foi sobre a Colômbia e parte Oeste da Venezuela que abrange até o lago Maracaibo.  


%\section{DISTRIBUIÇÃO GEOGRÁFICA}

Referente as distribuição de densidades geográficas, as maiores densidades de tempestades elétricas situam-se na parte Norte e Noroeste da AS, ou seja, na Colômbia associado ao extremo Norte da Cordilheira dos Andes e ao Norte/Noroeste da Floresta Amazônica respectivamente, com valores entre 3,5-4,7 $\times$ 10$^{-4}$ km$^{-2}$. Além disso observa-se uma vasta região com densidades superiores a  2,5 $\times$ 10$^{-4}$ km$^{-2}$, que conta com regiões de topografia elevada como à Noroeste do Lago Titicaca no Peru e algumas regiões do Planalto Brasileiro como sobre a Serra Catarinense e o Parque Nacional das Emas ao Sudoeste de Goias, além de grande parte da região Amazônica, Colômbia, Venezuela e Panamá.  

Em termos de sazonabilidade temos que a primavera apresenta a maior atividade de tempestades elétricas (57~861) seguidas pelo verão (46~077), outono (36~804) e inverno (16~850). 

Durante o verão a máxima atividade é encontrada ao Sul da Amazônia se estendendo pela parte central e Sudeste do Brasil, além da cordilheira dos Andes abrangendo o Peru e Bolívia. No Outono a maior atividade se concentra no litoral do Maranhão e Pará, Sudeste e Noroeste  da Colômbia. No inverno a máxima atividade  fica restrita na região costeira da Colômbia e Panamá. Finalmente na primavera observa-se o máximo se estendendo do Panamá, Colômbia, Venezuela, Sul da Venezuela, Noroeste e centro da Amazônia e Nordeste da Bolívia. 
 
As maiores densidade de raios por ano por quilômetro quadrado sobre a AS são observadas: na Foz do Rio Catatumbo na Venezuela com 371,2 raios ano$^{-1}$ km$^{-2}$ e em Cochabamba na Bolívia com $\simeq$60  raios ano$^{-1}$ km$^{-2}$. Regiões com densidades de raios entre 30-60 raios ano$^{-1}$ km$^{-2}$, correspondem ao Pico das Agulhas Negras na Serra da Mantiqueira entre Minas Gerais e o Rio de Janeiro, Pico da Neblina no Amazonas, parte central da Bacia do Prata, na cordilheira dos Andes do Peru, e extremo Norte da Cordilheira dos Andes sobre a Colômbia. 

Em termos sazonais, os extremos de densidade de raios ficam restritos ao inverno e primavera. Sendo que na primavera temos as maiores densidades correspondentes aos máximos anuais, exceto Pico das Agulhas Negras e parte central da Bacia do Prata. Já no inverno basicamente sobre o extremo Norte da Cordilheira dos Andes na Colômbia e no lago Maracaibo na Venezuela. As regiões do Pico das Agulhas Negras apresenta alta atividade durante o verão e primavera, já  a parte central da Bacia do Prata, apresenta atividade elétrica  no verão, outono e primavera.


Referente as regiões mais eficientes em termos de densidades de raios por tempestades, temos que a região do Lago Maracaibo, na Foz do Rio Catatumbo (pixel com 772 km$^{2}$) se mostra a mais eficiente da AS, atingindo o valor máximo de 11,73 $\times$ 10$^{-2}$ ano$^{-1}$ km$^{-2}$, que representa {114 333} raios ano$^{-1}$, seguida da Bacia do Prata e Serra de Córdoba da Argentina com valores $\simeq$ 5,5 $\times$ 10$^{-2}$ ano$^{-1}$ km$^{-2}$.


\section{TEMPESTADES ELÉTRICAS SEVERAS}

As tempestades elétricas severas, ou, as mais intensas e raras, foram definidas a partir do limiar de 90\% da distribuição de frequência dos índices FT e FTA, que tiveram pelo menos um perfil vertical do PR com chuva válida e $VT_m$ maior do que 60 segundos  definindo um grupo de 9475 tempestades elétricas extremas de FTA com valores entre 29,3--1258 $\times$ 10$^{-4}$ raios min$^{-1}$ km$^{-2}$, e mais 9475 tempestades elétricas extremas de FT com os valores entre 47,2--1283,6 raios min$^{-1}$.  Estes valores de FT são consistentes com os trabalhos de \citeonline{cecil2005}, que encontraram estes valores nos \textit{top} 0,01\% das PFs (categoria 3, 4, 5).


As tempestades elétricas extremas de FTA tem 72\% de fração convectiva e 32\% de fração estratiforme, enquanto as tempestades elétricas extremas de FT 20\% de fração convectiva e 75\% de fração estratiforme. Estas características indicam que os extremos de FTA correspondem a sistemas novos ou em via de maturação, enquanto os sistemas extremos de FT correspondem a sistemas maduros ou em fase de decaimento \cite{learyHouse1979}. 

O estudo da precipitação tridimensional por meio dos CFADs e CCFADs verifica-se que os perfis com raios possuem maiores valores de  $Z_c$ e são mais profundos na atmosfera. Entre 5--7 km de altitude os valores de $Z_c$ para a precipitação com raios atingem valores entre 5 dBZ e 10 dBZ superiores do que para a precipitação sem raios, tanto para os extremos de FTA quanto FT. As regiões sem raios dos eventos extremos de FT são mais estratiformes do que para os extremos de FTA e apresentam forte assinatura da banda brilhante. Os pixeis sem raios da precipitação dos extremos de FTA não há assinatura da banda brilhante nos sistemas sobre o continente, mas há banda brilhante sobre o oceano. Nas regiões com raios, as tempestades elétricas dos extremos de FTA possuem entre 1-3 dBZ a mais do que os valores de $Z_c$ dos extremos de FT, especialmente acima de 5 km de altitude. Do ponto de vista da intensidade convectiva, ambos FTA e FT, são mais vigorosos do equador para os sub-trópicos. 


Com a criação dos CFTDs e CCFTDs identifica-se que durante o desenvolvimento vertical, a intensidade convectiva pode ser mensurada avaliando a redução ($\simeq$7 dBZ) de $Z_c$ devido ao congelamento dos hidrometeoros acima da isoterma de 0 $^{\circ}$C. A condição de tempo severo, deve estar associada a um maior caminho de temperatura para o congelamento dos hidrometeoros. Com o aumento da intensidade da corrente ascendente há uma aumento da  concentração de gotas pequenas \cite{bigg1953}, o que aumenta a concentração de água super-resfriada e a espessura da camada fria de nuvem e consequentemente expõe os hidrometeoros a um número maior de colisões, gerando maior concentração e diversidade de hidrometeoros\footnote{Gotas de água super-resfriadas, cristais de gelo de diferente formas geométricas, agregados e flocos de neve, \textit{graupel} e granizo com diferentes tamanhos e densidades}, envigorando os processos de eletrificação das nuvens.  

A probabilidade de perfis de $Z_c$ por nível de temperatura (CFTD), apresenta quantis mais largos para os extremos de FTA do que nos extremos de FT. Esse efeito demonstra a maior diversidade de hidrometeoros no ambiente das tempestades elétricas extremas de FTA.    

%Isso implica que nas FTAs temos uma maior camada mista que possibilita maior carregamento ....
%Diferenças entre trópicas e sub-trópicos. Com e sem. 
%Falar das diferenças em termo de temperatura.
%Taxas de variação de $Z_c$ por temperatura.
%Combinando as observações de raios com a análise dos processos microfísicos de crescimentos dos hidrometeoros de nuvem com base na estrutura tridimensional da precipitação, 

A convecção mais intensa sobre a AS consiste em um fenômeno atmosférico capaz de processar a energia potencial convectiva em uma área entre 50-400 km$^{2}$, quando há extremos de FTA, que ocorre nos estágios iniciais das tempestades elétricas. Tempestades elétricas com FT extremo, consistem em um fenômeno  atmosférico com dimensões de $\simeq$50~000 km$^{2}$. As condições de tempo severo que são associada com a intensidade da corrente ascendente, devem ocorrer nas tempestades elétricas extremas de FT, porém quando há novas células convectivas (50-400 km$^{2}$) com FTA extremo, embebidas na extensão das tempestades elétricas maiores. Tempestades elétricas com FT extremo, podem possuir grande número de raios, porém distantes entre si. Nestes casos, temos tempestades grandes e com núcleos de raios pouco eficientes, que provavelmente não estão associadas com fortes correntes ascendentes, ou seja, com condições de tempo severo.


As tempestades elétricas associadas com condições de tempo severo devem estar contidas tanto no grupo dos valores extremos de FT quanto de FTA. A intersecção dos dois grupos de extremos indicará as regiões de máxima severidade. %Nesse sentido temos o Planalto Meridional Brasileiro e no semi-árido Argentino a Leste da Cordilheira dos Andes e da Serra de Córdoba.

Comparando as regiões da Bacia Amazônica (10S-0 e 70W-60W) com a região das tempestades mais severas sobre a AS, na Bacia do Prata (30S-20S e 60W-50W), observa-se que para o 95\textsuperscript{\underline{o}} percentil do CCFTD, entre 0 $^{\circ}$C e -15 $^{\circ}$C, as tempestades elétricas extremas de FTA na região Amazônica decrescem 5 dBZ enquanto que as tempestades elétricas extremas de FT decrescem 10 dBZ e na Bacia do Prata, FTA decresce também 5 dBZ, enquanto FT decresce apenas 6 dBZ. A máxima taxa de decrescimento de $Z_c$ em função da temperatura (50 \textsuperscript{\underline{o}} percentil) é de $\simeq$-1 dBZ $^{\circ}$C$^{-1}$ para a Amazônia e $\simeq$-0,85 dBZ $^{\circ}$C$^{-1}$ para a Argentina.


As tempestades elétricas com os maiores (95 \textsuperscript{\underline{o}} percentil) valores de FTA, são encontradas sobre uma vasta região da AS, com valores de 52,76 $\times$ 10$^{-4}$ raios min$^{-1}$ km$^{-2}$. Valores de FTA superiores a 84 $\times$ 10$^{-4}$ raios min$^{-1}$ km$^{-2}$ são encontrados ao Sul e Leste da Bacia Amazônica, parte central e Sudeste do Brasil, Sul do Peru, Bolívia, Paraguai, Oeste do Uruguai, Norte e centro da Argentina. Ao considerar o 99\textsuperscript{\underline{o}} percentil da amostragem de FTA, as tempestades elétricas passam a produzir 148,93 raios min$^{-1}$ km$^{-2}$, associadas a regiões de topografia elevada, principalmente entre o Pantanal Mato-grossense e o Planalto Central Brasileiro, entre as Bacias dos Rios Xingu, Araguaia e Tocantis, região do Planalto Meridional Brasileiro aonde está localizado os planaltos e chapadas da Bacia do Paraná e região Leste da Serra de Córdoba.


Os maiores valores do 95\textsuperscript{\underline{o}} ($\geq$ 92,84 raios min$^{-1}$) e 99\textsuperscript{\underline{o}} ($\geq$ 272,28 raios min$^{-1}$) percentil do índice FT ficam situados na região Sul da América do Sul.




\section{IMPLICAÇÕES PARA A ELETRIFICAÇÃO DAS NUVENS}



Como os hidrometeoros dos sistemas extremos de FTA crescem em regiões mais frias do que nos extremos de FT,
conforme \citeonline{Takahashi1978}, podemos considerar que os núcleos de raios das tempestades elétricas com extremo de FTA entre 10-0S e 70-60W, possuíram centros de cargas predominantemente negativos, enquanto que os núcleos de raios das tempestades elétricas dos extremos de FT, em que as maiores taxas de crescimento de $Z_c$ ocorreram entre -8\textsuperscript{o}C e 0\textsuperscript{o}C, os centros de cargas foram predominantemente positivos.

%exceto no 95\textsuperscript{\underline{o}} percentil em que o maior acréscimo de $Z_c$ foi em -10\textsuperscript{o}C
Apensar do crescimento do \textit{graupel} e o granizo em temperaturas entre -10\textsuperscript{o}C e 0\textsuperscript{o}C indicar carregamento positivo, as variações do conteúdo de água líquida de nuvem  podem inverter a polaridade das partículas que crescem em temperatura inferior a -10\textsuperscript{o}C \cite{Takahashi1978}. Durante uma tempestade, as correntes ascendentes e o entranhamento na região de crescimento do gelo poderão elevar ou diminuir o conteúdo de água líquida do ambiente.

Também, a taxa de coleta de gotículas de água líquida pelo \textit{graupel} durante o processo de acreção, pode ter mais influência no sinal das cargas dos hidrometeoros do que o conteúdo de água líquida \cite{jayaratne1983,saunders1991effect,brooks1997,Takahashi2002}. 
Apesar do conteúdo de água líquida efetivamente coletado no processo de acresção ser proporcional ao conteúdo de água líquida da nuvem, as velocidades relativas entre as partículas e a eficiência de coleta, além de influenciarem sobre a polaridade, também determinam a estrutura do granizo: poroso, compacto ou esponjoso \cite[p.~335]{mason1971_2ed}.  
 
Mesmo considerando hipoteticamente que entre 10-0S e 70-60W, as tempestades elétricas com extremos de FT possuíram maior probabilidade de centros de cargas principal positivos, a porcentagem de raios nuvem-solo  positivos pode ser menor do que a porcentagem de raios nuvem-solo negativos. Em \citeonline{carey2007}, tempestades elétricas com 25\% dos raios nuvem-solo positivos foram consideradas como tempestades elétricas positivas.

Em \citeonline{fernandes2005} e \citeonline{fernandes2006}, descreve-se que no ambiente amazônico seco e poluído, as nuvens podem ter altura da base mais elevadas e mesmo que os processos de crescimento de hidrometeoros ocorram em regiões de maior altitude e consequentemente de menor temperatura, o centro de carga principal fica mais distante da superfície, causando  o aumento da razão entre os raios intranuvens e nuvem-solo e favorecendo a ocorrência de raios nuvem-solo positivos a partir do centro de carga positivo na base da nuvem, que de acordo com o modelo do tripolo eletrostático proposto em \citeonline{williams1989}, fica mais próximo do solo do que o centro de cargas negativo principal.

%como sugerem os extremos de FTA em relação aos extremos de FT,
%A partir do banco de dados gerado no experimento CHUVA como descrevem \cite{machado2014chuva}, serão possível análises das taxa de crescimento de dBZ~\textsuperscript{o}C, com base em observações por radar, radio-sondagem, conjuntamente com as observações de raios e de campo eletrostático. Desta forma pode-se investigar se o processo microfísico    

Além  dos fatores que podem aumentar a taxa de acresção como: cisalhamento do vento próximo a superfície, aumento da instabilidade condicional e menor espessura entre a base da nuvem e a isoterma de 0\textsuperscript{o}C, que em \citeonline{carey2007,albrecht2011} estiveram associados a ocorrência de tempestades elétricas positivas as quais supostamente teriam centro de carga principal positivo, portanto um tripolo eletrostático invertido, a geometria e densidade volumétrica dos centros de cargas em relação a superfície pode determinar o caminho de menor resistência elétrica para a formação de um raio nuvem-solo positivo ou negativo.

