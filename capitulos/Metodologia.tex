\chapter{METODOLOGIA}
\label{metodologia}

A Metodologia consiste fundamentalmente na construção de um subconjunto de dados das observações dos sensores VIRS, LIS e PR abordo do satélite TRMM, durante o período entre 1998 e 2011. 

Além dos dados satelitais, as reanálises II do NCEP \sigla{name={NCEP},description={\textit{National Centers for Environmental Prediction}}} em níveis de pressão foram utilizadas para identificar os valores de temperatura nos níveis de altitude do PR.

As informações dos diferentes sensores foram combinadas de maneira a identificar sistemas denominados como tempestades elétricas, definidas como nuvens as quais possuem pelo menos um \textit{flash} detectado pelo LIS. 

%A seguir são apresentadas as principais características do TRMM 
%Para melhor entender as implicações que envolvem a construção de uma base de dados de sistemas individualmente a partir das observações do TRMM, inicialmente descreve-se algumas das principais características operacionais do satélite TRMM.

\section{O SATÉLITE TRMM}
\label{metodologiaTRMM}

O satélite \textit{Tropical Rainfall Measuring Mission} -- TRMM \sigla{name={TRMM},description={\textit{Tropical Rainfall Measuring Mission}}} faz parte de uma missão conjunta entre a \textit{National Aeronautics and Space Administration} -- NASA e \sigla{name={NASA},description={\textit{National Aeronautics and Space Administration}}} a \textit{Japan Aerospace Exploration Agency} -- JAXA \sigla{name={JAXA},description={\textit{Japan Aerospace Exploration Agency}}}, com o objetivo de determinar a distribuição geográfica e a variabilidade da chuva e do fluxo de calor latente de condensação para a região tropical e subtropical terrestre, informações fundamentais, por exemplo, nas inicializações de modelos globais e climáticos, principalmente quando se trata de previsão do tempo e clima nos trópicos \cite{kummerok1998,simpson1988}.
  
Os instrumentos a bordo do TRMM são: radar de precipitação (PR), radiômetro de microondas (TMI), radiômetro no visível e no infravermelho (VIRS), sistema de energia radiante da terra e das nuvens (CERES) e sensor para imageamento de relâmpagos (LIS) \cite{kummerok1998}.

Esse satélite possui uma órbita de aproximadamente 320 km de altura e inclinação de 30$^{\circ}$-35$^{\circ}$ para que possa visitar uma mesma região duas vezes ao dia, em horários distintos, sobre a região tropical do planeta Terra \cite{simpson1988}.  

O satélite foi lançado em 28 novembro de 1997 e continua em operação.
 

\subsection{Radar de Precipitação}

O \textit{Precipitation Radar} -- PR é o principal instrumento do TRMM. Se trata do primeiro radar meteorológico lançado no espaço sendo a maior inovação apresentada pela missão TRMM. 

Os maiores objetivos do PR são: prover a estrutura tridimensional da precipitação; e quantificar as taxas de precipitação sobre os continentes e oceanos, ambos objetivos focados na região tropical. 

A antena do PR possui uma largura de feixe de 0.71$^{\circ}$ $\times$ 0.71$^{\circ}$, abertura de 2.0 m $\times$ 2.0 m e faz uma varredura de $\pm$17$^{\circ}$ com 49 feixes, proporcionando a observação de uma faixa na superfície de $\simeq$215 km. A resolução horizontal no nadir é de 4.3 $\pm$ 0.12 km, e de até 5 km nos ângulos próximos dos $\pm$17$^{\circ}$. Verticalmente, o PR observa uma faixa de 20 km a partir da superfície, com resolução de 250 m. Portanto, a precipitação observada corresponde a uma matriz bidimensional de 49 $\times$ 80 posições: 49 feixes na varredura horizontal, com 80 níveis verticais de 250 m de altitude. Estas principais características operacionais do PR são representadas na figura \ref{prtrmm} \cite{kummerok1998}.

\begin{figure}[!hb]
  \centering{
  {{\includegraphics[height=11.1cm]{img/TRMM/PRTRMM}}}
  }
\caption{Principais características das observações do PR \cite{trmmhandbook}}
\label{prtrmm} 
\end{figure} 

Para esta pesquisa serão utilizados os dados do produto 2A25, o qual corrige a atenuação da refletividade do radar medida ($Z_m$) \simbolo{name={$Z_m$},description={Refletividade do radar medida}} e a partir do fator de refletividade corrigido por atenuação $Z_c$, estima a estrutura tridimensional da precipitação no instante da observação, bem como a taxa de precipitação em cada célula da resolução (46 $\times$ 80) do PR \cite{PRv7}. 

Além da estrutura tridimensional da precipitação do produto 2A25, nesta tese, utiliza-se das classificações do tipo de chuva do produto 2A23: convectivo, estratiforme, outros, etc \cite{2A25,PRv7}.
      
\sigla{name={PR},description={\textit{Precipitation Radar}}}

\subsection{Imageador de relâmpagos}

O \textit{Lightning Imaging Sensor} -- LIS  é um sensor óptico capaz de detectar e localizar raios individualmente, a partir da emissão óptica resultante da dissociação, excitação e recombinação dos constituintes atmosféricos durante uma descarga atmosférica. 

O sistema de imagens do LIS é constituído por um telescópio f/1.6 que expande o feixe luminoso observado, que passa por um filtro de interferência no comprimento de onda de 777.4 nm e largura de banda de 1 nm e atinge uma matriz de 128 $\times$ 128 CCDs\footnote{Um dos dispositivos eletrônicos utilizados para registro de imagens com base no efeito fotoelétrico.}. O LIS também possui uma lente angular que proporciona um campo de visão panorâmico sobre as tempestades elétricas sobrevoadas de 80$^{\circ}$ $\times$ 80$^{\circ}$, que corresponde a uma área de 580 km $\times$ 580 km no nadir. Um pixel do campo de visão do LIS, possui resolução entre 5-10km. O sistema de amostragem de dados do LIS garante o registro de 500 imagens por segundo. \cite{christian2000LISalgorithm,boccippio1996science,trmmhandbook}. 

Conforme é descrito por \citeonline{christian2000LISalgorithm}, a identificação dos raios/\textit{flashes} depende do brilho difuso e transiente observado no topo das tempestades elétricas. Dependendo da posição em que as CCDs são sensibilizadas e do intervalo de tempo entre os brilhos subsequentes, o algoritmo de processamento de imagens do LIS identifica os \textit{events}, \textit{groups} e os \textit{flashes}. Os \textit{flashes} são associados aos raios, sejam nuvem-solo ou intranuvem. Os \textit{events} e \textit{groups} estão associados os processos quem envolvem a formação das descargas de retorno, por exemplo, os processos de recuo e conexão dos líderes percursores das correntes elétricas dos raios.  

A figura \ref{LisImagemProcessa}, ilustra como é feita a identificação dos  \textit{events}, \textit{groups} até a caracterização de um \textit{flash}. Observe, que após 700 milissegundos, como mostra a sequência de figuras \ref{evgrfla}-\ref{evgrfle}, temos 4 \textit{flashes} (A, B, C e D),  8 \textit{groups} (a, b, c, d, e, f, g e h) e 14 \textit{events} \cite{christian2000LISalgorithm}.


O LIS possui a capacidade de identificar descargas nuvem-solo e intranuvens, tanto no período diurno quanto noturno.  Com sua velocidade orbital de 11 km/s, o sensor LIS possui um campo de visão que permite a observação de um ponto na Terra por 80-90 segundos, tempo suficiente para a estimativa da taxa de raios de uma tempestade no momento da observação \cite{christianTM,trmmhandbook}.
\sigla{name={LIS},description={\textit{Lightning Imaging Sensor}}}


\begin{figure}[!ht]
  \centering{
  \subfloat[]{{\includegraphics[height=3.3cm]{img/TRMM/time0}} \label{evgrfla}}
  \subfloat[]{{\includegraphics[height=3.3cm]{img/TRMM/time100}}}
  
  \subfloat[]{{\includegraphics[height=3.3cm]{img/TRMM/time350}}}
  \subfloat[]{{\includegraphics[height=3.3cm]{img/TRMM/time400}}}

  \subfloat[]{{\includegraphics[height=3.3cm]{img/TRMM/time700}} \label{evgrfle}}  
  }
\caption{Ilustração do algoritmo de identificação de \textit{events}, \textit{groups} e os \textit{flashes}  do LIS \cite{christian2000LISalgorithm}.}
\label{LisImagemProcessa} 
\end{figure} 



\subsection{Radiômetro no visível e infravermelho}

O \textit{Visible and InfraRed Scanner} -- VIRS é um radiômetro de varredura biaxial, um eixo com $\pm$45$^{\circ}$ e outro 360$^{\circ}$, que mede a radiância em 5 bandas espectrais: com comprimentos de onda de 0.63 $\mu$m e  1.61 $\mu$m, que correspondem a faixa do visível; 3.75 $\mu$m, 10.8 $\mu$m e 12 $\mu$m, na faixa do infravermelho para o monitoramento do vapor d'água e temperaturas de topo de nuvens. 

Com uma inclinação $\pm$45$^{\circ}$ na horizontal, observa uma faixa de 720 km, com resolução de 2,11 km no nadir\cite{kummerok1998,trmmhandbook}.

%Nesta pesquisa, utilizamos apenas o canal  10,8 $\mu$m, para estimativa da temperatura de topo de nuvens.
\sigla{name={VIRS},description={\textit{Visible and InfraRed Scanner}}}

%\subsection{Radiômetro de microondas}

%O TMI (\textit{TRMM Microwave Imager}) é um radiômetro passivo multicanal, 10,65 GHz, 19,35 GHz, 21,3 GHz, 37 GHz, e 85,5 GHz, com dupla polarização. Possui uma varredura cônica combinada com movimento de rotação de sua antena, a qual observa regiões elipsoidais quando projetadas na superfície \cite{kummerok1998}. Sua resolução horizontal varia entre 6-50 km, dependendo do ângulo entre o feixe e o nadir, e varredura de ~760 km \cite{trmmhandbook}. 
%\sigla{name={TMI},description={\textit{TRMM Microwave Imager}}}



\section{REANÁLISES (R2) DO NCEP-DOE}

Em continuidade ao projeto das reanálises do NCEP-NCAR (R1) , o projeto reanálises 2 (R2)  \sigla{name={R2},description={Reanálises 2 do NCEP-DOE}} -- \textit{NCEP-DOE Atmospheric Model Intercomparison Project (AMIP-II) reanalysis} -- faz o uso de dados de satélite e de modelos mais atuais do que no projeto R1, buscando corrigir erros humanos e erros conhecidos de versos anteriores de modelos atmosféricos utilizados no processo de integração e assimilação de dados de envolvem o projeto R1 \cite{kanamitsu}.

\sigla{name={R1},description={Reanálises do NCEP-NCAR}}

\sigla{name={NCEP--NCAR},description={\textit{National Centers for Environmental Prediction -- National Center for Atmospheric Research }}}

\sigla{name={NCEP--DOE},description={\textit{National Centers for Environmental Prediction -- Department of Energy}}}

\section{DADOS}

Os dados referentes as observações do TRMM foram obtidos foram transferidos a partir do servidor de FTP da NASA (ftp://disc2.nascom.nasa.gov) e do NCEP (ftp://ftp.cdc.noaa.gov).

Foram utilizadas nesta pesquisa os dados de temperatura em altura geopotencial em 17 níveis de pressão das reanálises 2 do NCEP-DOE e os arquivos orbitais do TRMM, produtos 1B01, 2A25 e 1B11, ambos na versão 7, durante o período entre 1998 e 2011. 
%Nesta etapa um conjunto de \textit{scripts} foi desenvolvido para download e verificação de integridade dos dados baixados. No total o volume de dados atingiu 28 TB.  %PR 16,5TB / VIRS 10TB / TMI 1,4TB /   

Os dados do LIS de \textit{flash}, \textit{group}, \textit{events} e \textit{view time} foram concedidos pela pesquisadora \citeonline{rachel}, que processou estes dados na NASA anteriormente a esta pesquisa. 

No total, os dados brutos desta pesquisa, representaram um volume de  aproximadamente 30 terabytes. 

Para este trabalho de pesquisa os dados do TRMM e das R2 foram amostrados sobre a região limitada entre 10N-40S e 91W-30W, que abrange toda a extensão da América do Sul. 
%Portanto foi feito um recorte nos dados orbitais apenas para esta região que cobre toda a América do Sul, o que reduziu bastante o volume de dados a serem utilizados e tornou o processamento possível perante a infraestrutura computacional do IAG-USP.

Diante as varáveis que integram os 3 produtos do TRMM utilizados nesta pesquisa, foram selecionais apenas algumas variáveis conforme descreve a tabela \ref{varsTRMM}.

\begin{table}[!h]
\centering
\small
\caption{Variáveis dos produtos do TRMM que foram utilizadas na identificação e descrição das tempestades elétricas.}
\label{varsTRMM}
\renewcommand {\tabularxcolumn }[1]{ >{\arraybackslash }m{#1}}
\newcolumntype{W}{>{\centering\arraybackslash }X}
\begin{tabularx}{\textwidth}{ W W W }
\hline
\hline
\textbf{Variável} & \textbf{Sensor TRMM} & \textbf{Produto} \\
\hline
\textit{Latitude} & VIRS & 1B01 \\
\textit{Longitude} & VIRS & 1B01 \\
\textit{Channels (4) 10.8 $\mu$m} & VIRS & 1B01 \\
\textit{Latitude} & PR & 2A25 \\
\textit{Longitude} & PR & 2A25 \\
\textit{Corrected Z-factor} & PR & 2A25 \\
\textit{Rain Type}  &  PR  & 2A23 \\
\textit{Lat. Lon. Flashes} & LIS &  \cite{rachel}\\
\textit{Lat. Lon. Events} & LIS & \cite{rachel}\\
\textit{Lat. Lon. Groups} & LIS &  \cite{rachel}\\
\textit{View Time 0.5° $\times$ 0.5°} & LIS &  \cite{rachel}\\
\hline
\end{tabularx} 
\end{table} 
 

\section{AS TEMPESTADES ELÉTRICAS}
\label{identificaTempestades}


%Após uma análise ponto a ponto, buscando associar cada raio com um perfil de refletividade do PR, partimos para uma análise de grupo, buscando identificar quais as tempestades elétricas que representam maior intensidade convectiva.

Para criar o banco de dados de nuvens de tempestades elétricas deste trabalho de pesquisa, a equação de Planck foi aplicada nos dados de radiância espectral do produto 1B01, canal 4 do VIRS (10,8 $\mu$m) e desta forma, as regiões com temperaturas de brilho mais frias do que 258 K e com pelo menos um \textit{flash} do LIS observado, definiram as  tempestades elétricas.

% Após, o algoritmo verifica se houve raios detectados pelo LIS na mesma área da nuvem. Havendo pelo menos um raio, o sistema era classificado como uma tempestade elétrica. 

A partir da definição do cluster de nuvem, o algorítimo extrai as variáveis listadas na tabela \ref{varsTRMM} refentes as observações do PR e LIS na mesma área do \textit{cluster} de nuvem de tempestade elétrica.

Como os sensores do TRMM possuem diferentes resoluções, técnicas numéricas de mudança de base foram utilizadas para projetar as observações orbitais do VIRS, PR e LIS em uma grade regular com 0.05$^{\circ}$ $\times$ 0.05$^{\circ}$ de resolução, de maneira à verificar as regiões com medidas coincidentes entre o PR, LIS e VIRS.


Desta forma, cada tempestade elétrica foi armazenada na forma de um arquivo HDF contendo medidas coincidentes do VIRS, LIS e PR. 

Inicialmente foram identificadas 154,189 tempestades elétricas, pelo instrumento numérico rastreador de sistemas, porém 331 tempestades elétricas, não corresponderam a um único sistema convectivo ou multicelular. 

Portanto, foi feita uma redivisão nos sistemas enormes considerando a temperatura de brilho de 221 K. Regiões com temperatura de brilho em infravermelho inferiores a 221 K são consideradas como a parte mais ativa dos sistemas convectivos de meso-escala identificados em \citeonline{Maddox1980}. 

Com a recategorização dos sistemas enormes, portanto o número total de tempestades elétricas identificadas nesta pesquisa é de 157,592.


\section{A TAXA DE RAIOS}
\label{metodoFtaFt}



A taxa de raios por minutos ou por hora por sistema, ou apenas a taxa de raios no tempo quando não se pode aferir a extensão do sistema,  pode ser utilizada para descrever a severidade de uma tempestade. 


Para descrever a morfologia das tempestades bem como identificar quais tempestades são mais eficientes ou severas, defini-se dois índices:

\begin{itemize}
\item FT -- A taxa de raios no tempo, sendo a razão entre o número de raios ($N_{fl}$) e o tempo médio ($VT_m$) em que o sensor LIS observou a tempestade elétrica, da mesma forma como foi calcula para as \textit{precipitation features} \cite{cecil2005, Nesbitt2000}. 

\item FTA -- A taxa de raios por tempo pela área da tempestade elétrica ($A_t$). 
\end{itemize}

\simbolo{name={$N_{fl}$},description={Número de flashes }}
\simbolo{name={$VT_m$},description={Tempo médio de visada do LIS}}
\simbolo{name={$A_t$},description={Área da tempestade elétrica}}

Para cada tempestade elétrica foram calculados os dois índices que podem estar associados com a severidade, o FT e FTA, conforme as equações \ref{eqFT} e \ref{eqFTA}. 

\simbolo{name={$FT$},description={Taxa de raios por tempo $[raios~minuto^{-1}]$}} \simbolo{name={$FTA$},description={Taxa de raios por tempo por área $[raios~dia^{-1}~km^{-2}]$}}.

\begin{equation}
FT = \frac{N_{fl} }{VT_m} 60 ~[raios~minuto^{-1}]  
\label{eqFT}  
\end{equation}
%31557600 ano
\begin{equation}
FTA = \frac{N_{fl} }{VT_m A_t } 60 ~[raios~minuto^{-1}~km^{-2}]
\label{eqFTA}
\end{equation}

\section{DENSIDADE GEOGRÁFICA DE RAIOS E SISTEMAS}
\label{metodoPass}

Neste trabalho, buscamos identificar regiões mais eficientes nos processos de eletrificação, as quais tem pouca densidade de sistemas porém com alta densidade de raios em comparação com as demais regiões da América do Sul.

O que se torna fundamental na construção destes mapas é considerar quantas vezes, ou qual o tempo em que o satélite ficou observando cada parte da região de estudo, pois uma determinada região pode ter muito mais amostragens do outras. Portanto, qualquer análise de densidade espacial com dados do TRMM que não considere o número de passagens ou tempo em que o sensor observou a região projetada na superfície, será tendenciosa.

Mesmo que o satélite TRMM visite o mesmo lugar do globo duas vezes por dia em função de sua orbita inclinada 35$^{\circ}$ e velocidade, entre o período de 1998--2011 o satélite passou 10,000 vezes mais sobre a região extra-topical do que na região tropical, como mostra a figura \ref{VirsVT}, que ilustra o número de orbitas cobertas pelo sensor VIRS em uma grade regular de 0.25$^{\circ}$  $\times$ 0.25$^{\circ}$ na América do Sul.


\begin{figure}[!hb]
  \centering
  {{\includegraphics[height=13.5cm]{img/grids/passagens_virs_1998-2011}}}
\caption{Número de observações do VIRS (0.25$^{\circ}$  $\times$ 0.25$^{\circ}$).}
\label{VirsVT}
\end{figure} 


%\label{gridAmostragem} 

Levando em consideração tempo de amostragem do LIS na superfície terrestre (\textit{view time}), o número de dias de amostragem na região de estudo pode ser observado na figura \ref{lisVT}. Observa-se que em 14 anos o LIS observou 10 dias a mais na latitude 34$^{\circ}$ S do que em 0$^{\circ}$.


\begin{figure}[!ht]
  \centering
  {{\includegraphics[height=13.5cm]{img/grids/vt_trmm}} }
  \caption{Tempo de amostragem (\textit{View time}) do LIS (0.25$^{\circ}$  $\times$ 0.25$^{\circ}$).}
\label{lisVT}
\end{figure} 

As figuras  \ref{lisVT} e \ref{VirsVT} representam duas matrizes que correspondem aos pontos de uma grade igualmente espaçada (grade regular), com 0.25$^{\circ}$ de resolução, projetada sobe a América do Sul. A matriz ($\mathbf{VT}_{lis}$), figura \ref{lisVT}, do tempo total de amostragem do sensor LIS sobre a superfície e a matriz ($\mathbf{VT}_{virs}$), figura \ref{VirsVT} do número de vezes que o satélite passou conforme o tamanho da varredura do radiômetro VIRS na superfície, as quais são utilizadas para normalizar as medidas de raios e de tempestades elétricas definidas pelo VIRS.  

\simbolo{name={$\mathbf{VT}_{lis}$},description={Matriz do tempo total da visada do sensor LIS sobre a superfície}}

Neste sentido podemos calcular a densidade de raios ($\mathbf{DE}_{fl}$) e a densidade de tempestades elétricas ($\mathbf{DE}_{te}$) em função do tempo de amostragem.

Para tanto, é necessário calcular o número total de raios observados ($\mathbf{FL}_{lis}$) e o número total de tempestades elétricas ($\mathbf{P}_{te}$) em cada ponto da grade de 0.25$^{\circ}$ $\times$ 0.25$^{\circ}$. Logo é possível identificar as regiões com a maior número de amostragem de raios, na figura \ref{taxatotalraios}, e de amostragem de tempestades elétricas, na figura \ref{taxaTotalTe}.

%Com as mesmas dimensões e resolução de grade que o tempo de observação e o número de passagens do satélite foram acumulados em duas matrizes, os raios foram acumulados na matriz ($\mathbf{FL}_{lis}$) e todos os píxeis do VIRS com radiância espectral associada com temperaturas de brilho inferiores a 258 K e que definiram as áreas das tempestades elétricas, foram acumulados na matriz ($\mathbf{P}_{te}$) que representa os locais com maior cobertura de nuvens de tempestades elétricas.
%A matriz $\mathbf{FL}_{lis}$ projeta sobre a América do Sul está representada na figura \ref{gridFL} e a matriz $\mathbf{P}_{te}$, na figura \ref{gridTe}.

Na figura \ref{taxaTotalTe} é notável o alto número de sistemas na região Sul da AS, com mesma ordem de magnitude do que em locais ao Norte aonde atua a Zona de Convergência Intertropical. Mas esse máximo no Sul da AS não indica maior ocorrência de tempestades elétricas e sim maior frequência de passagem do satélite TRMM. 

%Portanto, faz-se necessário normalizar as medidas do TRMM pelo tempo de amostragem ou número de passagens.

\begin{figure}[!ht]
  \centering
  \subfloat[]{{\includegraphics[height=13.5cm]{img/grids/densEspacial_19982011acumuladoTaxaFlashPolyfill}} } 
\caption{Acumulado de raios (\textit{flashes}) observados pelo LIS (0.25$^{\circ}$  $\times$ 0.25$^{\circ}$).}
\label{taxatotalraios}
\end{figure}   
  
\begin{figure}[!ht]
  \centering 
  {{\includegraphics[height=13.5cm]{img/grids/densEspacial19982011acumuladoTempestadesPolyfill}}}
\caption{Acumulado das 157,592 tempestade elétrica (0.25$^{\circ}$  $\times$ 0.25$^{\circ}$).}
\label{taxaTotalTe}
\end{figure} 

Mesmo que as matrizes representem pontos em uma grade com espaçamento angular regular, as áreas de cada ponto de grade não são iguais, pois a  o comprimento de arco de 0.25$^{\circ}$ na direção zonal depende da latitude da região. Assim a matriz que corresponde a área da grade regular ($\mathbf{A}_g$) foi calculada e considera nos cálculos de densidades.


A densidade de raios ($\mathbf{DE}_{fl}$) foi calculada conforme a equação \ref{defl}, que apresenta a razão entre $\mathbf{FL}_{lis}$ e o $\mathbf{VT}_{lis}$ e $\mathbf{A}_g$, multiplicada por 24 horas $\times$ 60 minutos $\times$ 60 segundos $\times$ 365.25 dias, o que converte o tempo de observação do LIS de segundos para anos. Então as densidades de raios, possuem dimensões de número de [raios] por [tempo] por [quilômetro quadrado] (raios ano$^{-1}$ km$^{-2}$).

\begin{equation}
\mathbf{DE}_{fl} = \frac{\mathbf{FL}_{lis}}{\mathbf{VT}_{lis} \mathbf{A}_g} 31557600     
\label{defl}
\end{equation}

No mesmo caminho as densidades de tempestades elétricas ($\mathbf{DE}_{te}$) foram obtidas conforme a equação \ref{dete}. Mas, note que, conforme descrito em \ref{metodologiaTRMM} o tempo de amostragem do LIS e do VIRS são bem distintos. Enquanto o LIS é um sistema de imageamento, o VIRS é um radiômetro que realiza varreduras durante a trajetória do satélite. 

%Na figura \ref{VirsVT} temos apenas o acumulado de passagens do VIRS, porém, é conveniente converter o número de vezes que o VIRS observou cada ponto da grade de 0.25$^{\circ}$  $\times$ 0.25$^{\circ}$ em unidade de tempo, pois desta forma teremos as mesmas dimensões físicas tanto para a $\mathbf{DE}_{te}$ quanto para a $\mathbf{DE}_{fl}$.
%Conforme descrito em \ref{metodologiaTRMM}, em função da velocidade e órbita do satélite TRMM, é possível sobrevoar uma mesma região tropical duas vezes por dia. Então podemos considerar que cada ponto da grande de 0.25$^{\circ}$  $\times$ 0.25$^{\circ}$ da figura \ref{VirsVT} é observado pelo VIRS 2 vezes por dia, portanto se multiplicarmos 2 passagens por 365.25 dias podemos concluir o satélite observa cada ponto da grade aproximadamente 730.5 vezes por ano.
%Porém a constante de conversão de tempo na equação \ref{dete} é diferente da equação \ref{defl}, pois o tempo que o VIRS observou a AS, foi estimado a partir do número de vezes que o satélite passou sobre a AS e considerando que cada ponto de grade na orbita foi observado por 90 segundos. 

Portanto, a densidade de tempestades elétricas $\mathbf{DE}_{te}$, como descreve a equação \ref{dete} é normalizada pelo número de vezes que o VIRS sobrevoou cada ponto da grade de 0.25$^{\circ}$  $\times$ 0.25$^{\circ}$, o que corresponde a dimensão física do número de [sistemas] por [observação] por [quilômetro quadrado] (sistemas observaçoes$^{-1}$ km$^{-2}$). Essa grandeza revela, por exemplo, que a cada 10,000 observações do VIRS, temos entre 1-4 tempestades elétricas observadas na América do Sul.

\begin{equation}
\mathbf{DE}_{te} = \frac{\mathbf{P}_{te}}{\mathbf{VT}_{virs} \mathbf{A}_g}    
\label{dete}
\end{equation}

\section{MORFOLOGIA DA ESTRUTURA TRIDIMENSIONAL DA PRECIPITAÇÃO}

O estudo para descrever a morfologia da precipitação foi realizado com base nas observações do PR, buscando avaliar como a precipitação está distribuída nos níveis de altitude e como os perfis de $Z_c$ estão associados com os processos de crescimento de hidrometeoros e de eletrificação.  



\subsection{Distribuições de probabilidades com a altitude}

A partir dos perfis de $Z_c$ selecionados pelo algoritmo de identificação de tempestades elétricas, foi estudada a probabilidade de ocorrência de $Z_c$ por altitude. Desta forma, foram obtidos Diagramas de Contorno de Frequência por Altitude, os CFADs \cite{yuter1995}.\sigla{name={CFAD},description={Diagrama de Contorno de Frequência por Altitude}}

Conforme descrevem \citeonline{yuter1995}, primeiramente obteve-se uma função de densidade de probabilidade com duas variáveis ($f_{pdf}(x,y)$), cuja a dimensão $x$ correspondeu à valores de $Z_{c}$ e $y$ os nível de altitude do PR. A função $f_{pdf}(x,y)$, foi representada numericamente por uma matriz bidimensional com a granularidade de 1 dBZ para cada 250 m de altitude \cite{yuter1995}.

\simbolo{name={$f_{pdf}(x,y)$},description={Função densidade de probabilidade com duas variáveis}}


Para a obtenção dos diagramas de probabilidade normalizados por nível de altitude, cada nível $y$ da função $f_{pdf}(x,y)$ foi normalizado pelo número total de ocorrências de valores de $Z_c$ distribuídos em $x$. Os níveis $y$ de altitude com número total de ocorrência de $Z_c$ em $x$, menor do que 10\% do nível de máxima ocorrência, foram desconsiderados dos contornos de probabilidade em todos os CFADs.

Com base na ($f_{pdf}(x,y)$) que definiu cada CFAD, foi calculada a função densidade de probabilidade cumulativa ($f_{cdf}(x,y)$) de $Z_c$ por altitude, que originaram os Diagramas de Contorno de Frequência Cumulativa por Altitude (CCFAD).  \sigla{name={CCFAD},description={Diagramas de Contorno de Frequência Cumulativa por Altitude}}   


\simbolo{name={$f_{cdf}(x,y)$},description={Função densidade de probabilidade cumulativa com duas variáveis}}

Os CCFADs auxiliam a investigar quais as diferenças entre os perfis de $Z_c$ associados à diferentes quantis da amostra de probabilidade, elucidando ainda mais as informações contidas nos CFADs.


%Então, objetivando uma análise dos processos de crescimento de hidrometeoros no perfil atmosférico e na mesma óptica de trabalhos como \citeonline{Takahashi1978,Saunders1999,Takahashi2002,avila2009}, ou seja, em função de diferentes condições de temperatura, nesta pesquisa construímos o diagrama denominado como Diagrama de Contorno de Frequência por Temperatura (CFTD), \sigla{name={CFTD},description={Diagrama de Contorno de Frequência por Temperatura}}.



\subsection{Distribuições de probabilidades com a temperatura}
A distribuição de probabilidade da precipitação com a altitude associa-se com o desenvolvimento vertical, porém o tipo de hidrometeoro de determinada altitude é função da temperatura e da razão de saturação naquela altura \cite{Takahashi1978,Saunders1999}. 

Objetivando uma análise dos processos de crescimento de hidrometeoros no perfil atmosférico, na mesma óptica de trabalhos como \citeonline{Takahashi1978,Saunders1999,Takahashi2002,avila2009}, ou seja, em função de diferentes condições de temperatura e valores de $Z_c$, nesta pesquisa, foi  construído o  Diagrama de Contorno de Frequência por Temperatura (CFTD), \sigla{name={CFTD},description={Diagrama de Contorno de Frequência por Temperatura}}. 

Nos CFTDs, os níveis de temperatura não correspondem as condições controladas em laboratório, e sim às variações de temperatura do perfil atmosférico. 

Foram utilizados os dados de reanálises (R2) do NCEP-DOE  entre 1998 e 2011, mais especificamente os dados em 17 níveis de pressão, de altura geopotencial e temperatura \cite{kanamitsu}. 

Os perfis de altura geopotencial e temperatura mais próximos ou coincidentes com cada região de tempestade elétrica observada pelo VIRS, foram extraídos. A partir dos 17 níveis verticais das R2, os 80 níveis de temperaturas associados aos 80 níveis de altitude das observações do PR foram obtidos por interpolação por método de mínimos quadrados.

Desta maneira, obteve-se a função $f_{pdf}(x,y)$, cuja a dimensão $x$ correspondeu à valores de $Z_{c}$ e $y$ os nível de temperaturas estimados a partir das reanálises II do NCEP. A função $f_{pdf}(x,y)$ de $Z_c$ por temperatura, foi representada por uma matriz bidimensional com a granularidade de 1 dBZ para cada 2 $^{\circ}$C. Nos CFTDs, os níveis superiores e inferiores foram definidos para temperaturas entre 20° C e -50° C.


Também foi calculada a função $f_{cdf}(x,y)$ de $Z_c$ por temperatura, que originaram os Diagramas de Contorno de Frequência Cumulativa por Temperatura (CCFTD).  \sigla{name={CCFTD},description={Diagramas de Contorno de Frequência Cumulativa por Temperatura}}   







