\chapter{METODOLOGIA}
\label{metodologia}

A Metodologia consiste fundamentalmente na construção de um subconjunto de dados das observações dos sensores VIRS, LIS e PR abordo do satélite TRMM, durante o período entre 1998 e 2011. 

Além dos dados satelitais, as reanálises II do NCEP \sigla{name={NCEP},description={\textit{National Centers for Environmental Prediction}}} em níveis de pressão foram utilizadas para identificar os valores de temperatura nos níveis de altitude do PR.

As informações dos diferentes sensores foram combinadas de maneira à identificar sistemas denominados como tempestades elétricas, definidas como nuvens as quais possuem pelo menos um \textit{flash} detectado pelo LIS. 

%A seguir são apresentadas as principais características do TRMM 
%Para melhor entender as implicações que envolvem a construção de uma base de dados de sistemas individualmente a partir das observações do TRMM, inicialmente descreve-se algumas das principais características operacionais do satélite TRMM.

\section{O SATÉLITE TRMM}

O satélite TRMM (\textit{Tropical Rainfall Measuring Mission})\sigla{name={TRMM},description={\textit{Tropical Rainfall Measuring Mission}}}  faz parte de uma missão conjunta entre a NASA (\textit{National Aeronautics and Space Administration} - EUA)\sigla{name={NASA},description={\textit{National Aeronautics and Space Administration}}} a JAXA 
\sigla{name={JAXA},description={\textit{Japan Aerospace Exploration Agency}}}
 (\textit{Japan Aerospace Exploration Agency}), com o objetivo de determinar médias mensais da distribuição e variabilidade da precipitação e calor latente na região tropical, principalmente das regiões oceânicas, para melhor compreensão do ciclo hidrológico e validação da modelagem numérica climática de curto prazo e modelagem atmosférica global  \cite{simpson1988}.

Os instrumentos a bordo do TRMM são: radar de precipitação (PR), radiômetro de microondas (TMI), radiômetro no visível e no infravermelho (VIRS), sistema de energia radiante da terra e das nuvens (CERES) e sensor para imageamento de relâmpagos (LIS) \cite{kummerok1998}.

Esse satélite possui uma órbita de aproximadamente 320 Km de altura e inclinação de 30$^{\circ}$-35$^{\circ}$ para que possa visitar uma mesma região duas vezes ao dia, em horários distintos, sobre a região tropical do planeta Terra \cite{simpson1988}.   

\subsection{Radar de Precipitação}

O PR (\textit{Precipitation Radar}) é um radar que opera na frequência de 13,8 GHz e possui uma resolução horizontal entre 4,3-5 km, 250 m de resolução vertical e uma varredura 215 km. Uma de suas características mais importantes é a capacidade para fornecer a estrutura tridimensional dos hidrometeoros de nuvens, desde a superfície até uma altura de 20 km \cite{kummerok1998}. Para esta pesquisa serão utilizados os dados 2A25 que apresentam o fator de refletividade do radar corrigido por atenuação da chuva \cite{2A25}.
\sigla{name={PR},description={\textit{Precipitation Radar}}}

\subsection{Imageador de relâmpagos}

O LIS (\textit{Lightning Imaging Sensor}) é um sensor óptico capaz de detectar e localizar relâmpagos em tempestades individuais, analisando a emissão óptica resultante da dissociação, excitação e recombinação dos constituintes atmosféricos, em resposta a ocorrência de descargas atmosféricas. Este sensor CCD\footnote{Um dos dispositivos eletrônicos utilizados para registro de imagens em câmeras digitais.}, que trabalha no comprimento de onda de 772 nm, identifica descargas nuvem-solo e intranuvens, tanto no período diurno quanto noturno, a partir da amostragem de 500 imagens por segundo. Combinado com a velocidade do satélite (11 km/s) e abertura da CCD, o sensor LIS possui um campo de visão que permite a observação de um ponto na Terra por 80 a 90 s, tempo suficiente para a estimativa da taxa de raios de uma tempestade no momento da observação \cite{christianTM,trmmhandbook}.
\sigla{name={LIS},description={\textit{Lightning Imaging Sensor}}}

\subsection{Radiômetro no visível e infravermelho}

O VIRS (\textit{Visible and InfraRed Scanner}) é um radiômetro passivo que realizada medidas de radiância em 5 bandas espectrais, com comprimentos de onda de 0,63 $\mu$m, 1,61 $\mu$m, 3,75 $\mu$m, 10,8 $\mu$m e 12 $\mu$m. Sua resolução horizontal atinge 2,11 km no nadir e 720 km de varredura \cite{trmmhandbook}.

Nesta pesquisa, utilizamos apenas o canal  10,8 $\mu$m, para estimativa da temperatura de topo de nuvens.
\sigla{name={VIRS},description={\textit{Visible and InfraRed Scanner}}}

\subsection{Radiômetro de microondas}

O TMI (\textit{TRMM Microwave Imager}) é um radiômetro passivo multicanal, 10,65 GHz, 19,35 GHz, 21,3 GHz, 37 GHz, e 85,5 GHz, com dupla polarização. Possui uma varredura cônica combinada com movimento de rotação de sua antena, a qual observa regiões elipsoidais quando projetadas na superfície \cite{kummerok1998}. Sua resolução horizontal varia entre 6-50 km, dependendo do ângulo entre o feixe e o nadir, e varredura de ~760 km \cite{trmmhandbook}. 
\sigla{name={TMI},description={\textit{TRMM Microwave Imager}}}

\section{FONTE DE DADOS}

A fonte de dados foi obtida utilizando a infra-estrutura de rede do IAG-USP, aonde os dados foram transferidos a partir do servidor de FTP da NASA (ftp://disc2.nascom.nasa.gov) e do NCEP (ftp://ftp.cdc.noaa.gov).


Foram baixados os dados de temperatura em altura geopotencial em 17 níveis de pressão das reanálises II do NCEP e os arquivos orbitais do TRMM na versão 7, produtos 1B01, 2A25 e 1B11  para o período entre 1998 e 2011. Nesta etapa um conjunto de \textit{scripts} foi desenvolvido para download e verificação de integridade dos dados baixados. No total o volume de dados atingiu 28 TB.  %PR 16,5TB / VIRS 10TB / TMI 1,4TB /   

Os dados do LIS de \textit{flash}, \textit{group}, \textit{events} e \textit{view time} foram concedidos pela pesquisadora \citeonline{rachel}, quem já possuía essa base de dados no Brasil. 

Como as observações globais do PR, LIS, VIRS e TMI entre 1998-2011 representam um volume de aproximadamente 30 TB, a região de estudo foi limitada entre 10N-40S e 91W-30W. Portanto foi feito um recorte nos dados orbitais apenas para esta região que cobre toda a América do Sul, o que reduziu bastante o volume de dados a serem utilizados e tornou o processamento possível perante a infraestrutura computacional do IAG-USP.

\section{RAIOS COM DIFERENTES TAXAS DE DESCARGAS DE RETORNO}

O estudo da Morfologia das tempestades foi iniciado pela construção de um algoritmo que fez a extração de perfis verticais do fator de refletividade corrigida por atenuação ($Z_c$), \simbolo{name={$Z_c$},description={Fator de refletividade corrigida por atenuação, produto TRMM 2A25}} produto 2A25 \cite{iguchi2009}, nos pontos de grade onde ocorreram descargas atmosféricas (\textit{flashes}) observadas pelo LIS.

Após a extração dos perfis verticais de $Z_c$ orientada pela ocorrência de raios, foi constituída uma base de dados com as seguintes características:

\begin{itemize}
\item Para cada raio observado pelo LIS existia um perfil vertical de refletividade do radar.
\item Além dos 80 níveis verticais de cada perfil de refletividade do radar, temos também a classificação do tipo de chuva identificada pelo produto TRMM 2A25 (convectiva, estratiforme, etc).
\item Cada raio (\textit{flash}) possui o seu respectivo número de eventos (pixels da CCD iluminados), número de grupo (grupos de \textit{pixels} iluminados na CCD que compõem o raio), e tempo de duração em milisegundos. 
\end{itemize}

A morfologia da estrutura 3D da precipitação observada pelo PR foi estudada para diferentes classes de perfis separados conforme o número de descargas de retorno (\textit{groups}) de cada raio (\textit{flash}). 

Nesta etapa foi investigada se a taxa de descargas de retorno representa maior definição de precipitação em altitude principalmente na região de fase mista, entre 5 e 7 km de altitude. 

\section{IDENTIFICAÇÃO DAS TEMPESTADES ELÉTRICAS}

Após uma análise ponto a ponto, buscando associar cada raio com um perfil de refletividade do PR, partimos para uma análise de grupo, buscando identificar quais as tempestades elétricas que representam maior intensidade convectiva.

Técnicas numéricas de mudança de eixo ordenados foram utilizadas para projetar as
observações orbitais do VIRS, PR e LIS em uma grade regular com 0,05$^{\circ}$ de resolução, a qual foi utilizada para verificar regiões com medidas coincidentes entre os sensores.

A equação de Planck foi aplicada nos dados de radiância espectral do produto 1B01, canal 4 do VIRS (10,8 µm), e áreas com temperaturas de corpo negro em infravermelho mais frias do que 258 K delimitaram os \textit{clusters} de nuvens. Após, o algoritmo verifica se houve raios detectados pelo LIS na mesma área da nuvem. Havendo pelo menos um raio, o sistema era classificado como uma tempestade elétrica. 

Desta forma, cada tempestade elétrica foi armazenada na forma de um arquivo HDF contendo medidas coincidentes do VIRS, LIS e PR. Os arquivos de tempestades elétricas são compostos pelas seguintes informações contidas nos produtos do TRMM:

\begin{itemize}
\item VIRS: 1B01 -- \textit{latitude, longitude, Radiance channel 4} (10,8 µm)
\item PR: 2A25 -- \textit{latitude, longitude, Corrected Z-factor, Rain Type} 
\item LIS: \textit{latitude and longitude of, flashes, groups, events and View Time}  
\end{itemize} 

Foram identificadas 154,189 tempestades elétrica e devido a varredura do PR ser menor do que a do VIRS, apenas 96,281 tiveram pelo menos um perfil de chuva válido observado pelo radar a bordo do satélite.


\section{A TAXA DE RAIOS POR TEMPESTADE ELÉTRICA}
\label{metodoFtaFt}
A taxa de raios no tempo (FT), foi definida como a razão entre o número de flashes ($N_{fl}$) e o tempo médio ($VT_m$) em que o sensor LIS observou a tempestade elétrica, da mesma forma como foi calcula para as \textit{precipitation features} \cite{cecil2005, Nesbitt2000}. 

\simbolo{name={$N_{fl}$},description={Número de flashes }}
\simbolo{name={$VT_m$},description={Tempo médio de visada do LIS}}


A taxa de raios no tempo também foi normalizada pela área da tempestade elétrica ($A_t$), obtendo também o índice da taxa de raios no tempo por área (FTA). 
\simbolo{name={$A_t$},description={Área da tempestade elétrica}}

\begin{equation}
FT = \frac{N_{fl} }{VT_m} 60 ~[raios~minuto^{-1}]  
\label{eqFT}  
\end{equation}
%31557600 ano
\begin{equation}
FTA = \frac{N_{fl} }{VT_m A_t } 86400 ~[raios~dia^{-1}~km^{-2}]
\label{eqFTA}
\end{equation}

Para cada sistema foram calculados os dois índices que podem estar associados com a severidade de tempo, o FT e FTA, conforme mostra as equações \ref{eqFT} e \ref{eqFTA} 
\simbolo{name={$FT$},description={Taxa de raios por tempo $[raios~minuto^{-1}]$}} \simbolo{name={$FTA$},description={Taxa de raios por tempo por área $[raios~dia^{-1}~km^{-2}]$}}.

\section{DENSIDADES ESPACIAIS DE RAIOS E SISTEMAS}
\label{metodoPass}

Neste trabalho, buscamos identificar espacialmente as regiões mais eficientes nos processos de eletrificação, as quais possuem pouca densidade de sistemas porém alta densidade de raios em comparação com as demais regiões da América do Sul.

O que se torna fundamental na construção destes mapas é considerar quantas vezes, ou qual o tempo em que o satélite ficou observando cada parte da região de estudo. Qualquer análise de densidade espacial com dados do TRMM que não considere o número de passagens ou tempo em que o sensor observou a região projetada na superfície, será tendenciosa.

Mesmo que o satélite TRMM visite o mesmo lugar do globo duas vezes por dia em função de sua orbita inclinada 35° e velocidade, entre 1998 e 2011, o satélite passou 10,000 vezes mais sobre a região extra-topical do que na região tropical, como mostra a figura \ref{VirsVT}, com todas as orbitas e as varreduras do VIRS projetadas e acumuladas sobre a América do Sul. 

\begin{figure}[!ht]
  \centering{
  \subfloat[Tempo de visada do LIS (0,25°).]{{\includegraphics[height=10.65cm]{img/grids/vt_trmm}} \label{lisVT}}\\
  \subfloat[Número de passagens do VIRS (0,25°).]{{\includegraphics[height=10.65cm]{img/grids/passagens_virs_1998-2011}} \label{VirsVT}}
  }

\caption{Observações do TRMM sobre a América do Sul.}
\label{gridVT} 
\end{figure} 

Fazendo o acumulado do tempo de visada do LIS na superfície, como mostra a figura \ref{lisVT}, observa-se que em 14 anos o LIS passou 10 dias a mais na latitude -34°S do que em 0°.

Na figura \ref{gridVT}, estão representadas duas matrizes que correspondem aos pontos de uma grade igualmente espaçada (grade regular), com 0,25° de resolução, projetada sobe a América do Sul. A matriz ($\mathbf{VT}_{lis}$) do tempo total da visada do sensor LIS sobre a superfície e a matriz ($\mathbf{VT}_{virs}$), do número de vezes que o satélite passou conforme o tamanho da varredura do radiômetro VIRS na superfície.  

\simbolo{name={$\mathbf{VT}_{lis}$},description={Matriz do tempo total da visada do sensor LIS sobre a superfície}}

Com as mesmas dimensões e resolução de grade que o tempo de observação e o número de passagens do satélite foram acumulados em duas matrizes, os raios foram acumulados na matriz ($\mathbf{FL}_{lis}$) e todos os píxeis do VIRS com radiância espectral associada com temperaturas de brilho inferiores a 258 K e que definiram as áreas das tempestades elétricas, foram acumulados na matriz ($\mathbf{P}_{te}$) que representa os locais com maior cobertura de nuvens de tempestades elétricas.

A matriz $\mathbf{FL}_{lis}$ projeta sobre a América do Sul está representada na figura \ref{gridFL} e a matriz $\mathbf{P}_{te}$, na figura \ref{gridTe}. Principalmente na figura \ref{gridTe} é notável o alto número de sistemas na região Sul da AS, com mesma ordem de magnitude do que em locais ao Norte aonde atua a Zona de Convergência Intertropical. Mas esse máximo no Sul da AS não indica maior ocorrência de tempestades elétricas e sim maior frequência de passagem do satélite TRMM.

\begin{figure}[!ht]
  \centering{
  \subfloat[Acumulado de raios observados pelo LIS (0,25°).]{{\includegraphics[height=10.65cm]{img/grids/densEspacial_19982011acumuladoTaxaFlashPolyfill}} \label{gridFL}}\\
  \subfloat[Acumulado das áreas de tempestade elétrica (0,25°).]{{\includegraphics[height=10.65cm]{img/grids/densEspacial19982011acumuladoTempestadesPolyfill}} \label{gridTe}}
  }

\caption{Acumulados dos raios e áreas das 154,189 tempestades elétricas identificadas.}
\label{gridSistemas} 
\end{figure} 

Mesmo que as matrizes representem pontos em uma grade com espaçamento angular regular, as áreas de cada ponto de grade não são iguais, pois a  o comprimento de arco de 0,25° na direção zonal depende da latitude da região. Assim a matriz que corresponde a área da grade regular ($\mathbf{A}_g$) foi calculada e considera nos cálculos de densidades espaciais.


Portanto, a densidade espacial de raios ($\mathbf{DE}_{fl}$) é calculada conforme a equação \ref{defl}. Note que a razão de $\mathbf{FL}_{lis}$ por $\mathbf{VT}_{lis}$ e $\mathbf{A}_g$ é multiplicada por $24\times60\times60\times365,25$, o que converte o tempo de observação do LIS de segundos para anos. Então as densidades espaciais de raios, possuem dimensões de número de [raios] por [tempo] por [quilômetro quadrado].

No mesmo caminho as densidades espacias de tempestades elétricas ($\mathbf{DE}_{te}$) foram obtidas conforme a equação \ref{dete}. Porém a constante de conversão de tempo na equação \ref{dete} é diferente da equação \ref{defl}, pois o tempo que o VIRS observou a AS, foi estimado a partir do número de vezes que o satélite passou sobre a AS e considerando que cada ponto de grade na orbita foi observado por 90 segundos. 

Portanto ao converter a matriz $\mathbf{VT}_{virs}$ para segundos de observação, temos um fator de 90 no denominador, que está implícito na equação \ref{dete}. 


\begin{equation}
\mathbf{DE}_{fl} = \frac{\mathbf{FL}_{lis}}{\mathbf{VT}_{lis} \mathbf{A}_g} 31557600 ~[raios~ano^{-1}~km^{-2}]    
\label{defl}
\end{equation}

\begin{equation}
\mathbf{DE}_{te} = \frac{\mathbf{P}_{te}}{\mathbf{VT}_{virs} \mathbf{A}_g} 350640 ~[sistemas~ano^{-1}~km^{-2}]    
\label{dete}
\end{equation}

\section{MORFOLOGIA DA ESTRUTURA 3D DA PRECIPITAÇÃO}


O estudo para descrever a morfologia da precipitação foi realizado com base nas observações do PR, buscando avaliar como a precipitação está distribuída nos níveis de altitude e como os perfis de $Z_c$ estão associados com os processos de crescimento de hidrometeoros e de eletrificação.  

A partir dos perfis de $Z_c$ selecionados pelo algoritmo de identificação de tempestades elétricas, foi estudada a probabilidade de ocorrência de $Z_c$ por altitude. Desta forma, foram obtidos Diagramas de Contorno de Frequência por Altitude, os CFADs.\sigla{name={CFAD},description={Diagrama de Contorno de Frequência por Altitude}}

Conforme descrevem \citeonline{yuter1995}, primeiramente obteve-se uma função de densidade de probabilidade com duas variáveis ($f_{pdf}(x,y)$), cuja a dimensão $x$ correspondeu à valores de $Z_{c}$ e $y$ os nível de altitude do PR. A função $f_{pdf}(x,y)$, foi representada numericamente por uma matriz bidimensional com a granularidade de 1 dBZ para cada 250 m de altitude.

\simbolo{name={$f_{pdf}(x,y)$},description={Função densidade de probabilidade com duas variáveis}}


Para a obtenção dos diagramas de probabilidade normalizados por nível de altitude, cada nível $y$ da função $f_{pdf}(x,y)$ foi normalizado pelo número total de ocorrências de valores de $Z_c$ distribuídos em $x$. Os níveis $y$ de altitude com número total de ocorrência de $Z_c$ em $x$, menor do que 10\% do nível de máxima ocorrência, foram desconsiderados dos contornos de probabilidade em todos os CFADs.

Com base na função de densidade de probabilidade ($f_{pdf}(x,y)$) que definiu cada CFAD, foi calculada a função densidade de probabilidade cumulativa ($f_{cdf}(x,y)$) de $Z_c$ por altitude, que originaram os Diagramas de Contorno de Frequência Cumulativa por Altitude (CCFAD).  \sigla{name={CCFAD},description={Diagramas de Contorno de Frequência Cumulativa por Altitude}}   


\simbolo{name={$f_{cdf}(x,y)$},description={Função densidade de probabilidade cumulativa com duas variáveis}}

Os CCFADs auxiliam a investigar quais as diferenças entre os perfis de $Z_c$ associados à diferentes quantis da amostra de probabilidade, elucidando ainda mais as informações contidas nos CFADs.


\subsection{Estrutura tridimensional da precipitação na óptica dos processos microfísicos}

\label{chuvaEtemperatura}


Em \citeonline{Fabry1995}, é mostrado que processos como a agregação, acreção e colisão coalescência, podem ser estudados em função da espessura da camada de derretimento e flutuações nos valores do fator de refletividade no perfil atmosférico. 

Pois, sendo o fator de refletividade do radar proporcional ao diâmetro das gotas no volume iluminado elevado a 6 potência, os processos de crescimento de flocos de neves, granizo e gotas, são marcados por aumentos abruptos no fator de refletividade do radar. 

E considerando um regime de precipitação estratiforme, o qual é muito mais governado por processos de agregação do que acreção, será observado um aumento acentuado no fator de refletividade do radar em torno da isoterma de 0 $^{\circ}$C associado ao derretimento de flocos de neve. \simbolo{name={$^{\circ}$C},description={Grau Celcius}} Como o índice de refração de micro-ondas no gelo é de $\sim$0,1 e na água líquida de $\sim$0,9, a transição de fase sólida para líquida representa um aumento de 7 dBZ na potência do sinal do radar.


% durante o caminho que a precipitação percorre até a superfície ou temperaturas acima de 0°C.
%Na figura \ref{fabry}, \citeonline{Fabry1995}
%e o trabalho de 
%\begin{figure}[hbp]
%  \centering{
%  \subfloat[\cite{Fabry1995}]{{\includegraphics[scale=0.25]{img/ilustracoes/fabry}} \label{fabry}}
%  \subfloat[\cite{Takahashi2002}]{{\includegraphics[scale=0.35]{img/ilustracoes/takahashi}} \label{taka}}
%  }
%\caption{Fabry Taka}
%\label{fabyTaka} 
%\end{figure} 

Em um ambiente de precipitação convectiva a transição de fase é perturbada por correntes ascendentes e os processos de agregação, acreção e colisão coalescência, os quais são os maiores responsáveis pelo aumento do diâmetro dos hidrometeoros de nuvem, tornam-se mais eficientes. 

A mudança do índice de refração da água não ocorre em torno de 0 $^{\circ}$C, pois no ambiente convectivo teremos água super-resfriada em temperaturas de -15 $^{\circ}$C, o que intensifica o processo de acreção podendo gerar gelo sólido que cai até a superfície.

Portanto, quanto maior a espessura da camada de derretimento, podemos pressupor que, o ambiente terá maior intensidade convectiva, pois terá processos de crescimento de granizo mais ativos.

Consequentemente, a taxa de raios associa-se com a intensidade convectiva devido a acreção\footnote{A acreção é o processo de \textit{rimming} descrito no trabalho de \citeonline{Takahashi1978}.} ser o processo mais eficiente de eletrificação de nuvens, principalmente quando há presença de flocos de neve embebidos na região de fase mista \cite{Takahashi1978,Takahashi2002}. 

Então, objetivando uma análise dos processos de crescimento de hidrometeoros no perfil atmosférico e na mesma óptica de trabalhos como \citeonline{Takahashi1978,Saunders1999,Takahashi2002,avila2009}, ou seja, em função de diferentes condições de temperatura, nesta pesquisa construímos o diagrama denominado como Diagrama de Contorno de Frequência por Temperatura (CFTD), \sigla{name={CFTD},description={Diagrama de Contorno de Frequência por Temperatura}}.

Nos CFTDs, os níveis de temperatura não correspondem as condições controladas em laboratório, e sim às variações de temperatura do perfil atmosférico. 

Foram utilizados os dados de reanálises II do NCEP entre 1998 e 2011, mais especificamente os dados em 17 níveis de pressão, de altura geopotencial e temperatura \cite{kanamitsu}. 

Os perfis de altura geopotencial e temperatura mais próximos ou coincidentes com cada região de tempestade elétrica observada pelo TRMM, foram extraídos. Utilizando interpolação de dados, os 80 níveis de altitudes referentes as observações do PR, foram convertidos em 80 níveis de temperatura.

Desta maneira, obteve-se a função $f_{pdf}(x,y)$, cuja a dimensão $x$ correspondeu à valores de $Z_{c}$ e $y$ os nível de temperaturas estimados a partir das reanálises II do NCEP. A função $f_{pdf}(x,y)$ de $Z_c$ por temperatura, foi representada por uma matriz bidimensional com a granularidade de 1 dBZ para cada 2 $^{\circ}$C. Nos CFTDs, os níveis superiores e inferiores foram definidos para temperaturas entre 20° C e -50° C.


Também foi calculada a função $f_{cdf}(x,y)$ de $Z_c$ por temperatura, que originaram os Diagramas de Contorno de Frequência Cumulativa por Temperatura (CCFTD).  \sigla{name={CCFTD},description={Diagramas de Contorno de Frequência Cumulativa por Temperatura}}   







