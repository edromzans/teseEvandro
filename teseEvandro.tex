\documentclass[12pt,a4paper,tocpage=plain,floatnumber=continuous,chapter=TITLE,appendix=nobox,font=plain, header=ruled,hyperindex=false]{abnt}
\usepackage[utf8x]{inputenc}
\usepackage[portuges, brazil]{babel}
\usepackage{tikz}
\usetikzlibrary{positioning}
\usetikzlibrary{shadows}
\usepgflibrary{arrows}
\usepackage{bm}
\usepackage{amsmath}
\usepackage{amsfonts}
\usepackage{amssymb}
\usepackage{lmodern}
\usepackage{courier}
\usepackage{indentfirst}
\usepackage{latexsym,amstext,amsxtra,amsopn}
\usepackage{hyperref}
\usepackage[alf,abnt-full-initials=no,abnt-etal-list=0,abnt-emphasize=bf,abnt-thesis-year=final,abnt-year-extra-label=yes,abnt-etal-cite=2]{abntcite}
\usepackage{graphicx}
\usepackage[font=small,labelfont=small,textfont=small]{caption}
\usepackage[singlelinecheck=true,margin=10pt,font=small]{subfig}
\usepackage{rotating}
\usepackage{marvosym}
\usepackage{remreset}
\usepackage{setspace}
\usepackage{listings}
\usepackage{adjustbox}
\usepackage{multirow}
\usepackage{pdfpages}
%\usepackage[hang,center]{subfigure}
\usepackage{makeidx}
\usepackage{tabularx,array,colortbl}
\usepackage[section]{placeins} %too many floats!!
%\usepackage[table]{xcolor}
%\usepackage[style=list,number=none,border=none,header=none,cols=2]{glossary}
\usepackage[style=long,number=none,border=none,header=none, cols=2 ]{glossary}
%\usepackage{glossaries}
%\usepackage[subentrycounter,seeautonumberlist]{glossaries}
%\usepackage{lineno}
%Numeração das equações
\usepackage{remreset}
\makeatletter\@removefromreset{equation}{chapter}\makeatother

%defini glossarios
\newglossarytype[agl]{sigla}{sig}{sgl}
\newglossarytype[sgl]{simbolo}{sim}{sbl}
%\newglossary[agl]{sigla}{sig}{sgl}{Siglas}
%\newglossary[sgl]{simbolo}{sim}{sbl}{Simbolos}

%\usepackage{geometry}
%\geometry{tmargin= 3cm,bmargin= 2cm,lmargin= 3cm,rmargin=2cm}
\makeglossary
%\makeglossaries
\makeindex
\makeatletter
\@removefromreset{footnote}{chapter}
\makeatother
\author{Evandro Moimaz Anselmo}
\renewcommand{\ABNTchapterfont}{\bfseries}
\renewcommand{\ABNTchaptersize}{\normalsize}
\renewcommand{\ABNTanapsize}{\normalsize}
\renewcommand{\ABNTsectionfontsize}{\normalsize} 
\renewcommand{\ABNTsectionfont}{\rm} 
\renewcommand{\ABNTsubsectionfontsize}{\normalsize}
\renewcommand{\ABNTsubsectionfont}{\rm\bfseries}
\renewcommand{\captionfont}{\small}
%%códigos fontes
%\renewcommand{\lstlistingname}{Código fonte}
%\renewcommand{\lstlistlistingname}{Lista de Códigos Fonte}
%\renewcommand*\thelstlisting{\arabic{lstlisting}}
\renewcommand{\theequation}{\arabic{equation}}

%\renewcommand{\descriptionwidth}{0.85\textwidth}
%\renewcommand{\glossaryalignment}{\textwidth}
\hypersetup{pdfborder=000,colorlinks=false,pdftitle={Morfologia das tempestades elétricas na América do Sul}, pdfcreator={Evandro Moimaz Anselmo}, pdfproducer={Evandro Moimaz Anselmo}, pdfkeywords={Raios, tempestades elétricas, precipitação tridimensional}}
%\hypersetup{colorlinks, citecolor=black,filecolor=black,linkcolor=black,urlcolor=black}
%\hypersetup{colorlinks=false}

%\hypersetup{backref=true,pdfpagemode=UseOutlines,colorlinks=true,a5paper,breaklinks=true,hyperindex,linkcolor=blue,
%anchorcolor=black,citecolor=green,filecolor=magenta,menucolor=red,pagecolor=red,urlcolor=cyan,bookmarks=true,
%bookmarksopen=true,pdfpagelayout=SinglePage,pdfpagetransition=Dissolve}
\hypersetup{bookmarksopen=true,backref=true,pdfpagemode=true,hyperindex,bookmarks=true}

\lstset{ 
language=SQL,                % choose the language of the code
extendedchars=true,
basicstyle=\scriptsize,       % the size of the fonts that are used for the code
numbers=left,                   % where to put the line-numbers'z
numberstyle=\footnotesize,      % the size of the fonts that are used for the line-numbers
stepnumber=2,                   % the step between two line-numbers. If it's 1 each line will be numbered
numbersep=8pt,                  % how far the line-numbers are from the code
numberstyle=\tiny,
%backgroundcolor=\color{white},  % choose the background color. You must add \usepackage{color}
showspaces=false,               % show spaces adding particular underscores
showstringspaces=false,         % underline spaces within strings
showtabs=false,                 % show tabs within strings adding particular underscores
frame=trBL,			% adds a frame around the code
frameround=fttt,
tabsize=2,			% sets default tabsize to 2 spaces
captionpos=b,			% sets the caption-position to bottom
breaklines=true,		% sets automatic line breaking
breakatwhitespace=false,	% sets if automatic breaks should only happen at whitespace
escapeinside={\%*}{*)},        % if you want to add a comment within your code
basicstyle=\fontfamily{pcr}\fontseries{m}\selectfont\footnotesize,
commentstyle=\ttfamily\color{gray}
%columns=fullflexible
}
%%%
\setcounter{table}{0}
\setcounter{figure}{0}
\begin{document}
%\linenumbers
\autor{Evandro Moimaz Anselmo}
\titulo{Morfologia das tempestades elétricas na América do Sul}
\orientador[Orientador:\\]{Prof. Dr. Carlos Augusto Morales Rodriguez }

\comentario{Tese ao departamento de Ciências Atmosféricas, realizada como pré-requisito para obtenção do título de Doutor em Ciências.}
\instituicao{Instituto de Astronomia, Geofísica e Ciências Atmosféricas da Universidade de São Paulo}
\local{São Paulo - SP}
\data{2015}
\capa
\folhaderosto
%\begin{folhadeaprovacao}
%Dissertação titulada como \textit{Estudo da razão entre o número de relâmpagos intranuvens e nuvem-solo para sistemas convectivos na cidade de Campo Grande - MS}, defendida por Evandro Moimaz Anselmo e aprovada em 29 de Maio de 2009 na cidade de Campo Grande - MS, Universidade Federal de Mato Grosso do Sul, Departamento de Física, pela seguinte banca examinadora: 
%\setlength{\ABNTsignthickness}{0.4pt}
%\setlength{\ABNTsignskip}{2cm}
%\assinatura{Prof. Dr. Widinei Alver Fernandes\\ Universidade Federal de Mato Grosso do Sul \\ Orientador}
%\assinatura{Prof. Ph. D. Moacir Lacerda\\ Universidade Federal de Mato Grosso do Sul \\ Co-orientador}
%\assinatura{Prof. Ph. D. Carlos Augusto Morales \\ Universidade de São Paulo}
%\assinatura{Prof. Dr. Roberto Ferreira dos Santos\\ Universidade Federal de Mato Grosso do Sul}
%\end{folhadeaprovacao}
\pretextualchapter{}
\vfill
\begin{flushright}
\textit{\large \textit{In memorian} de Darci Priciliano Anselmo e Dulcílio Moimáz}
\end{flushright}

\pretextualchapter{AGRADECIMENTOS}

A Coordenação de Aperfeiçoamento de Pessoal de Nível Superior (CAPES) e o seu Programa de Excelência Acadêmica (PROEX), ao Conselho Nacional de Desenvolvimento Científico e Tecnológico (CNPQ) e a Companhia Energética do Estado do Ceará (COELCE) PD-0039-0040/2010 pelo apoio financeiro que tornou possível a realização deste trabalho de pesquisa científica.

Ao professor Dr. Carlos Augusto Morales Rodrigues, por sua dedicação e nível de exigência/confiança com este trabalho de pesquisa. 

A professora Rachel Albretch que concedeu os dados do LIS (eventos, grupos e raios), com tempo de visada com resolução de 0,25$^{\circ}$ $\times$ 0,25$^{\circ}$, processados na NASA.

Ao professor Dr. Colin Price que durante sua visita no IAG, ministrou seminário, debateu a respeito deste trabalho no laboratório STORM-T, indicando referencias e contribuindo com sugestões. Também ao professor Dr. Daniel Cecil que durante a sua participação no experimento CHUVA em Santa Maria, dedicou seu tempo para debater a respeito deste trabalho.   

Ao João Neves, Fernando, Vinícius e Diego por auxiliaram na manutenção da infraestrutura de pesquisa do laboratório STORM-T, ambiente aonde esse trabalho de pesquisa foi feito.

A minha família, minha mãe Aparecida Moimaz, meu pai, Francisco Jairo Anselmo e minha irmã Minéia Moimaz Anselmo. 

A minha companheira Sabrina Miranda Areco.

A todos os meteorologistas que durante o cumprimento dos créditos, me auxiliaram a estudar Meteorologia, foi uma trabalheira com  Glauber, Mercel, Juliano, Rafael, Bruna, Camila, Carol, Lívia, Fabiani, Luana, Andreia, Luiz, Maria, IAG é nois.

Aos professores, e técnicos do IAG.

\begin{resumo}

Com base nas observações do satélite \textit{Tropical Rainfall Measuring Mission} (TRMM) entre janeiro de  1998 e dezembro de 2011, foi criado um banco de dados de  tempestades elétricas sobre a América do Sul (10$^{\circ}$ Norte--40$^{\circ}$ Sul e 91$^{\circ}$--30$^{\circ}$ Oeste). As tempestades elétricas foram definidas por regiões de pixeis contíguos com $T_b$ $\leq$ 258 K  (10,8 $\mu$m do \textit{Visible and InfraRed Scanner} (VIRS)) e com ocorrência de pelo menos um raio -- flashe --  do \textit{Lightning Imaging Sensor} (LIS). Definidas as distribuições espaciais das tempestades elétricas, os perfis verticais do fator de refletividade  do radar ($Z_c$) do \textit{Precipitation Radar} (PR) foram extraídos sobre as regiões coincidentes às tempestades elétricas. A partir desta metodologia, identifica-se 157~592 tempestades elétricas do TRMM, tornando possível determinar o ciclo diurno, ciclo anual, distribuição geográfica: de densidade de raios, de densidade de tempestades elétricas e da densidade de raios por tempestades sobre a América do Sul (AS). Identifica-se que a estação de tempestades elétricas na AS se configura entre outubro e março e  possui um pico em janeiro, durante o verão austral, e outro em outubro, período de transição entre a estação seca e chuvosa, quando foi observada a maior atividade de tempestades elétricas. A hora local de maior ocorrência de tempestades elétricas  foi durante às 14h, em que a probabilidade de ocorrência de tempestade elétrica foi 5,4 vezes maior do que no horário de menor ocorrência, às 9h. Em termos de sazonabilidade temos que a primavera apresenta a maior atividade de tempestades elétricas (57~861) seguidas pelo verão (46~077), outono (36~804) e inverno (16~850). As regiões mais eficientes em termos de densidades de raios por tempestades, ficam situadas na Foz do Rio Catatumbo (pixel com 772 km$^{2}$) com valor de 11,73 $\times$ 10$^{-2}$ ano$^{-1}$ km$^{-2}$, que representa {114 333} raios ano$^{-1}$ e em seguida a Bacia do Prata e Serra de Córdoba da Argentina com valores $\simeq$5,5 $\times$ 10$^{-2}$ ano$^{-1}$ km$^{-2}$. Adicionalmente, foi estudado a severidade das tempestades elétricas a partir da taxa de raios no tempo -- FT -- (raios [min$^{-1}$]) e taxa de raios no tempo normalizada pela área do sistema -- FTA -- (raios [min$^{-1}$] [km$^{-2}$]) combinada com o estudo da probabilidade de ocorrência dos perfis de $Z_c$ por nível de altitude e nível de temperatura, sendo a altitude das observações do PR convertidas em níveis de temperatura a partir das reanálises do NCEP-DOE.  As tempestades elétricas potencialmente severas estiveram associadas com dois grupos de eventos extremos, os com FTA entre 29,3--1258,7 $\times$ 10$^{-4}$ raios min$^{-1}$ km$^{-2}$ e com FT entre 47,2--1283,6 raios min$^{-1}$. Os sistemas com valores extremos de FTA, são mais frequentemente observados com área de 3 ordens de grandeza menor do que a área dos extremos de FT, com 70\% de área convectiva e 30\% de área estratiforme, enquanto que para os extremos de FT, 20\% de fração convectiva e 75\% de fração estratiforme. A análise da morfologia da estrutura tridimensional da precipitação mostra maior quantidade de água líquida super-resfriada, portanto, processo de acreção mais vigoroso nas regiões dos núcleos de raios das tempestades elétricas com extremos de FTA do que nos núcleos de raios das tempestades elétricas com extremos de FT.

\bigskip
\bigskip
Palavras-chave: raios, tempestades elétricas, TRMM, precipitação tridimensional, eletrificação de nuvens, tempo severo.
\end{resumo}

\begin{abstract}
\paragraph*{}
Abstract

\bigskip
\bigskip
Key-words: lightning, storms, tracking.
\end{abstract}

\listoffigures
\listoftables
%\lstlistoflistings
%\sumario
\cleardoublepage
\phantomsection
%\ABNTaddcontentsline{lot}{chapter}{SUMÁRIO}
\tableofcontents


\makesigla
\renewcommand{\glossaryname}{LISTA DE SIGLAS E ABREVIATURAS}
\cleardoublepage
\phantomsection
\printsigla
\makesimbolo
\renewcommand{\glossaryname}{LISTA DE S\'{I}MBOLOS}
\cleardoublepage
\phantomsection
\printsimbolo

%\printglossaries
\chapter{INTRODUÇÃO}

Desde \citeonline{whipple1929}, ao associar medidas de campo eletrostático com as observações meteorológicas de superfície dos dias com tempestades elétricas, verifica-se que a  América, África e Continente Marítimo são as chaminés de descargas elétricas atmosféricas -- raios -- globais. Em 1929 já se observava que a maior intensificação do campo elétrico de bom tempo está relacionada à atividade de tempestades elétricas sobre a América do Sul (AS). Apesar da Teoria do Circuito Elétrico Atmosférico Global mostrar que a América é a chaminé dominante no processo de manutenção do circuito elétrico atmosférico global, não era possível saber se a maior intensidade do campo eletrostático estava associada com uma maior taxa de raios \cite{dolezalek1972}.


% Portanto, a AS é a chaminé dominante no processo de manutenção do circuito elétrico atmosférico global. 
\sigla{name={AS},description={América do Sul}}
%Conhecer a distribuição geográfica de raios e de tempestades elétricas, bem como das taxas de raios por tempestades elétricas sobre a AS pode auxiliar na compreensão de questões fundamentais  em eletricidade atmosférica como descrito em \cite{dolezalek1972}. Por exemplo em \citeonline{williams2004}, buscou-se 
%Apenas quatro décadas mais tarde, medidas de raios por redes de sensores de VLF em solo começaram a serem desenvolvidas conforme descreve \citeonline{dolezalek1972},

%Mais tarde, foram métodos de identificação de raios a partir de imagens de satélite no visível e infravermelho próximo .

Os estudos de \citeonline{vorpahl1967frequency, sparrow169satellite}, utilizando imagens dos satélites \textit{Orbiting Solar Observatory} (OSO) e de  \citeonline{turman1978}, com satélite do \textit{Defense Meteorological Satellite Program observations} (DMSP), foram importantes para o entendimento da atividade elétrica atmosférica global com base na frequência de ocorrência de raios e não na frequência global de dias de tempestades elétricas como descrito em \citeonline{brooks1925distribution}. Com o uso do sensor de raios \textit{Supplementary Sensor L} (SSL) a bordo do DMSP 8531, \citeonline{turman1978} calculou a taxa de raios total\footnote{ O termo raios total, faz referência aos raios intranuvens e nuvem-solo conjuntamente.} por unidade de área de um complexo de tempestades (4 $\times$ 10$^{-5}$ s$^{-1}$ km$^{-2}$) e a taxa de raios por unidade de área observada pelo SSL (6 $\times$ 10$^{-8}$ s$^{-1}$ km$^{-2}$), equivalente a uma taxa de raios global de 31 s$^{-1}$, valor menor do que estimado em \citeonline{brooks1925distribution}, de 100 s$^{-1}$.

Em \citeonline{orville1979global}, a taxa de raios global foi calculada e obtido valores correspondentes com os estimados por \citeonline{brooks1925distribution}.  \citeonline{orville1979global} obtiveram pela primeira vez a taxa de raios global para cada mês do ano, proporcionando uma compreensão sobre a sazonalidade da taxa de raios global. Constatou-se que a taxa de raios total sobre o continente era entre 8-20 vezes maior do que sobre o oceano, e que durante o verão do hemisfério Norte a taxa de raios global era 1,4 vezes maior do que durante o verão do hemisfério Sul.

\sigla{name={LIS},description={Sensor imageador de raios}}
\sigla{name={DMSP},description={\textit{Defense Meteorological Satellite Program observations}}}
\sigla{name={OSO},description={\textit{Orbiting Solar Observatory}}}

Considerando observações mais recentes de raios total, como em \cite{christian2003global}, que utilizaram dados \textit{Optical Transient Detector} (OTD) a bordo do satélite MicroLab-1,  estimou-se que taxa de raios na média anual sobre o oceano era de 5 s$^{-1}$, sobre as regiões continentais entre 31--49 s$^{-1}$ e a  média anual da taxa de raios global de 44$\pm$5 s$^{-1}$. Por meio de mapas da distribuição da densidade geográfica de raios, verificou-se que as maiores extensões em área com as maiores ocorrências de raios por quilômetro quadrado por ano ficavam situadas sobre o continente Americano e Africano, sendo a Bacia do Rio Congo a região mais extensa com as maiores taxas de raios do globo. Atualmente, com as observações do OTD e do \textit{Lightning Imaging Sensor} (LIS) a bordo do TRMM estima-se uma taxa de raios global de 46 s$^{-1}$, e o local com a maior densidade de raios anual global é a região oeste do Congo com 160   km$^{-2}$ ano$^{-1}$ \cite{cecil2014gridded}.

\citeonline{williams2004} buscaram entender a maior resposta da Curva de Carnegie associada ao horário de maior atividade de tempestades elétricas sobre a América fazendo um estudo comparativo entre as regiões da Bacia Amazônica e Bacia do Congo. Sobre a bacia hidrográfica conguês, as taxa de raios por km$^2$ por ano são maiores enquanto que os sistemas precipitantes sobre a Bacia Amazônica possuem menor densidade de raios porém maior volume de chuva. Com base nas observações de \citeonline{soula2003surface}, \citeonline{williams2004} sugerem que a chuva eletricamente carregada sobre a Bacia Amazônica pode funcionar como um processo de carregamento da superfície terrestre com cargas negativas, ou seja, pode funcionar como bateria do Circuito Elétrico Global.

Apesar da contribuição meridional na média anual da taxa de raios global ser liderada pelo continente Africano, durante o inverno e a primavera austral da América, as taxas de raios são maiores sobre a América \cite{christian2003global}. Em \citeonline{albrecht2011b}, a partir de treze anos de dados do LIS, foi constatado que o local da maior densidade anual de raios global havia se movido da região do Congo para a região do Lago Maracaibo na Venezuela.

Portanto, no contexto das medidas globais de raios, a América do Sul encontra-se em um dos locais mais eletricamente ativos do globo. Saber quando e aonde as tempestades elétricas ocorrem, bem como, quais os locais em que os sistemas são mais eficientes na produção de raios, torna-se fundamental para o planejamento da infraestrutura dos países Sul-americanos, no sentido de garantir segurança no transporte aéreo, fluvial e terrestre, nas linhas de transmissão de dados e de energia elétrica, etc, setores estratégicos que quando paralisados devidos aos danos causados pela queda de raios refletem em prejuízos em cascata em todos os setores econômicos. 

Por exemplo, uma falha no sistema de distribuição de energia elétrica pode cessar a energia elétrica de um bairro, cidade, etc. Pode causar queima de equipamentos eletroeletrônicos devido sobre tensão elétrica, causar quedas na rede de internet, o que pode paralisar  setores como: educação, pesquisa, comércio e industrias. Também gera grande número de ações judiciais indenizatórias contra as operadoras do sistema de energia, sobrecarregando o sistema judiciário. No Brasil, as empresas prestadoras de serviços de  fornecimento de energia elétrica ao consumidor lideram as reclamações nos PROCONs ao lado de empresas de telecomunicações, evidenciando a falta de infraestrutura destes setores. Em \citeonline{pinto2005arte,noticiainpe2007}, estima-se prejuízos na ordem de 1 bilhão de dólares anuais em função da densidade de raios observada apenas sobre o Brasil. 
  
    
%Desta forma, o processo de eletrificação dos hidrometeoros até a formação de um raio, depende da capacidade do ar quente e úmido da superfície romper a estabilidade atmosférica e atingir altitude entre 4--15 km, regiões acima da isoterma de 0 $^{\circ}$C. Portanto, as tempestades elétricas podem indicar condições dinâmicas e termodinâmicas associadas a convecção profunda na atmosfera \cite{doswell2001,zipser2006}.

Além dos raios que atingem o solo -- raios nuvem-solo -- causando danos a sociedade, as tempestades elétricas indicam condições dinâmicas e termodinâmicas associadas a convecção profunda na atmosfera, pois o processo de eletrificação dos hidrometeoros relacionado com a formação de um raio, depende da capacidade do ar quente e úmido da superfície romper a estabilidade atmosférica e atingir altitude entre 4--15 km, regiões acima da isoterma de 0 $^{\circ}$C \cite{doswell2001,zipser2006}. Por exemplo,   \citeonline{macgorman1989,carey1998,williams1999}, mostram que condições de tempo severo (frentes de rajadas com velocidade superior a 92,6 km h$^{-1}$, queda de granizo com diâmetro maior do que 1,9 cm ou tornados) são geralmente precedidas de um salto na taxa de raios total das tempestades elétricas associado ao intenso crescimento de hidrometeoros na região mista.      
%governado por raios que não atingem o solo -- raios intra-nuvens 

Neste sentido, técnicas de seleção de sistemas meteorológicos a partir de dados de sensoriamento remoto, combinadas com o monitoramento da taxa de raios dos sistemas são de grande importância para o monitoramento de tempo severo. 

Diversos estudos definiram sistemas meteorológicos fazendo o agrupamento de regiões na superfície a partir de limiares de temperatura de brilho observadas em satélite. \citeonline{Maddox1980}, observou a ocorrência de duas regiões: uma com temperatura de brilho $\leqslant$ -32°C (241K) e área $\geqslant$ 100,000 km$^2$;  outra região menor, no interior da região maior, com temperatura de brilho $\leqslant$ -52°C (221K) e área $\geqslant$ 50,000 km$^2$, que estavam associadas com Sistemas Convectivos de Meso-escala (SCM) nos Estados Unidos.  \citeonline{mapes1993}, utilizaram esta metodologia e também fazem uma síntese de trabalhos que buscaram selecionar \textit{clusters} de nuvens a partir de limiares de temperatura de brilho em infravermelho. Considerando monitoramento de sistemas sobre a AS, destaco os estudos de \citeonline{machado1998,laurent2002} que foram precursores para a operacionalização do sistema ForTraCC descrito em \cite{vila2008}.


Em termos do monitoramento de sistemas juntamente com a taxa de raios, \citeonline{morales2003} desenvolveram um algoritmo hidro-estimador estudando regiões de temperatura de brilho em infravermelho do  \textit{Geostationary Operational Environmental Satellite} (GOES) coincidentes com medidas de raios da  \textit{Sferics Timing and Ranging Network} (STARNET) e a precipitação observada pelo \textit{Precipitation Radar} (PR)  a bordo do satélite \textit{Tropical Rainfall Measuring Mission} (TRMM). Os dados da STARNET foram utilizados para definir \textit{clusters} de nuvens com raios e sem raios, em que os raios estiveram associados ao núcleos mais intensos de precipitação. Foi observado que as descargas localizadas pela STARNET, em 90\% dos casos, estiveram associados a regiões com temperatura de brilho em infravermelho menores do que 258 K.

\sigla{name={GOES},description={\textit{Geostationary Operational Environmental Satellite}}}
\sigla{name={PR},description={\textit{Precipitation Radar}}}
\sigla{name={TRMM},description={\textit{Tropical Rainfall Measuring Mission}}}
\sigla{name={STARNET},description={\textit{Sferics Timing and Ranging Network}}}

%Porém a radiação infravermelha observada por satélites, corresponde apenas a irradiação do topo das nuvens. Nuvens finas, com formação acima da isoterma de 0°C, como por exemplo as nuvem cirrus, podem cobrir grandes extensões e não estar associadas a precipitação nem descargas elétricas.


Em \citeonline{houze1993} Linhas de Instabilidades (LI) foram definidas observando extensões com chuva contínua observada por radar. Em \citeonline{MohrZipser1996} Sistemas Convectivos de Meso-escala (SCM) sobre os trópicos foram observados a partir do espalhamento radiativo em micro-ondas (85 GHz  \textit{Polarization Corrected Temperature} (PCT)), em que regiões contínuas $\geqslant$ 2000 km$^2$ com PCT $\leqslant$ 250 K foram os principais critérios para a identificação dos sistemas.

\sigla{name={PCT},description={\textit{Polarization Corrected Temperature}}}
\sigla{name={SCM},description={\textit{Sistema Convectivo de Meso-escala}}}
\sigla{name={LI},description={\textit{Linha de Instabilidade}}}

Combinando dados do PR e o \textit{TRMM Microwave Imager} (TMI) abordo do satélite TRMM, \citeonline{Nesbitt2000} desenvolveu uma metodologia para selecionar sistemas precipitantes denominados como \textit{Precipitation Features} (PFs), em que o principal critério de seleção foi identificar área contínua de chuva na superfície, seja estimada pelas observações de radar ou micro-ondas quando os sistemas estiveram fora da varredura do PR. A parir desta metodologia, \citeonline{cecil2005} mostraram que apenas 2,4\% das PFs observadas pelo TRMM em todo globo entre 12/1997--10/2000 possuíram atividade elétrica. Na AS as PFs classificadas com as maiores taxas de raios, concentraram-se na região da Bacia do Prata.% associadas a Sistemas Convectivos de Meso-escala \cite{Velasco1987,Durkee2009}.   

\sigla{name={PFs},description={\textit{Precipitation Features}}}
\sigla{name={TMI},description={\textit{TRMM Microwave Imager}}}
\sigla{name={TRMM},description={\textit{Tropical Rainfall Measuring Mission}}}

Mais tarde, \citeonline{zipser2006} utilizou medidas dos sensores do TRMM associadas a intensidade convectiva das PFs, estimando os locais das tempestades elétricas mais severas do globo. Apenas 0,1\% da amostragem das PFs (período de 7 anos) foram observadas taxa de raios acima de 32,9 min$^{-1}$.  Entre as regiões do globo com as maiores concentrações de PFs que indicaram valores extremos (0,001\%), seja de taxa de raios, seja de mínimas temperaturas de brilho (85 e 37 GHz) ou de máxima altitude com 40 dBZ de fator de refletividade do radar, encontra-se a região Sul da AS que engloba a Bacia do Prata e o extremo Norte da Cordilheira dos Andes que abrange a Colômbia e região do Lago Maracaibo na Venezuela. 
%Conforme aumenta a intensidade das tempestades elétricas, menor será sua probabilidade de ocorrência. Apenas 0.1\% das PFs observou-se taxa de raios acima de 32.9 por minuto. 

Com base em estudos a respeito da severidade dos sistemas como em \citeonline{doswell2001,brooks2003,zipser2006}, em que as medidas  associadas a intensidade convectiva indicaram os locais com condição de tempo severo, o estudo da morfologia da estrutura tridimensional da precipitação observadas pelo PR associado com a taxa de raios observada pelo LIS, pode esclarecer informações a respeito da intensidade convectiva, de modo a verificar a relação entre a taxa de raios e a severidade dos sistemas. 


\section{INTENSIDADE CONVECTIVA E A PRECIPITAÇÃO TRIDIMENSIONAL OBSERVADA POR RADAR}
\label{introRadar}

A Refletividade do Radar %($\eta$) 
\begin{equation}
\eta = \sum_{i=1, 2, 3 ... }^{\Delta V} \sigma_i,
\label{refletividade}
\end{equation}
é a somatória da seção de retro-espalhamento ($\sigma$) dos hidrometeoros contidos em um elemento de volume ($\Delta V$) iluminado -- \textit{gate} -- do feixe do radar \cite{battan1973}.

\simbolo{name={$\eta$},description={Refletividade do Radar}}
\simbolo{name={$\Delta V$},description={Elemento de volume ($\Delta V$) iluminado pelo radar}}
\simbolo{name={$\sigma$},description={seção de retro-espalhamento}}

Considerando que o parâmetro de tamanho ($\alpha$), dado pela relação 
\begin{equation}
\alpha = \dfrac{2\pi R_{h} }{\lambda},
\label{parametroTam} 
\end{equation}
em que ($2\pi R_{h}$) é a área da seção transversal do hidrometeoro  precipitável na atmosfera e ($\lambda$) o comprimento de onda emitido pelo radar, quando $\alpha$ $<<$ 1, o espalhamento da radiação eletromagnética pelas partículas será do tipo Rayleigh. Neste caso, a seção transversal de retro-espalhamento $\sigma$ pode ser escrita como 
\begin{equation}
\sigma = \dfrac{\lambda^2 \alpha^6}{\pi} |K|^2,
\label{sigma}
\end{equation}
em que $|K|^2$ corresponde ao índice de refração dos hidrometeoros. Como a equação \ref{parametroTam} depende do raio $R_h$, podemos reescrever a equação \ref{sigma} considerando o diâmetro 
\begin{equation}
D_h = 2 R_h.
\label{diametro}
\end{equation}

\simbolo{name={$\alpha$},description={Parâmetro de tamanho}}
\simbolo{name={$\pi$},description={Relação entre o perímetro e o diâmetro da circunferência}}
\simbolo{name={$R_{h}$},description={Raio do hidrometeoro}}
\simbolo{name={$\vert K \vert$},description={Grandeza relacionada ao índice de refração}}
\simbolo{name={$D_h$},description={Diâmetro do hidrometeoro}}

Então, substituindo as equações \ref{parametroTam} e \ref{diametro} em \ref{sigma}, obtemos que
\begin{equation}
\sigma = \dfrac{\pi^5 K^2  }{ \lambda^4 } D_h^6.
\label{sigma2}
\end{equation}

Considerando o hidrometeoro como esférico, podemos relacionar o $D_h$ com a sua quantidade de massa, sendo  
\begin{equation}
D_h = \left( \dfrac{6 M_h}{\pi \rho} \right)^{\frac{1}{3}},
\label{dh}
\end{equation}
em que, $\rho$ é a densidade e $M_h$ a massa do hidrometeoro.
\simbolo{name={$M_h$},description={Massa do hidrometeoro}}
\simbolo{name={$\rho$},description={Densidade}}

Então substituindo a equação \ref{dh} em \ref{sigma2}, obtém-se
\begin{equation}
\sigma = \dfrac{36 \pi^3 |K|^2  }{ \lambda^4 \rho^2 M_h^2} .
\label{sigma3}
\end{equation}

Portanto, observe que a Refletividade do Radar (equação  \ref{refletividade}), depende  de uma relação entre quantidade de massa $M_h$, densidade $\rho$ e também do índice de refração $|K|^2$ dos hidrometeoros, conforme mostra a equação \ref{sigma3}. Entretanto, o radar mede apenas a potência do sinal retro-espalhado ($P_r$). Logo, pode-se ter uma ideia da concentração de obstáculos espalhadores, porém não podemos ter certeza a respeito da massa, índice de refração e densidade dos alvos. 

\simbolo{name={$P_r$},description={Potência recebida}}

O que se faz para estimar a chuva é considerar que todos os alvos associados ao espalhamento do feixe do radar são gotas de água líquida esféricas. 

Neste caso, podemos combinar as equações \ref{refletividade} e \ref{sigma2}, obtendo $\eta$ em função do Fator de Refletividade  do radar (Z), em que 
%\begin{equation}
%P_r = V_{m} \dfrac{P_t G^2 \lambda^2}{2\ln2(4\pi)^3 r^4}   \dfrac{\pi^5 |K|^2  }{ \lambda^4 } \sum_{i=1, 2, 3 ... }^{V_{m}}  D_{h_i}^6.      
%\end{equation} 
%\begin{equation}
%P_r = \dfrac{P_t G^2 \lambda^2 \phi \varphi H}{ 512 (2\ln2)\pi^2 r^2} \dfrac{\pi^5 |K|^2  }{ \lambda^4 }  \sum_{i=1, 2, 3 ... }^{\Delta V}  D_{h_i}^6.   
%\end{equation} 
%\begin{equation}
%P_r = \dfrac{\pi^3 P_t G^2  \phi \varphi H }{ 512 (2\ln2) %\lambda^2 } \dfrac{ |K|^2  }{ r^2 }  \sum_{i=1, 2, 3 ... }^{\Delta %V}  D_{h_i}^6, 
%\end{equation} 
\begin{equation}
\eta =  \dfrac{\pi^5 |K|^2  }{ \lambda^4 } \sum_{i=1, 2, 3 ... }^{\Delta V} D_{h_i}^6,
\end{equation}
sendo o Fator de Refletividade do Radar
\begin{equation}
Z =  \sum_{i=1, 2, 3 ... }^{\Delta V}  D_{h_i}^6.
\label{fz}
\end{equation}


Conforme a equação do radar descrita em \citeonline{battan1973}, cada medida de $P_r$ depende de parâmetros fixos como ganho da antena, potencia do sinal transmitido, largura de pulso, efeito de lóbulo, comprimento de onda e ângulo sólido associado ao feixe emitido. Considerando que todas as constantes que envolvem a equação do radar equivalem a $C$ e $r$ sendo a distância entre o radar e o alvo espalhador, temos que a cada \textit{gate} do radar, $P_r$ será  
\begin{equation}
P_r = C \dfrac{|K|^2}{r^2}  Z .
\end{equation}

Desta maneira, a partir da $P_r$ medida pelo radar, sabendo a distância $r$ do alvo e considerando que a chuva é composta de esferas de água líquida ($K_{\mathrm{agua}}^2=0,931$), então podemos determinar 
\begin{equation}
Z = \dfrac{P_r r^2}{C |K|^2}.
\label{zsimples}
\end{equation}
%pois, $|K|^2=0.931$ para água líquida.
Conforme mostra a equação \ref{fz}, $Z$ depende de $D_h$ que está associado com a massa ou o volume de chuva pela equação \ref{dh}.

\simbolo{name={$Z$},description={Fator de refletividade do radar}}

No entanto, ao observar o perfil vertical do fator de refletividade do radar $Z$, verifica-se que feixe atinge regiões na atmosfera com temperaturas abaixo de 0 $^{\circ}$C. Nestes casos, a potência $P_r$ estará associada ao espalhamento em gelo de nuvem em vez da água líquida. Mesmo conhecendo as distâncias $r$ dos alvos espalhadores, não podemos afirmar sobre a temperatura da atmosfera para cada distância $r$ do radar, bem como se haverá água super-resfriada acima de 0 $^{\circ}$C ou gelo sólido caindo na superfície. Então, os dados brutos das observações de radar, consideram $|K|^2$ como constante, geralmente $\vert K_{\mathrm{agua}}\vert^2 = 0.931$. 

%Portanto, acima da da isoterma de 0 $^{\circ}$C, espera-se uma diminuição abrupta de $Z$ devido a mudança da constante dielétrica da água associada ao seu congelamento, sendo que a equação do radar mantem $|K|_{\mathrm{agua}}^2$ fixo para as medidas acima da isoterma de 0 $^{\circ}$C.

Aplicando $10\log_{10}$, na equação \ref{zsimples}, e assumindo  $\vert K_{\mathrm{agua}}\vert^2 = 0.931$ e $\vert K_{\mathrm{gelo}}\vert^2= 0.197$, para uma mesma medida de $P_r$, as diferenças observadas nos valores de $Z$ em dB irão corresponder a

\begin{align}
dBZ_{\mathrm{agua}}-dBZ_{\mathrm{gelo}} &=  10\log_{10}(\vert K_{\mathrm{gelo}}\vert^2 ) - 10\log_{10}(\vert K_{\mathrm{agua}}\vert^2)\\
dBZ_{\mathrm{agua}}- dBZ_{\mathrm{gelo}} &= -6,7 dBZ,
\end{align}
%dBZ_{\mathrm{gelo}} - dBZ_{\mathrm{agua}} = - 10\log_{10}(K_{\mathrm{gelo}}^2 ) + 10\log_{10}(K_{\mathrm{agua}}^2).
%Substituindo os valores de $\vert K_{\mathrm{agua}}\vert^2$ e $\vert K_{\mathrm{gelo}}\vert^2$, obtém-se que 

mostrando que devido ao índice de refração do gelo ser menor do que o índice de refração da água ($\vert K_{\mathrm{agua}}\vert^2 > \vert K_{\mathrm{gelo}}\vert^2$), ao considerar $\vert K_{\mathrm{agua}}\vert^2$ em regiões que os hidrometeoros estão congelados, haverá uma redução de 6,7 dBZ em relação a considerar $\vert K_{\mathrm{gelo}}\vert^2$.  

Nas observações de $Z$ no perfil vertical, a região ou camada de derretimento do gelo é bastante marcada, pois haverá uma redução de $\simeq$7 dBZ em regiões acima da isoterma de 0 $^{\circ}$C devido ao congelamento dos hidrometeoros, enquanto que logo abaixo da isoterma 0 $^{\circ}$C, haverá um aumento de $Z$ devido ao derretimento dos hidrometeoros. Este efeito foi explorado por  \citeonline{Fabry1995}, para analisar os processos  de  agregação, acreção e colisão coalescência, que foram observados a partir da espessura da camada de derretimento e flutuações nos valores do Fator de Refletividade $Z$ no perfil atmosférico. 

A espessura da camada de derretimento está relacionada com o lapse-rate da atmosfera \cite[p.~462]{mason1971_2ed}. Em uma atmosfera instável, com convecção profunda e precipitação convectiva, a camada de transição de fase de gelo para a água liquida é perturbada por correntes ascendentes. A mudança do índice de refração da água não ocorre apenas logo abaixo de 0 $^{\circ}$C, pois no ambiente convectivo teremos água super-resfriada em temperaturas de -15 $^{\circ}$C, o que intensifica o processo de acreção podendo gerar gelo sólido que cai derretendo até a superfície. Nestes casos espera-se uma camada de derretimento mais espessa.

Considerando um regime de precipitação estratiforme, que é governado por processos de agregação, será observado um aumento acentuado no fator de refletividade do radar logo abaixo da isoterma de 0 $^{\circ}$C associado ao derretimento de flocos de neve, que denomina-se banda brilhante. Neste caso espera-se uma camada de derretimento menos espessa, pois os flocos de neve possuem velocidade terminal e densidade inferior as partículas de gelo compacto (ganizo ou saraiva/$graupel$) portanto derretem mais rapidamente, ou seja, percorrem um caminho menor durante o derretimento. 

\simbolo{name={$^{\circ}$C},description={Grau Celsius}} 

%\begin{xalignat}{3}
%\mathbf{n} \cdot \mathbf{E} = 0 && &e  && \mathbf{n} \cdot \mathbf{B} = 0.
%\end{xalignat}

%\begin{equation}
%|K|^2 = \left( \dfrac{m^2-1}{m^2+2}\right)^2
%\end{equation}

Também, sabendo que $Z$ é proporcional ao diâmetro dos hidrometeoros $D_h$ elevado a 6 potência, como mostra a equação \ref{fz}, os processos de crescimento de flocos de neves, granizo e gotas, são marcados por aumentos exponenciais no Fator de Refletividade do Radar ($Z$) no perfil de altitude. 

%Considerando um regime de precipitação estratiforme, que é governado por processos de agregação, será observado um aumento acentuado no fator de refletividade do radar em logo abaixo da isoterma de 0 $^{\circ}$C associado ao derretimento de flocos de neve. \simbolo{name={$^{\circ}$C},description={Grau Celcius}} 

Acima da região de derretimento, um aumento abruto nos valores de $Z$ podem indicar processo de crescimento de cristais de gelo e de gelo compacto. Enquanto que abaixo da região de derretimento, os acréscimos nos valores de $Z$ podem indicar processos de colisão coalescência e os decréscimos de $Z$, devido a evaporação e rompimento/quebra das gotas. 


% durante o caminho que a precipitação percorre até a superfície ou temperaturas acima de 0°C.
%Na figura \ref{fabry}, \citeonline{Fabry1995}
%e o trabalho de 
%\begin{figure}[hbp]
%  \centering{
%  \subfloat[\cite{Fabry1995}]{{\includegraphics[scale=0.25]{img/ilustracoes/fabry}} \label{fabry}}
%  \subfloat[\cite{Takahashi2002}]{{\includegraphics[scale=0.35]{img/ilustracoes/takahashi}} \label{taka}}
%  }
%\caption{Fabry Taka}
%\label{fabyTaka} 
%\end{figure} 

%Consequentemente, a taxa de raios associa-se com a intensidade convectiva devido a acreção\footnote{A acreção é o processo de \textit{rimming} descrito no trabalho de \citeonline{Takahashi1978}.} ser o processo mais eficiente de eletrificação de nuvens, principalmente quando há presença de flocos de neve embebidos na região de fase mista \cite{Takahashi1978,Takahashi2002}. 

\section{PROCESSOS DE ELETRIFICAÇÃO DAS NUVENS}

Os processos de eletrificação das nuvens são intrínsecos ao processo de desenvolvimento da precipitação, especialmente em regiões com temperaturas entre -5 $^{\circ}$C e -40 $^{\circ}$C, portanto, está fortemente relacionado ao crescimento do gelo de nuvem \cite{mason1953}. 

A partir de medidas em superfície, observa-se que os campos eletrostáticos produzidos pelas tempestades elétricas são da ordem de dezenas de milhares de volts por metro, que correspondem a centros de cargas nas nuvens com dezenas de coulombs. 
\citeonline{williams1989} mostra uma síntese de trabalhos com medidas de campo eletrostático de tempestades elétricas no período entre 1752 e 1989 relatando que na maioria das observações a curva de campo elétrico observada correspondia com a perturbação causada por uma estrutura tripolar de cargas nas nuvens, havendo um centro de carga positivo na parte superior, um centro de carga negativa na região central e um centro de carga positiva menos intenso na base da nuvem conforme mostra a figura \ref{fig:tripeletr}.


\begin{figure}[ht]
\centering 
\includegraphics[width=\textwidth]{img/ilustracoes/tripoloeletr}
\caption{Representação do tripolo eletrostático de uma tempestade elétrica. A medida de campo eletrostático na superfície (parte inferior da figura), corresponde aos centros de cargas $Q_{+}$, $Q_{-}$ e $q_{+}$ (adaptada de \citeonline{ogawahandbook}).}
\label{fig:tripeletr}
\end{figure}


Mesmo que o modelo do tripolo eletrostático proposto por \citeonline{williams1989} seja uma teoria condizente com a estrutura de cargas dominante em uma tempestade elétrica, considerando sondagens de campo elétrico no interior das tempestades \citeonline{rust1996}, verificam que os centros de cargas podem estar distribuídos de maneira mais complexa. 
\citeonline{stolzenburg1998} criou um modelo conceitual para a estrutura de cargas das tempestades elétricas exposto na figura \ref{fig:multipcentros}, sugerindo que nas regiões aonde ocorrem ventos ascendentes pode haver 4 ou mais centros de carga enquanto as regiões com correntes descendentes 6 ou mais centros de cargas.


\begin{figure}[ht]
\centering 
\includegraphics[width=\textwidth]{img/ilustracoes/nuvem}
\caption{Estrutura elétrica de uma tempestade elétrica idealizada a partir de sondagens de campo eletrostático realizadas no interior de nuvens de tempestades (adaptada de \citeonline{stolzenburg1998}).}
\label{fig:multipcentros}
\end{figure}

%Experimentos relacionados a simulação de nuvens em laboratório mostram que a presença de gelo é fundamental no processo e que a carga adquirida pelos granizos depende da temperatura do ambiente, do conteúdo de água líquida da nuvem, da velocidade de colisão entre os hidrometeoros e dos tamanhos dos cristais de gelo \cite{Takahashi1978,saunders2008}. 

De modo a explicar os intensos campos elétricos associados as nuvens de tempestades elétricas, e o confinamento dos centros de cargas em regiões com temperaturas entre -5 $^{\circ}$C e -40 $^{\circ}$C, as teorias de eletrificação de hidrometeoros podem ser divididas em duas grandes frentes teóricas: Eletrificação por Convecção e a Eletrificação por Precipitação.

%Os processos de eletrificação das nuvens não é completamente entendido ou descrito na literatura devido a sua complexidade, que envolve desde fenômenos em meso-escala, por exemplo a convergência de massas de ar até as propriedades físico-químicas da água.  


\subsection{Eletrificação por Convecção}
\index{Teoria!Eletrificação por Convecção}

Pressupõe-se que as cargas elétricas são geradas por fontes externas às nuvens, associado a ionização de moléculas do ar atmosférico por átomos radioativos na superfície terrestre ou por radiação cósmica \cite{wilson1956,grenet1947, vonnegut1962,phillips1967}.

Devido a esta distribuição de íons livres na atmosfera, o campo elétrico de bom tempo, atrai os íons positivos para próximo a superfície terrestre e quando há rompimento da estabilidade atmosférica, o movimento ascendente transporta os íons positivos próximos a superfície terrestre para o interior das nuvem, como ilustrado na figura \ref{fig:elec-a}. Conforme a nuvem se desenvolve verticalmente, íons negativos são atraídos pelas cargas positivas introjetadas na nuvem tornando o topo da nuvem negativamente carregado, como ilustra a figura \ref{fig:elec-b}. Com o acumulo de íons negativos no topo das nuvem e a atuação de correntes descendentes ocorre estranhamento lateral das cargas negativas do topo da nuvem, concentrando regiões de cargas negativas próximas a base da nuvem nas regiões laterais intensificando a atração de íons positivos a partir da superfície, como ilustra a figura \ref{fig:elec-c} \cite{vonnegut1962,wagner1981,vonnegut1995}.

%Conforme descrito por \citeonline{vonnegut1995}, o campo elétrico de tempo bom pode concentrar íons positivos na baixa atmosfera. Na ocorrência de térmicas os íons positivos podem ser transportados para o interior das nuvens eletrificandos-as positivamente, conforme pode ser visualizado na figura \ref{fig:elec-a}. Com o crescimento vertical da nuvem e o excesso de cargas positivas,  íons  negativos são atraídos tornando o topo da nuvem negativamente carregado (ver figura \ref{fig:elec-b}).
% Esse mecanismo de eletrificação pode ser produzido pela distribuição de íons livres na atmosfera \cite{wilson1956,phillips1967}  \cite[apud \cite{vonnegut1995}]{grenet,wagner1981}. 

\begin{figure}[ht]
   \centering
   \subfloat[Íons positivos injetados pela convecção.]{\includegraphics[width=5cm]{img/ilustracoes/eleconvectiva-a}\label{fig:elec-a}} 
   \subfloat[Íons negativos são atraídos.]{\includegraphics[width=5cm]{img/ilustracoes/eleconvectiva-b} \label{fig:elec-b}}  
   \subfloat[Retro-alimentação positiva devido ao efeito corona.]{\includegraphics[width=5.8cm]{img/ilustracoes/eleconvectiva-c} \label{fig:elec-c}}
   \caption{Representação do processo convectivo de eletrização.}
   \label{fig:elec}
\end{figure}

\citeonline{phillips1967,vonnegut1991}, mostram evidências que comprovam a participação da Eletrização por Convecção na eletrificação das nuvens. Porém verifica-se que a disponibilidade de íons da atmosfera não é suficiente para proporcionar centros de cargas tão intensos conforme se observa nas medias de campo eletrostático na superfície. 

\subsection{Eletrificação por Precipitação}
\index{Teoria!Eletrificação por Precipitação}

A teoria de Eletrificação por Precipitação, parte do pressuposto de que os centros de cargas são gerados por processos no interior da nuvem, associado com a interação entre os hidrometeoros de nuvens.

Em função dos diferentes níveis de temperatura pressão de vapor e correntes ascendentes e descendentes, o ambiente de nuvem proporciona o crescimento de gotas com diferentes tamanhos, cristais de gelo diversos e granizo de diferentes tamanhos. Com a atuação da força gravitacional, as correntes ascendentes, descendentes e a força de arraste, os hidrometeoros adquirem velocidades diferenciadas e colidem entre si.

Ao colidirem, poderá haver transferência de cargas entre as partículas de nuvens, especialmente se o tempo de colisão for  pequeno ou se as partículas se quebram. Durante as colisões, a eletrificação poderá ocorrer por Processo de Colisão Indutivo ou Processo de Colisão Não-indutivo.

\subsubsection{Processo de Colisão Indutivo} 
\index{Processo!de Colisão Indutivo}

Ocorre quando não há coalescência ou acreção portanto, torna-se mais provável na colisão entre o \textit{graupel}\footnote{Granizo com diâmetro menor que 2 mm.} e cristais de gelo.

As colisões ocorrem sobre a influência de um campo elétrico $\mathbf{E}$ já existente na nuvem, provocando a polarização do \textit{graupel} e dos cristais de gelo conforme na figura \ref{fig:ind-a}. Ao colidirem, cargas são transferidas por contato, tornando o \textit{graupel} carregado com carga negativa e os cristais de gelo com carga positiva ou falta de elétrons, conforme é mostrado na ilustração da figura \ref{fig:ind-b}.

\begin{figure}[ht]
   \centering
   \subfloat[Indução elétrica causada por um campo elétrico $\mathbf{E}$.]{\includegraphics[height=10cm]{img/ilustracoes/indutivopdf-a} \label{fig:ind-a}}  
   \subfloat[Colisão entre o \textit{graupel} e o cristal de gelo e eletrização.]{\includegraphics[height=10cm]{img/ilustracoes/indutivopdf-b} \label{fig:ind-b}}
   \caption{Ilustração da eletrização de hidrometeoros por Processo de Colisão Indutivo.}
   \label{fig:ind}
\end{figure}

Porém, \citeonline{macgorman1998} apontam que a intensidade do campo elétrico de bom tempo, não possui intensidade suficiente para polarizar as partículas de gelo de nuvem. Então, o Processo de Colisão Indutivo deve ocorrer após um mecanismo de eletrização não-indutivo promover um campo elétrico $\mathbf{E}$ no interior da nuvem. 

\subsubsection{Processo de Colisão Não-indutivo}
\index{Processo!de Colisão Não-indutivo}

Conforme diversos estudos dos mecanismos de eletrificação de nuvem realizados em laboratórios que buscaram recriar as condições atmosféricas relacionadas ao crescimento dos hidrometeoros, foi constatado que o carregamento não-indutivo depende: do tamanho dos hidrometeoros, do conteúdo de água líquida dentro da nuvem, da temperatura, e da velocidade de impacto entre os hidrometeoros  \cite{reynolds1957, Takahashi1978,baker1994,Saunders1999,pereyra2000}.  \citeonline{Takahashi1978} e \citeonline{Takahashi2002}, apontam que o mecanismo mais eficiente de na eletrização dos hidrometeoros por Processo de Colisão Não-indutivo envolve a colisões entre o \textit{graupel}/granizo com neve seca. 


Associado ao tamanho das partículas considera-se que exista uma interface entre a água e o ar, gelo e ar, e água e gelo, composta de água quase líquida (QLL)\sigla{name={QLL},description={\textit{Quasi-liquid Layer}}}, que no momento da colisão, é transferida das partículas com maior quantidade de QLL para as partículas com camada de QLL menor \cite[apud \cite{rachel}]{baker1994}.
Em \citeonline{saunders2008} são mostrados resultados de experiências controladas em laboratório nas quais analisa-se a polaridade adquirida pelo \textit{rimer} em função da temperatura e presença de água líquida (ver figura \ref{fig:saunders}). 

\begin{figure}[htp]
\centering 
\includegraphics[width=9cm]{img/ilustracoes/saunders}
\caption{Cada resultado obtido desenha uma linha de fronteira entre as regiões nas quais a temperatura e quantidade de água líquida influenciam na carga, positiva ou negativa, adquirida pelo \textit{rimer} (adaptada de \citeonline{saunders2008}).}
\label{fig:saunders}
\end{figure}

O estudo da microfísica dos processos de eletrificação não indutiva apresenta pouco consenso entre os pesquisadores, conforme mostra a figura \ref{fig:saunders} em que os resultados obtidos por diferentes autores, divergem. Porém, é o processo que melhor compreende a estrutura tripolar das nuvens de tempestades, pois, converge com o fato de que os centros de cargas identificados no interior das nuvens correspondem à sedimentação de partículas de mesma massa e mesma polaridade elétrica.  
 
\section{PROPOSTA}
Nesta tese, faz-se a identificação de sistemas que possuem atividade elétrica, ou seja, tempestades elétricas, apenas sobre a América do Sul  a partir das observações orbitais do satélite TRMM, mais especificamente do sensor de raios (LIS), radiômetro no infravermelho (VIRS) e o radar de precipitação (PR) a bordo do da satélite entre os anos de 1998 e 2011. Desta forma cria-se um banco de dados de tempestades elétricas do TRMM.

Com este subconjunto de dados do TRMM, é estudada a sazonalidade, o ciclo diurno e ciclo anual das tempestades elétricas, bem como a densidade geográfica de raios e de tempestades elétricas, buscando evidenciar regiões ou estações do ano em que as tempestades elétricas possuem processo de eletrificação mais eficientes sobre a América do Sul.

A intensidade das tempestades elétricas é estudada com base na taxa de raios e aspectos morfológicos como: dimensões relacionadas a sua extensão e a estrutura tridimensional da precipitação observada pelo PR, buscando identificar qual é a taxa de raios que corresponde potencialmente a condições de convecção profunda e consequentemente de tempo severo, conforme cada localidade da extensa região da AS.

%A estrutura tridimensional da precipitação observada pelo PR é estuda com base na probabilidade de ocorrência por altitude, conforme descreve \cite{yuter1995}, e também conforme a probabilidade de ocorrência por níveis de temperatura do perfil atmosférica. As seções \ref{introRadar} e \ref{derretimento} fazem uma revisão de conceitos fundamentais que devem ser considerados na interpretação dos diagramas de probabilidade elaborados associados as observações por radar.
 
%------------------------------------------------------\\
%\textit{Acho que essa parte de baixo é Metodologia... ou discussão dos resultados}

%A grande extensão territorial do Brasil na America do Sul  
%m base nos experimentos do LBA, investigar a morfologia dos sistemas na pré monsão, na região amazônica. 

%As descargas na região amazônica parecem estar mais associadas com os processos de inibição do que precipitação... Em uma região tropical dominada por processos quentes, as descargas podem indicar inibição de colisão coalescência, desenvolvimento de fase fria e menos precipitação.

%mas no período úmido de leste raios relacionam-se mais com a precipitação.

%...A Rachel já fez uma boa discussão sobre a microfísica dos sistemas da amazônia, períodos seco úmido e de transição. As tempestades foram organizadas em clusters, estudou-se o ciclo de vida, a ocorrência de raios em áreas desmatadas e com floresta/outras, e foi explicado a microfísica dos sistemas basicamente com: taxa de raios positivos e negativos, eco tops, VIL, CAPE, CINE. Falta explotar a os CFADS para essa região. Como varia a probabilidade de ocorrência por altitude dos perfis de refletividade nos períodos seco de transição e úmido.

%Em desenvolvimento ...
%As tempestades mais eficientes estão mostrando tamanhos diversos. Tanto cluster grandes quanto pequenos podem ser eficientes. Não é uma relação que depende apenas da área, ou da fração convectiva ou estratiforme. Os raios relacionam-se com o ciclo de vida, que no caso é aleatório. Depende do ciclo de vida

%Relação exponencial entre max fl no pixel versus fl-rate/km2. Maior a concentração de raios em um único pixel, menos eficiente e mais chuva. Talvez uma reintensificação do sistema maduro. Mas existem duas categorias de sistemas :
%1 – os mais eficiente com área ~10^3 e chuva ~10^5
%2 – as com maior fl/rate no pixel, menos eficientes e com chuva ~10^9



%Com o experimento de campo LBA (Large-Scale Biosphere-Atmosphere Experiment in Amazonia) realizado na região de Rondônia entre janeiro e fevereiro de 1999, foi possível identificar alguns fatores importantes que regulam a precipitação na região Amazônica. Além disso o LBA foi importante para validação de dados do satélite TRMM que são amplamente utilizados nesta pesquisa \cite{silva2002lba,williams2002,albrecht2011}.

%Silva Dias M. A. F. et al, (2002), fazem uma síntese dos principais resultados e objetivos do LBA, entre estes destaco os estudos de Anagnostou e Morales, (2002), \citeonline{Carvalho2002} que mostram dois regimes de vento em 700 mb, de Leste e de Oeste, em que observou-se maior precipitação convectiva e atividade elétrica durante o regime de ventos de Leste. Petersen W. A. et al, (2002), investigaram como que esses dois regimes de vento (Leste-Oeste) observados durante o LBA em Rondônia, influenciam no número de descargas elétricas observadas pelo LIS (Ligthning Image Sensor), não apenas para região Amazônica mas para toda a América do Sul durante 4 verões entre 1997 e 2000.

%A variação intra-sazonal da atividade elétrica durante o período chuvoso mostrou-se evidente. \citeonline{petersen2002trmm},  identificaram regiões de extremos opostos de atividade elétrica que devem estar associados ao mecanismos de manutenção da monção na América do Sul, principalmente com a dinâmica que envolve Zona de Convergência do Atlântico Sul (ZCAS) \cite{CarvalhoJones2002,Carvalho2002}.   

%\section{OBJETIVOS... PROPOSTA...}
%\begin{itemize}
%\item Criar um banco de dados de tempestades elétricas do TRMM. 
%\item Criar mapas que identifique a densidade de tempestades elétricas e de raios sobre a América do Sul.
%\item Descrever o ciclo diurno e o ciclo anual das tempestades elétricas do TRMM.
%\item Classificar a intensidade das tempestades elétricas com base na taxa de raios e no estudo da frequência de ocorrência do Fator de Refletividade do radar por temperatura e por altura. 
%\end{itemize}
%...
\chapter{METODOLOGIA}

Consiste fundamentalmente na construção de um subconjunto de dados provindos das observações dos sensores abordo do satélite TRMM, que estiveram em orbita planetária entre 1998 e 2011. 

As informações dos diferentes sensores foram combinadas de maneira à identificar-se sistemas denominados como Tempestades Elétricas.


\begin{itemize}
\item Orbital TRMM LIS, VIRS (1B01), and PR (2A25) data from 1998-2011. 
\item NCEP RII reanalysis from 1998-2011: geopotential height and temperature in 17 pressure levels.
\item Region over South America -- SA: 40S-10N and 90-30W.
\item 68,230 TRMM orbits and 154,189 thunderstorms found, only 96,281 thunderstorm had a least one valid PR profile.
\item Thunderstorms have been definite as clouds with brightness temperature below 258 K in the 1B01 10.8 $\mu$m channel and had at least one LIS flash [Morales and Anagnostou, 2003]\footnote{Morales, C. A., and E. N. Anagnostou, Extending the capabilities of high-frequency rainfall estimation from geostationary-based satellite infrared via a network of long-range lightning observations, J. Hydrometeor, 4, 141–159, 2003.}.
\item 3D precipitation structure were studied using Contour Frequency of PR reflectivity by Altitude and by Temperature Diagrams.
\end{itemize}



Para melhor entender as implicações que envolvem a construção de uma base de dados de sistemas individualmente a partir das observações do TRMM, inicialmente descreve-se algumas das principais características operacionais do satélite TRMM.

\section{O SATÉLITE TRMM}

O satélite TRMM (\textit{Tropical Rainfall Measuring Mission})\sigla{name={TRMM},description={\textit{Tropical Rainfall Measuring Mission}}}  faz parte de uma missão conjunta entre a NASA (\textit{National Aeronautics and Space Administration} - EUA)\sigla{name={NASA},description={\textit{National Aeronautics and Space Administration}}} a JAXA (\textit{Japan Aerospace Exploration Agency}) \cite{simpson1988}. Os instrumentos a bordo do TRMM são; radar de precipitação (PR), radiômetro de microondas (TMI),  radiômetro no visível e no infravermelho (VIRS), sistema de energia radiante da terra e das nuvens (CERES) e sensor para imageamento de relâmpagos (LIS) \cite{kummerok1998}. 

\sigla{name={JAXA},description={\textit{Japan Aerospace Exploration Agency}}}

Esse satélite possui uma órbita de aproximadamente 320 Km de altura e inclinação de 30$^{\circ}$-35$^{\circ}$ para que possa visitar uma mesma região duas vezes ao dia, em horários distintos, sobre a região tropical do planeta Terra \cite{simpson1988}.   

\subsection{Radar de Precipitação}

O PR (\textit{Precipitation Radar}) é um radar que opera na frequência de 13,8 GHz e possui uma resolução horizontal entre 4,3-5 km, 250 m de resolução vertical e uma varredura 215 km. Uma de suas características mais importantes é a capacidade para fornecer a estrutura tridimensional dos hidrometeoros de nuvens, desde a superfície até uma altura de 20 km \cite{kummerok1998}. Para esta pesquisa serão utilizados os dados 2A25 que apresentam o fator de refletividade do radar corrigido por atenuação da chuva \cite{2A25}.
\sigla{name={PR},description={\textit{Precipitation Radar}}}

\subsection{Imageador de relâmpagos}

O LIS (\textit{Lightning Imaging Sensor}) é um sensor óptico capaz de detectar e localizar relâmpagos em tempestades individuais, analisando a emissão óptica resultante da dissociação, excitação e recombinação dos constituintes atmosféricos, em resposta a ocorrência de descargas atmosféricas. Este sensor CCD\footnote{Um dos dispositivos eletrônicos utilizados para registro de imagens em câmeras digitais.}, que trabalha no comprimento de onda de 772 nm, identifica descargas nuvem-solo e intranuvens, tanto no período diurno quanto noturno, a partir da amostragem de 500 imagens por segundo. Combinado com a velocidade do satélite (11 km/s) e abertura da CCD, o sensor LIS possui um campo de visão que permite a observação de um ponto na Terra por 80 a 90 s, tempo suficiente para a estimativa da taxa de raios de uma tempestade no momento da observação \cite{christianTM,trmmhandbook}.
\sigla{name={LIS},description={\textit{Lightning Imaging Sensor}}}

\subsection{Radiômetro no visível e infravermelho}

O VIRS (\textit{Visible and InfraRed Scanner}) é um radiômetro passivo que realizada medidas de radiância em 5 bandas espectrais, com comprimentos de onda de 0,63 $\mu$m, 1,61 $\mu$m, 3,75 $\mu$m, 10,8 $\mu$m e 12 $\mu$m. Sua resolução horizontal atinge 2,11 km no nadir e 720 km de varredura \cite{trmmhandbook}.

Nesta pesquisa, utilizamos apenas o canal  10,8$\mu$m, para estimativa da temperatura de topo de nuvens.
\sigla{name={VIRS},description={\textit{Visible and InfraRed Scanner}}}

\subsection{Radiômetro de microondas}

O TMI (\textit{TRMM Microwave Imager}) é um radiômetro passivo multicanal, 10,65 GHz, 19,35 GHz, 21,3 GHz, 37 GHz, e 85,5 GHz, com dupla polarização. Possui uma varredura cônica combinada com movimento de rotação de sua antena, a qual observa regiões elipsoidais quando projetadas na superfície \cite{kummerok1998}. Sua resolução horizontal varia entre 6-50 km, dependendo do ângulo entre o feixe e o nadir, e varredura de ~760 km \cite{trmmhandbook}. 
\sigla{name={TMI},description={\textit{TRMM Microwave Imager}}}

\section{FONTE DE DADOS}

A fonte de dados foi obtida utilizando a infra-estrutura de rede do IAG-USP, aonde os dados foram transferidos a partir do servidor de FTP da NASA (ftp://disc2.nascom.nasa.gov/ftp/data/s4pa). 

Os arquivos orbitais do TRMM na versão 7, produtos 1B01, 2A25 e 1B11 foram baixados para o período entre 1998 e 2011. Nesta etapa um conjunto de \textit{scripts} foi desenvolvido para download e verificação de integridade dos dados baixados. No total o volume de dados atingiu 28 TB.  %PR 16,5TB / VIRS 10TB / TMI 1,4TB /   

Os dados do LIS de \textit{flash}, \textit{group}, \textit{events} e \textit{view time} foram concedidos pela pesquisadora \citeonline{rachel}, quem já possuía essa base de dados no Brasil. 

Como as observações globais do PR, LIS, VIRS e TMI entre 1998-2011 representam um volume de aproximadamente 30 TB, a região de estudo foi limitada entre 10N-40S e 91W-30W. Portanto foi feito um recorte nos dados orbitais apenas para esta região que cobre toda a América do Sul, o que reduziu bastante o volume de dados a serem utilizados e tornou o processamento possível perante a infraestrutura computacional do IAG-USP.

\section{RAIOS COM DIFERENTES TAXAS DE DESCARGAS DE RETORNO}

O estudo da Morfologia das tempestades foi iniciado pela construção de um algoritmo que fez a extração de perfis verticais do fator de refletividade corrigida por atenuação ($Z_c$) \simbolo{name={$Z_e$},description={\textit{Fator de refletividade corrigida por atenuação, produto TRMM 2A25}}}, produto 2A25, nos pontos de grade onde ocorreram descargas atmosféricas observadas pelo LIS.

%Esse processamento envolveu uma quantidade de dados de aproximadamente 16 TB. Foram utilizados 4 HDs (Hard Disk) de 1 TB realizando \textit{striping} por \textit{RAID0} via software, permitindo o processamento integral na massa de dados do LIS e 2A25 fosse realizado em aproximadamente 7 dias. Sem esse planejamento de hardware o as extrações ficariam inviáveis, pois mesmo que o algoritmo não exija grande demanda de processador, a leitura e escrita em disco gerava I/O que davam \textit{kernel panic} no sistema.

Após a extração dos perfis verticais de $Z_e$ orientada pela ocorrência de raios, foi constituída uma base de dados com as seguintes características:

\begin{itemize}
\item Para cada raio observado pelo LIS existia um perfil vertical de refletividade do radar.
\item Além dos 80 níveis verticais de cada perfil de refletividade do radar, temos também a classificação do tipo de chuva identificada pelo produto TRMM 2A25 (convectiva, estratiforme, etc).
\item Cada raio (\textit{flash}) possui o seu respectivo número de eventos (pixels da CCD iluminados), número de grupo (grupos de \textit{pixels} iluminados na CCD que compõem o raio), e tempo de duração em milisegundos. 
\end{itemize}

A morfologia da estrutura 3D da precipitação observada pelo PR foi estudada para diferentes classes de perfis separados conforme o número de descargas de retorno (\textit{groups}) de cada raio (\textit{flash}). 

Nesta etapa foi investigada se a taxa de descargas de retorno representa maior definição de precipitação em altitude principalmente na região de fase mista, entre 5 e 7 km de altitude. 

\section{IDENTIFICAÇÃO DAS TEMPESTADES ELÉTRICAS}

Após uma análise ponto a ponto, buscando associar cada raio com um perfil de refletividade do PR, partimos para uma análise de grupo, buscando identificar quais as tempestades elétricas que representam maior intensidade convectiva.

Técnicas numéricas de mudança de eixo ordenados foram utilizadas para projetar as
observações orbitais do VIRS, PR e LIS em uma grade regular com 0,05$^{\circ}$ de resolução, a qual foi utilizada para verificar regiões com medidas coincidentes entre os sensores.

A equação de Planck foi aplicada nos dados de radiância espectral do produto 1B01, canal 4 do VIRS (10,8 µm), e áreas com temperaturas de corpo negro em infravermelho mais frias do que 258 K delimitaram os \textit{clusters} de nuvens. Após, o algoritmo verifica se houve raios detectados pelo LIS na mesma área da nuvem. Havendo pelo menos um raio, o sistema era classificado como uma tempestade elétrica. 

Desta forma, cada tempestade elétrica foi armazenada na forma de um arquivo HDF contendo medidas coincidentes do VIRS, LIS e PR. Os arquivos de tempestades elétricas são compostos pelas seguintes informações contidas nos produtos do TRMM:

\begin{itemize}
\item VIRS: 1B01 -- \textit{latitude, longitude, Radiance channel 4} (10,8 µm)
\item PR: 2A25 -- \textit{latitude, longitude, Corrected Z-factor, Rain Type} 
\item LIS: \textit{latitude and longitude of, flashes, groups, events and View Time}  
\end{itemize} 

Foram identificadas 154,189 tempestades elétrica e devido a varredura do PR ser menor do que a do VIRS, apenas 96,281 tiveram pelo menos um perfil de chuva válido observado pelo radar a bordo do satélite.


\section{A TAXA DE RAIOS POR TEMPESTADE ELÉTRICA}

A taxa de raios no tempo (FT), foi definida como a razão entre o número de flashes ($N_{fl}$) e o tempo médio ($VT_m$) em que o sensor LIS observou a tempestade elétrica, da mesma forma como foi calcula para as \textit{precipitation features} \cite{cecil2005, Nesbitt2000}. 

%Os pixels do view time do LIS de cada orbita foram projetados em uma grade regular com 0.25 de resolução 

A taxa de raios no tempo também foi normalizada pela área da tempestade elétrica ($A_t$), obtendo também o índice da taxa de raios no tempo por área (FTA). 

\begin{equation}
FT = \frac{N_{fl} }{VT_m} 60 ~[raios~minuto^{-1}]  
\label{eqFT}  
\end{equation}
%31557600 ano
\begin{equation}
FTA = \frac{N_{fl} }{VT_m A_t } 86400 ~[raios~dia^{-1}~km^{-2}]
\label{eqFTA}
\end{equation}

Para cada sistema foram calculados os dois índices que podem estar associados com a severidade de tempo, o FT e FTA, conforme mostra as equações \ref{eqFT} e \ref{eqFTA} 
\simbolo{name={FT},description={\textit{Taxa de raios por tempo $[raios~minuto^{-1}]$}}} \simbolo{name={FTA},description={\textit{Taxa de raios por tempo por área $[raios~dia^{-1}~km^{-2}]$}}}.

\section{DENSIDADES ESPACIAIS DE RAIOS E TEMPESTADES ELÉTRICAS}

Neste trabalho, buscamos identificar espacialmente as regiões mais eficientes nos processos de eletrificação, as quais possuem pouca densidade de sistemas porém alta densidade de raios em comparação com as demais regiões da América do Sul.

O que se torna fundamental na construção destes mapas é considerar quantas vezes, ou qual o tempo em que o satélite ficou observando cada parte da região de estudo. Qualquer análise de densidade espacial com dados do TRMM que não considere o número de passagens ou tempo em que o sensor observou a região projetada na superfície, será tendenciosa.

Mesmo que o satélite TRMM visite o mesmo lugar do globo duas vezes por dia em função de sua orbita inclinada 35° e velocidade, entre 1998 e 2011, o satélite passou 10,000 vezes mais sobre a região extra-topical do que na região tropical, como mostra a figura \ref{VirsVT}, com todas as orbitas e as varreduras do VIRS projetadas e acumuladas sobre a América do Sul. 

\begin{figure}[!Hb]
  \centering{
  \subfloat[Tempo de visada do LIS (0,25°). Valores convertidos em dias de observação.]{{\includegraphics[scale=0.9]{img/grids/vt_trmm}} \label{lisVT}}
  \subfloat[Número de passagens do VIRS (0,25°).]{{\includegraphics[scale=0.9]{img/grids/passagens_virs_1998-2011}} \label{VirsVT}}
  }

\caption{Observações do TRMM sobre a América do Sul.}
\label{gridVT} 
\end{figure} 

Fazendo o acumulado do tempo de visada do LIS na superfície, como mostra a figura \ref{lisVT}, observa-se que em 14 anos o LIS passou 10 dias a mais na latitude -34°S do que em 0°.

Na figura \ref{gridVT}, estão representadas duas matrizes que correspondem aos pontos de uma grade igualmente espaçada (grade regular), com 0,25° de resolução, projetada sobe a América do Sul. A matriz ($\mathbf{VT}_{lis}$) do tempo total da visada do sensor LIS sobre a superfície e a matriz ($\mathbf{VT}_{virs}$), do número de vezes que o satélite passou conforme o tamanho da varredura do radiômetro VIRS na superfície.  

Com as mesmas dimensões e resolução de grade que o tempo de observação e o número de passagens do satélite foram acumulados em duas matrizes, os raios foram acumulados na matriz ($\mathbf{FL}_{lis}$) e todos os píxeis do VIRS com radiância espectral associada com temperaturas de brilho inferiores a 258 K e que definiram as áreas das tempestades elétricas, foram acumulados na matriz ($\mathbf{P}_{te}$) que representa os locais com maior cobertura de nuvens de tempestades elétricas.

A matriz $\mathbf{FL}_{lis}$ projeta sobre a América do Sul está representada na figura \ref{gridFL} e a matriz $\mathbf{P}_{te}$, na figura \ref{gridTe}. Note que principalmente a figura \ref{gridTe} é notável o alto número de sistemas na região Sul da AS, com a mesma magnitude do que em locais ao Norte aonde atua a Zona de Convergência Intertropical. Mas esse máximo no Sul da AS não indica maior ocorrência de tempestades elétricas e sim maior frequência de passagem do satélite TRMM.

\begin{figure}[hbp]
  \centering{
  \subfloat[Acumulado de raios observados pelo LIS (0,25°).]{{\includegraphics[scale=0.9]{img/grids/densEspacial_19982011acumuladoTaxaFlashPolyfill}} \label{gridFL}}
  \subfloat[Acumulado das áreas de tempestade elétrica (0,25°).]{{\includegraphics[scale=0.9]{img/grids/densEspacial19982011acumuladoTempestadesPolyfill}} \label{gridTe}}
  }

\caption{Acumulados dos raios e áreas das 154,189 tempestades elétricas identificadas.}
\label{gridSistemas} 
\end{figure} 

Mesmo que as matrizes representem pontos em uma grade com espaçamento angular regular, as áreas de cada ponto de grade não são iguais, pois a  o comprimento de arco de 0.25° na direção zonal depende da latitude da região. Assim a matriz que corresponde a área da grade regular ($\mathbf{A}_g$) foi calculada e considera nos cálculos de densidades espaciais.


Portanto, a densidade espacial de raios ($\mathbf{DE}_{fl}$) é calculada conforme a equação \ref{defl}. Note que a razão de $\mathbf{FL}_{lis}$ por $\mathbf{VT}_{lis}$ e $\mathbf{A}_g$ é multiplicada por $24\times60\times60\times365,25$, o que converte o tempo de observação do LIS de segundos para anos. Então as densidades espaciais de raios, possuem dimensões de número de [raios] por [tempo] por [quilômetro quadrado].

No mesmo caminho as densidades espacias de tempestades elétricas ($\mathbf{DE}_{te}$) foram obtidas conforme a equação \ref{dete}. Porém a constante de conversão de tempo na equação \ref{dete} é diferente da equação \ref{defl}, pois o tempo que o VIRS observou a AS, foi estimado a partir do número de vezes que o satélite passou sobre a AS e considerando que cada ponto de grade na orbita foi observado por 90 segundos. Portanto ao converter a matriz $\mathbf{VT}_{virs}$ para segundos de observação, temos um fator de 90 no denominador, que está implícito na equação \ref{dete}.



\begin{equation}
\mathbf{DE}_{fl} = \frac{\mathbf{FL}_{lis}}{\mathbf{VT}_{lis} \mathbf{A}_g} 31557600 ~[raios~ano^{-1}~km^{-2}]    
\label{defl}
\end{equation}

\begin{equation}
\mathbf{DE}_{te} = \frac{\mathbf{P}_{te}}{\mathbf{VT}_{virs} \mathbf{A}_g} 350640 ~[sistemas~ano^{-1}~km^{-2}]    
\label{dete}
\end{equation}


\section{MORFOLOGIA DA PRECIPITAÇÃO}

descrever CFAD CCFD CFTD

A morfologia da precipitação observada pelo PR foi estudada por meio de diagramas de contorno de frequência de ocorrência por altitude, os CFADs, utilizando os perfis de refletividade efetiva ($Z_{ef}$) do produto 2A25.

Conforme descreve \cite{yuter1995}, primeiramente obtivemos histogramas bidimensionais (hist2D) da densidade de probabilidade de ocorrência de $Z_{ef}$ por nível de altitude com tamanho de bin de 1 dBZ para cada 250 m de altitude. Cada nível de altitude do hist2D foi normalizado pelo total de ocorrência de refletividade (N$_{Zef}$) do respectivo nível.

Para cada diagrama foi calculado o percentual (\%) de perfis classificados como convectivo, estratiforme e outros, (P) o número de perfis que compõe o CFAD, (L) o número de ocorrência de refletividade do nível de máxima ocorrência e (H) a altitude do nível de máxima ocorrência. Ao plotar os contornos dos CFADs, os níveis de altitude que representaram menos do que 10\% de L foram desconsiderados.

Com base nos dados de Reanálises rII do NCEP, o eixo de altitude dos perfis do radar foram projetados em um eixo de temperatura conforme os campos de altura geopotencial e temperatura, por nível de pressão.

Desta forma desenvolvemos o estudo das frequências de ocorrência de refletividade por nível de temperatura, construindo diagramas que foram denominados como Diagramas de Contorno de Frequência por Temperatura (CFTD) e Diagramas de Contorno de Frequência Cumulativa por Temperatura (CCFTD).








\chapter{MARCO DAS TEMPESTADES ELÉTRICAS NA AMÉRICA DO SUL}

O Marco das tempestades elétricas descreve os locais e quando estes sistemas ocorrem na América do Sul. Para tanto, determina-se a sazonalidade, o ciclo diurno, o ciclo anual e a densidade geográfica de raios e das tempestades elétricas.

\section{CICLO DIURNO}

Utilizando a base de dados de tempestades elétricas construída nesta pesquisa, determinou-se a frequência de ocorrências dos sistemas no decorrer das horas do dia, figura \ref{ciclodiurnototal}. Deste modo, obtivemos o ciclo diurno das tempestades elétricas por meio da distribuição de probabilidade de ocorrências.

\begin{figure}[!hb]
  \centering
  {{\includegraphics[height=7.5cm]{img/ciclos/ciclodiurno19982011total}}}
\caption{Ciclo diurno das tempestades elétricas observadas em hora local. Os valores de probabilidade foram normalizados pelo total dos 157,592 sistemas identificados.}
\label{ciclodiurnototal}
\end{figure} 

Observa-se que 40\%  das tempestades elétricas observadas pelo TRMM ocorrem entre 13h e 17h, indicando que o aquecimento da superfície do continente e o aumento da camada limite planetária no decorrer do dia são ingredientes que podem aumentar a probabilidade de ocorrência em  relação aos horários de menor fluxo de calor sensível para a atmosfera. Por exemplo, às 9h a probabilidade de tempestade elétrica é de 1.6\% e às 14h é de 8.8\%, portanto às 14h a probabilidade de ocorrência de tempestade tempestade elétrica é 5.4 vezes maior do que às 9h.

% Enquanto o TRMM observou 2312 tempestades elétricas às 9h (hora local), às 14h foram observadas 13,877.

O ciclo diurno também foi estudado para cada região de 10 por 10 graus, como mostra a figura \ref{diurno}. 

\begin{figure}[!hb]
\centering
{\includegraphics[height=13.5cm]{img/ciclos/ciclodiurno10x1019982011localtime}}  
\caption{Ciclo diurno em hora local para as tempestades elétricas observadas em cada região de 10 por 10 graus. Os valores de probabilidade são mostrados em porcentagem e foram normalizados pelo total de 157,592 sistemas observados.}
\label{diurno}
\end{figure}

Pode-se observar que existe um predomínio de ocorrências de tempestades elétricas entre 13h e 17h sobre o continente. Sobre o oceano observa-se  uma distribuição bimodal, com pico no começo da noite e durante a madruga.

%, que correspondem ao ciclo diurno das \textit{precipitation features} em \citeonline{nesbitt2003diurnal}. 

Sobre os oceanos, os processos de formação de nuvens e consequentemente de formação de tempestades elétricas se mostram mais ativo no horário em que a temperatura superficial e a probabilidade de ocorrência de sistemas sobre o continente  diminui. Neste horário a superfície do oceano pode estar com temperaturas maiores do que as temperaturas sobre a superfície do continente, aumentando a convergência sobre o oceano. A atividade convectiva intensa entre 13-17h sobre o continente também aumenta a cobertura de nuvens do tipo cirrus sobre o oceano inibindo a formação de nuvens \cite{nesbitt2003diurnal}.

% associado ao efeito radiativo de espalhamento e aquecimento de camadas mais elevadas da superfície    

%Mesmo que em uma análise geral mostre a importância do aquecimento superficial do continente para a ocorrência de tempestades elétricas, sistemas noturnos sobre a Colômbia e Venezuela são bastante frequentes. 

Entre 0$^{\circ}$--10$^{\circ}$ Norte e 80$^{\circ}$--70$^{\circ}$ Oeste e às 0h, observou-se a maior probabilidade ($\simeq$0.4\%) de tempestades elétricas noturnas da América do Sul, o que representou um número de 630 sistemas observados em 14 anos, apenas entre 0h e 00:59h. A circulação de vale e montanha associada com a topografia elevada na Colômbia, principalmente a região do Parque Nacional Natural Paramillo, e o Lago Maracaibo na Venezuela, e a atuação da Zona de Convergência Intertropical (ZCIT), promovem condições para o desenvolvimento de tempestades elétricas noturnas de maneira mais eficiente do que as demais regiões \cite{burgesser2012}.\sigla{name={ZCIT},description={Zona de Convergência Intertropical}}

Entre 0$^{\circ}$--10$^{\circ}$ Norte e 90$^{\circ}$--80$^{\circ}$ Oeste, abrangendo o Panamá e parte Sul da Costa Rica, e a região do Oceano Pacífico que engloba o Parque Nacional da Ilha do Coco na Costa Rica e parte das ilhas Galápagos no Equador,  a  região oceânica e costeira com a maior probabilidade de ocorrência de tempestades elétricas. Observe os valores de densidade de tempestades elétricas neste quadrante geográfico, na próxima seção, em \ref{secaoDensidades} na figura \ref{densidadeTempestade}. O ciclo diurno das tempestades elétricas nesta região revela uma distribuição bimodal, com um pico às 4h e outro às 14h. O pico das 14h, provavelmente está associado as tempestades elétricas da região do Panamá, Costa Rica e suas respectivas regiões costeiras adjacentes, as quais sofrem maior aquecimento superficial durante o dia, enquanto o maior pico que  ocorreu às 4h provavelmente corresponde com as trocas de energia na forma de calor entre o oceano e atmosfera.

Entre 30$^{\circ}$--10$^{\circ}$ Sul e 90$^{\circ}$--80$^{\circ}$ Oeste e entre 30$^{\circ}$--20$^{\circ}$ Sul e 80$^{\circ}$--70$^{\circ}$ Oeste, região do Pacífico, as tempestades elétricas são mais raras do que as demais regiões devido a atuação permanente da subsidência da Célula de Hadley que modula a Alta Subtropical do Pacífico Sul, responsável também por regiões como o Deserto do Atacama e parte do semi-árido Argentino \cite{reboita2010regimes}.

Na região do Atlântico Subtropical, a probabilidade de tempestades elétricas é maior do que no Atlântico Norte. A passagem de sistemas transientes entre 40$^{\circ}$--30$^{\circ}$ Sul e 50$^{\circ}$--30$^{\circ}$ Oeste e 30$^{\circ}$--20$^{\circ}$ Sul e 40$^{\circ}$--30$^{\circ}$ Oeste, gera maior número de tempestades elétricas oceânicas do que com a atuação da ITCZ no Atlântico Tropical. Observa-se também que nas regiões oceânicas o ciclo diurno das tempestades elétricas indica maior atividade noturna.


A maior atividade horária de tempestades elétricas, ocorreu entre 10$^{\circ}$--0$^{\circ}$ Sul e 70$^{\circ}$--50$^{\circ}$ Oeste e 20$^{\circ}$--10$^{\circ}$ Sul e 60$^{\circ}$--50$^{\circ}$ Oeste. Em cada uma destas três regiões observou-se a probabilidade de aproximadamente 0.8\% entre as 14h e 16h, mostrando que em toda esta área o TRMM observou 3 tempestades elétricas a cada 2 dias, apenas durante estas duas horas.


Entre 30$^{\circ}$--20$^{\circ}$ Sul e 60$^{\circ}$--50$^{\circ}$ Oeste,  região de grande atividade de Sistemas Convectivos de Meso-escala (MCS) conforme descrevem \citeonline{Durkee2009}, encontra-se um máximo durante a tarde e os sistemas noturnos tiveram probabilidade de ocorrência 2.7 vezes menor do que os valores encontrados sobre os vales na Colômbia e Venezuela, mostrando que a ocorrência dos MCS ao Sul da América do Sul com ciclo de vida maior do que 9h ou com formação noturna, não possuem probabilidade de ocorrência que destaca-se em relação as demais regiões continentais, mesmo neste banco de dados composto apenas por tempestades elétricas. 
\sigla{name={MCS},description={Sistemas Convectivos de Meso-escala}}

\section{CICLO ANUAL}

Quando se analisa a sazonalidade, observa-se que a estação de tempestades elétricas na América do Sul se configura entre outubro e março e possui dois picos: janeiro, durante o verão austral; e outubro, período de transição entre a estação seca e chuvosa. 


\begin{figure}[!h]
\centering
{\includegraphics[height=7.5cm]{img/ciclos/cicloanual19982011total}}
\caption{Ciclo anual das tempestades elétricas observadas em hora local. Os valores de probabilidade foram normalizados pelo total dos 157,592 sistemas identificados.}
\label{cicloanualtotal}
\end{figure} 

A maior probabilidade foi de 10.9\%  no mês de outubro, conforme mostra a figura \ref{cicloanualtotal}. De outubro até março foram observadas 60.2\% das tempestades elétricas. Em junho observou-se a mínima probabilidade de tempestades elétricas com valor de 5.1\%. Portanto, entre o período de máximo e mínimo anual o número de ocorrência de tempestades elétricas reduz aproximadamente pela metade.


%O ciclo anual representado na figura \ref{cicloanualtotal}, também foi obtido por meio da distribuição de probabilidade de ocorrências, porém, no decorrer dos meses do ano.

O ciclo anual das tempestades elétricas também foi estudado para cada região de 10 por 10 graus de latitude e longitude. O valor de 70 por cento da máxima probabilidade de tempestade elétrica em cada região foi definido como limiar para considerar que a ocorrência de tempestades elétricas aumentou o suficiente para definir uma estação e este valor é representado pela linha horizontal que corta cada gráfico em cada ponto da grade de 10$^{\circ}$ $\times$ 10$^{\circ}$ da figura \ref{anual}. 
%Desta forma, definimos quais os meses que fazem parte da estação das tempestades elétricas em cada região de 10 por 10 graus.
%Foram selecionados os meses em que a probabilidade de ocorrência de tempestade elétrica foi superior à 70 por cento da máxima probabilidade observada em cada região.

\begin{figure}[!h]
\centering{\includegraphics[height=13.5cm]{img/ciclos/cicloanual10x1019982011localtime}}  
\caption{Ciclo anual em hora local para as tempestades elétricas observadas em cada região de 10 por 10 graus de latitude e longitude. Os valores de probabilidade são mostrados em porcentagem e foram normalizados pelo total de 157,592 sistemas observados. As linhas horizontais cortam o valor de 0.7 do máximo de probabilidade, utilizado como limiar para definir as estações de tempestades elétricas.}
\label{anual}
\end{figure}

A tabela \ref{caracEstacao} mostra os meses de duração das estações de tempestades elétricas de acordo com cada região conforme mostra a figura \ref{anual}. Considerando o ciclo anual em cada ponto da grade de 10$^{\circ}$ $\times$ 10$^{\circ}$, observa-se em média uma estação de tempestades elétricas com duração de 4.5 meses. 


\begin{table}[!ht]
\caption{Principais características do ciclo anual de probabilidade de ocorrência de tempestades elétricas observadas entre 1998-2011, em cada região de 10 por 10 graus de latitude longitude.}
\label{caracEstacao}
\centering
\small
\newcommand{\grayline}{\rowcolor[gray]{.88}}
\renewcommand {\tabularxcolumn }[1]{ >{\arraybackslash }m{#1}}
\newcolumntype{W}{>{\centering\arraybackslash}X}
\begin{tabularx}{\textwidth}{p{0.6cm} p{3.5cm} W W W W} %{|p{10cm}|X|X|X|X|X|X|X|X| }
\hline\hline 
\grayline  & Localização & Número de sistemas & Estação (meses) & Duração (meses) & Máximo\\[1.5pt]
\hline
1&0$^{\circ}$--10$^{\circ}$N, 90$^{\circ}$--80$^{\circ}$O& 4173  & Abr--Set &6& Abr\\[1.5pt]\grayline
2&0$^{\circ}$--10$^{\circ}$N, 80$^{\circ}$--70$^{\circ}$O& 14,232 & Mar--Nov &9& Out\\[1.5pt]
3&0$^{\circ}$--10$^{\circ}$N, 70$^{\circ}$--60$^{\circ}$O& 11,946 & Jul--Nov &5& Ago--Set\\[1.5pt]\grayline
4&0$^{\circ}$--10$^{\circ}$N, 60$^{\circ}$--50$^{\circ}$O&  4895 & Jul--Set &3& Ago\\[1.5pt]
5&0$^{\circ}$--10$^{\circ}$N, 50$^{\circ}$--40$^{\circ}$O& 645 & Mai--Jun, Set--Nov &5& Set\\[1.5pt] \grayline
6&0$^{\circ}$--10$^{\circ}$N, 40$^{\circ}$--30$^{\circ}$O& 824 & Out--Dez &3& Dez\\[1.5pt]

7&10$^{\circ}$--0$^{\circ}$S, 90$^{\circ}$--80$^{\circ}$O& 225 & Mar &1& Mar\\[1.5pt]\grayline
8&10$^{\circ}$--0$^{\circ}$S, 80$^{\circ}$--70$^{\circ}$O& 10,014 & Set--Dez &4& Out\\[1.5pt]
9&10$^{\circ}$--0$^{\circ}$S, 70$^{\circ}$--60$^{\circ}$O& 12,605 & Ago--Nov &4& Out\\[1.5pt]\grayline
10&10$^{\circ}$--0$^{\circ}$S, 60$^{\circ}$--50$^{\circ}$O& 12,590 & Set--Dez &4& Out\\[1.5pt]
11&10$^{\circ}$--0$^{\circ}$S, 50$^{\circ}$--40$^{\circ}$O& 7863 & Jan--Abr, Nov--Dez &6&  Jan\\[1.5pt]\grayline
12&10$^{\circ}$--0$^{\circ}$S, 40$^{\circ}$--30$^{\circ}$O& 1363 & Fev--Abr &3&  Mar\\[1.5pt]

13&20$^{\circ}$--10$^{\circ}$S, 90$^{\circ}$--80$^{\circ}$O& 1  &   --0--  & --0-- & --0-- \\[1.5pt]\grayline
14&20$^{\circ}$--10$^{\circ}$S, 80$^{\circ}$--70$^{\circ}$O& 4344 & Jan--Fev,  Set--Dez  &6& Out\\[1.5pt]
15&20$^{\circ}$--10$^{\circ}$S, 70$^{\circ}$--60$^{\circ}$O& 8895 & Jan--Mar, Out--Dez &6& Out\\[1.5pt]\grayline
16&20$^{\circ}$--10$^{\circ}$S, 60$^{\circ}$--50$^{\circ}$O& 10,973 & Jan--Mar,  Out--Dez &6& Out\\[1.5pt]
17&20$^{\circ}$--10$^{\circ}$S, 50$^{\circ}$--40$^{\circ}$O& 8524 & Jan--Mar, Out--Dez &6&  Jan\\[1.5pt]\grayline
18&20$^{\circ}$--10$^{\circ}$S, 40$^{\circ}$--30$^{\circ}$O& 625  & Jan--Mar &3&  Mar\\[1.5pt]

19&30$^{\circ}$--20$^{\circ}$S, 90$^{\circ}$--80$^{\circ}$O& 32 & --0-- & --0-- &  --0-- \\[1.5pt]\grayline
20&30$^{\circ}$--20$^{\circ}$S, 80$^{\circ}$--70$^{\circ}$O& 32 & --0-- &--0--&  --0--\\[1.5pt]
21&30$^{\circ}$--20$^{\circ}$S, 70$^{\circ}$--60$^{\circ}$O& 5607 & Dez--Mar &4& Jan\\[1.5pt]\grayline
22&30$^{\circ}$--20$^{\circ}$S, 60$^{\circ}$--50$^{\circ}$O& 8885  & Dez--Mar &4& Jan\\[1.5pt]
23&30$^{\circ}$--20$^{\circ}$S, 50$^{\circ}$--40$^{\circ}$O& 6121 & Dez--Mar &4& Jan\\[1.5pt]\grayline
24&30$^{\circ}$--20$^{\circ}$S, 40$^{\circ}$--30$^{\circ}$O& 1884 & Dez--Mai&4&  Mar\\[1.5pt]

25&40$^{\circ}$--30$^{\circ}$S, 90$^{\circ}$--80$^{\circ}$O& 258 & Jun &1&  Jun \\[1.5pt]\grayline
26&40$^{\circ}$--30$^{\circ}$S, 80$^{\circ}$--70$^{\circ}$O& 366 & Jan--Mar, Mai--Jun &5& Jan \\[1.5pt]
27&40$^{\circ}$--30$^{\circ}$S, 70$^{\circ}$--60$^{\circ}$O& 7652 & Dez--Jan &2&  Jan\\[1.5pt]\grayline
28&40$^{\circ}$--30$^{\circ}$S, 60$^{\circ}$--50$^{\circ}$O& 5440 & Dez--Mar  &4&  Jan\\[1.5pt]
29&40$^{\circ}$--30$^{\circ}$S, 50$^{\circ}$--40$^{\circ}$O& 2949 & Jan--Set &9&  Abr\\[1.5pt]\grayline
30&40$^{\circ}$--30$^{\circ}$S, 40$^{\circ}$--30$^{\circ}$O& 2301 & Abr--Jun  &3& Mai\\[1.5pt]


\hline 
\end{tabularx}
\end{table}

 
%No ciclo anual mostrado na figura \ref{anual}, observa-se uma clara diferença sazonal na ocorrência de tempestades elétricas entre os dois Hemisférios. Sobre o Hemisfério Norte as tempestades ocorrem principalmente entre os meses de abril e agosto, enquanto no Hemisfério Sul entre setembro e março, apesar das características individuais de cada região como por exemplo dois ou três picos de atividade durante a estação.
%O deslocamento da ZCIT durante o verão setentrional, inverte a estação de tempestades elétricas entre 10$^{\circ}$ Sul e 10$^{\circ}$ Norte, diminuindo o número de sistemas no inverno austral.

Na região referente as linhas 21 e 27 da tabela \ref{caracEstacao} (40$^{\circ}$--20$^{\circ}$ Sul e 70$^{\circ}$--60$^{\circ}$ Oeste),  entre o clima semi-árido na Argentina e parte da Bacia do Prata,  local das tempestades mais severas e de maior probabilidade de ocorrência de de núcleos de convecção profunda da AS como apontam \citeonline{cecil2005, Romatschke2010}, foi encontrada a estação de tempestades elétricas com a maior amplitude entre o máximo de ocorrências em janeiro e o mínimo em junho. A probabilidade de tempestades elétricas em junho foi aproximadamente 10 vezes menor do que em janeiro.




Sobre a Colômbia e parte Oeste da Venezuela que abrange o lago Maracaibo, região referente a linha 2 da tabela \ref{caracEstacao}, foi a região em que o TRMM observou o maior número de tempestades elétricas (14,232). Nesta região a estação de tempestades elétricas dura  9 meses, com dois picos: abril e outubro.


As máximas probabilidades de ocorrência de tempestades elétricas durante o ciclo anual, não ocorrem em fase com os máximos anuais de precipitação em algumas regiões da AS, portanto, a definição de uma estação de tempestades elétricas não implica na definição de uma estação chuvosa. 

Durante o final do verão setentrional, entre julho e setembro, a estação chuvosa começa a se deslocar do Hemisfério Norte para o Hemisfério Sul, de Noroeste da AS para Sudeste da AS intensificando-se progressivamente até atingir os maiores acumulados de chuva entre dezembro e janeiro \cite{grimm2003nino,reboita2010regimes,Marengo2012,shi-atlas,bombardi2008variabilidade,cusdodioTese}.
 
Na parte central da AS, linhas  8, 9, 10, 14, 15, 16 da tabela \ref{caracEstacao} (20$^{\circ}$--0$^{\circ}$ Sul e 80$^{\circ}$--50$^{\circ}$ Oeste), as tempestades elétricas ocorrem com maior frequência em outubro, entre a estação seca e chuvosa. Porém os máximos sazonais de precipitação nesta região ocorrem defasados aproximadamente em 5 meses do máximo de tempestades elétricas, entre os meses de fevereiro e abril \cite{grimm2003nino,reboita2010regimes,shi-atlas,bombardi2008variabilidade,cusdodioTese}.

%Assim como as estações chuvosas nas diferentes localidades da América do Sul, as estações de tempestades elétricas se configuram conforme o Sistema de Monção da América do Sul (SMAS) \cite{Zhou1998}.  
%O aumento da temperatura da superfície na primavera austral e o baixo acumulado de chuva e baixa nebulosidade durante o período pré-monção promovem      
%Em \citeonline{Petersen2001}, o estudo realizado referente a estrutura tridimensional da precipitação observada pelo TRMM sobre a região Central da Amazônia, mostrou que a convecção mais profunda ocorre também na transição do período seco para o chuvoso, exatamente quando começa a reversão sazonal do vento em baixos níveis associado ao SMAS conforme apontam \citeonline{Zhou1998}.

Na região Sul da AS, à Leste da Cordilheira dos Andes referentes as linhas  21 e 27 da tabela \ref{caracEstacao}, a estação de tempestades elétricas ocorre entre dezembro e janeiro, em fase com a estação chuvosa. Aqui, a probabilidade de tempestades elétricas aumenta quando o Jato de Baixos Níveis da América do Sul (JBNAS) intensifica o transporte de umidade entre a Bacia Amazônica e a Bacia do Prata durante a atuação do SMAS \cite{marengo2004}.   

A região Sul e Sudeste da América do Sul, referentes as linhas 17, 22, 23 e 28 da tabela \ref{caracEstacao}, a estação de tempestades elétricas ocorre também em fase com o período de máxima configuração do SMAS, entre dezembro e janeiro. \citeonline{petersen2002trmm}, mostram que mudanças na circulação atmosférica durante o período chuvoso associadas as fases ativa e de pausa da SMAS influenciam na densidade de raios sobre a AS, indicando que, nas regiões em que a estação de tempestades elétricas coincide com o período de atuação da ZCAS, a variabilidade intra-sazonal do SMAS  governa sobre a sazonalidade das tempestades elétricas \cite{CarvalhoJones2002,carvalho2004south}.

Apesar da convergência em grande escala associada a ZCAS, sua atuação contínua durante meses poderia diminuir a instabilidade atmosférica devido a chuvas contínuas e baixa incidência de radiação de onda curta na superfície, o que abaixaria a temperatura da superfície dando características oceânicas para toda a extensão da ZCAS. As pausas do SMAS podem ser importantes para o aumento do \textit{lapse rate} da atmosfera e quando há novamente uma fase ativa do SMAS, a atmosfera possui energia e umidade suficiente para eventos extremos de precipitação e de tempestades elétricas.


%Conforme descrito em \citeonline{albrecht2008eletrificaccao},  
%Durante o regime de Oeste, fase de pausa da SMAS, \citeonline{petersen2002trmm} observaram aumento entre 500-700\% da densidade de raios em relação ao regime de Leste, principalmente: no Sul e costa Sul do Brasil; ao Norte do Uruguai; Norte da Argentina, Sul da Bolívia e Oeste do Paraguai; e porção Noroeste da Argentina fronteiriça com o Uruguai. Durante o regime de Leste, \citeonline{petersen2002trmm} observaram aumento entre 50-200\% densidade de raios na região Central da Bacia do Prata, entre o Sudoeste da Argentina e Norte do Paraguai e região Sul da Argentina\cite{jones2002active,albrecht2008eletrificaccao}.

%Durante a fase ativa, apesar da atuação da Zona de Convergência do Atlântico Sul (ZCAS), o aquecimento da superfície diminui devido o aumenta da nebulosidade e chuvas contínuas, não proporcionando condições termodinâmicas propícias para convecção explosiva e sim grandes extensões de nebulosidade com chuvas estratiformes \cite{jones2002active,albrecht2008eletrificaccao,albrecht2011}.

\sigla{name={SMAS},description={Sistema de Monção da América do Sul}}
\sigla{name={AS},description={América do Sul}}

Na região da costa Nordeste da AS, referente as linhas 12 e 18 da tabela \ref{caracEstacao}, a estação de tempestades elétricas possui máxima atividade em março e mínimo em torno de agosto. Nestas regiões o máximo sazonal de precipitação ocorre em fase com o máximo sazonal de tempestades elétricas, porém, durante o período pós-monção na AS \cite{grimm2003nino,reboita2010regimes,shi-atlas,bombardi2008variabilidade,cusdodioTese}.

As regiões oceânicas, referentes as linhas, 1, 5, 7, 25, 29 e 30 da tabela \ref{caracEstacao}, o pico de atividade de tempestades elétricas não ocorre entre outubro e março como mostra o ciclo anual total da figura \ref{cicloanualtotal}, mas entre março e setembro durante o outono e inverno austral. 

%Sendo a capacidade térmica dos oceanos maior do que nos continentes, sua temperatura superficial sofre menores oscilações sazonais do que o nos continentes. 


 Durante o inverno, a diferença entre a temperatura da superfície do continente e a temperatura da superfície do oceano é menor, o que provavelmente favorece a convergência  sobre o oceano, intensificando os processos de eletrificação fora do período de maior insolação. Conforme descrito por \citeonline{cusdodioTese}, o máximo sazonal de precipitação observada pelo PR nestas regiões oceânicas ocorrem em fase com os máximos sazonais de tempestades elétricas desta pesquisa. 

A região do Atlântico Tropical refente a linha 6 da tabela \ref{caracEstacao}, possui o máximo sazonal de ocorrência de tempestades elétricas em dezembro, enquanto o máximo sazonal de precipitação ocorre durante o outono austral \cite{cusdodioTese}.


%Durante abril e maio, o SMAS vai se desconfigurando e o máximo de chuva começa a retornar para o Hemisfério Norte caminhando de Sudeste para o Nordeste do Brasil e subindo pelo lado Leste da Bacia Amazônica. Neste retorno é que ocorrem os máximos de precipitação em toda a região da Bacia Amazônica, porém o máximo de ocorrência de tempestades elétrica ocorreu na vinda da estação chuvosa para o Hemisfério Sul.



\section{DENSIDADES GEOGRÁFICAS}
\label{secaoDensidades}

Considerando o método descrito em \ref{metodoPass}, referente ao cálculo da densidade de tempestades elétricas e da densidade de raios, foram obtidos os mapas da densidade total de tempestades elétricas e de raios, figuras \ref{densidadeTempestade} e \ref{densidadeRaios}, assim como os mapas da densidade sazonal de raios e de tempestades elétricas que são mostrados nas  figuras \ref{densidadeRaios} e \ref{DensidadeTempestadesSazonal}. Note que as densidade de raios  tanto total como sazonal que são apresentadas nesta seção, não corresponde a amostragem total de \textit{flash} observados pelo LIS corrigida pela eficiência como é mostrado em trabalhos como  \citeonline{albrecht2009tropical,cecil2014gridded}, mas correspondem ao subconjunto dos \textit{flashes} do LIS, os quais passaram pelos critérios de identificação de tempestades elétricas descritos em \ref{identificaTempestades}.

%Nesta seção será possível avaliar se as regiões aonde ocorrem o maior/menor número de sistemas correspondem com as regiões com maior/menor número de raios.
%Na figura \ref{densidadeTempestade}, observa-se que a convergência de umidade e liberação calor latente (evapotranspiração) e sensível na região central da Bacia Amazônica promove a máxima ocorrência de tempestades elétricas, principalmente sobre: a Colômbia, Venezuela, Panamá  e região central da Bacia Amazônica abrangendo a parte brasileira, colombiana, venezuelana e peruana.

Na figura \ref{densidadeTempestade}, observa-se que as grandes regiões de máxima densidade de tempestades elétricas, ou seja com valores acima de 2.5 $\times$ 10$^{-4}$ km$^{-2}$, situam-se na parte Norte e Nordeste da AS. Observa-se dois máximos: um na região da Colômbia associado ao extremo Norte da Cordilheira dos Andes, outro ao Norte/Noroeste da Floresta Amazônica, abrangendo a parte brasileira, colombiana, venezuelana e peruana.  

A costa Oeste da Colômbia e Panamá, destaca-se como a região de maior densidade de tempestades elétricas costeiras, pois o escoamento atmosférico predominantemente de Leste devido a convergência dos Alísios ao encontrar o extremo Norte da Cordilheira dos Andes, entre 0--10$^{\circ}$ Norte, sofre pertubações que disparam tempestades elétricas que continuam a se propagar em sentido Oeste para o Pacífico tropical.  

\begin{figure}[!ht]
 \centering
 {\includegraphics[height=13.5cm]{img/DensidadeTempestades/densEspacial19982011TotalTempestadesPolyfill}}
\caption{Densidade espacial total de tempestades elétricas. Os valores correspondem ao número de sistemas por ano por quilômetro quadrado em cada pronto da grade de 0.25 graus.}
 \label{densidadeTempestade}
\end{figure}
  
\begin{figure}[!ht]
 \centering
  {\includegraphics[height=13.5cm]{img/TaxaFlash/densEspacial_19982011totalTaxaFlashPolyfill}}
  \caption{Densidade espacial total de raios. Os valores correspondem ao número de raios por ano por quilômetro quadrado em cada pronto da grade de 0.25 graus.}
  \label{densidadeRaios}
%\caption{Densidade espacial de tempestades elétricas e raios observados entre 1998 e 2011.}
%\label{tempesRaios}
\end{figure}

%A atuação da ITCZ em contato com a Cordilheira dos Andes no extremo Norte da AS, entre 0--10$^{\circ}$ Norte, e a convergência de umidade e liberação calor latente (evapotranspiração) e sensível na região central da Bacia Amazônica,   são os principais propulsores de tempestades elétricas da AS.  %que associa-se intimamente com o SMAS,

A maior extensão em área das máximas densidade de tempestades elétricas sobre a AS foi observada na região da Floresta Amazônica, principalmente a Oeste e Sudoeste do Pico da Neblina região da cabeceira do Rio Negro, de tríplice fronteira entre Brasil, Colômbia e Venezuela. É notável que a topografia da região Amazônica aumenta a densidade de tempestades elétricas como é o caso da região do Pico da Neblina, principalmente entre a Venezuela e o Brasil, porém esta vasta região contínua com valores de densidade de tempestades elétricas superior a  2.5 $\times$ 10$^{-4}$ km$^{-2}$, sugere que os efeitos da circulação atmosférica de grande escala e a termodinâmica associada a instabilidade atmosférica no ambiente amazônico são os principais agentes que promovem o maior número de tempestades elétricas da AS. 


No entanto, ao analisar a densidade de raios na figura \ref{densidadeRaios}, observa-se que na região central da Bacia Amazônica há pontos da grade de 0.25$^{\circ}$ $\times$ 0.25$^{\circ}$ com valores superior a 30 raios por ano por quilômetro quadrado (ano$^{-1}$ km$^{-2}$) e densidades de sistemas (figura \ref{densidadeTempestade}) acima de 2.9 $\times$ 10$^{-4}$ tempestades elétricas por quilômetro quadrado (sistemas km$^{-2}$). Na região da Argentina e Paraguai, valores de densidade de raios de mesma magnitude são observados com uma densidade de sistemas em torno de 1.8 $\times$ 10$^{-4}$ km$^{-2}$. No geral considerando os dados de 14 anos de observações do TRMM, podemos afirmar em alguns pontos de grade de 0.25$^{\circ}$ $\times$ 0.25$^{\circ}$ sobre a Bacia do Prata, as tempestades elétricas produzem 62\% mais raios por quilometro quadrado por ano observado pelo LIS do que as tempestades elétricas sobre a Bacia Amazônica. 

Como mostra a figura \ref{densidadeRaios}, a maior extensão contínua em área com densidade anual de raios acima de 25 ano$^{-1}$ km$^{-2}$ é a região da Bacia do Prata. Porém a maior extensão contínua das maiores densidades de  tempestades elétricas, com valores superiores a 2.9 $\times$ 10$^{-4}$ km$^{-2}$, foi observada na região Norte e Nordeste da AS. O que sugere que os processos de eletrificação são mais eficientes nas regiões ao Sul da AS.


O fato da região da Bacia do Prata possuir 



Nas figuras \ref{TaxaFlash} e \ref{DensidadeTempestadesSazonal}, a densidade espacial de raios e de tempestades elétricas, foi calculada para os períodos associados a cada estação do ano: dezembro, janeiro e fevereiro (DJF), março, abril e maio (MAM), junho, julho e agosto (JJA) e setembro, outubro e novembro (SON). A tabela \ref{EstacaoQtd} mostra o acumulado de sistemas observados em cada estação do ano.
\sigla{name={DJF},description={Dezembro, janeiro e fevereiro}} 
\sigla{name={MAM},description={Março, abril e maio}}
\sigla{name={JJA},description={Junho, julho e agosto}}
\sigla{name={SON},description={Setembro, outubro e novembro}}


\begin{figure}[!ht]
  \centering{
  \subfloat[DJF]{{\includegraphics[height=6.5cm, trim=0 7 0 0, clip]{img/TaxaFlash/densEspacial_19982011djfTaxaFlashPolyfill}} \label{txDJF}}
  \subfloat[MAM]{{\includegraphics[height=6.5cm, trim=0 7 0 0, clip]{img/TaxaFlash/densEspacial_19982011mamTaxaFlashPolyfill}} \label{txMAM}}

  \subfloat[JJA]{{\includegraphics[height=6.5cm, trim=0 7 0 0, clip]{img/TaxaFlash/densEspacial_19982011jjaTaxaFlashPolyfill}} \label{txJJA}}
  \subfloat[SON]{{\includegraphics[height=6.5cm, trim=0 7 0 0, clip]{img/TaxaFlash/densEspacial_19982011sonTaxaFlashPolyfill}} \label{txSON}}
  }    
  \caption{Densidade espacial sazonal de raios.} 
\label{TaxaFlash}
\end{figure} 


Na primavera austral (SON), início do SMAS, a intensificação dos alísios vindos do Atlântico Norte, e o aumento gradativo da evapotranspiração na Floresta Amazônica vão intensificando o transporte de umidade da bacia do Amazônias para a bacia do Prata \cite{marengo2004}.  Esse processo de início da configuração do SMAS provoca a estação com a maior taxa de raios do continente Sul Americano, e esta, ocorre em regiões no centro no continente principalmente a Leste da Cordilheira dos Andes: na Amazônia Central, Argentina, Paraguai e Sul do Brasil.


Neste período destaca-se a taxa de raios sobre o Lago Maracaibo durante SON, na Venezuela, que no acumulado dos 14 anos atingiu o valor de 30 raios por mês de observação por quilômetro quadrado em cada ponto da gade de 0.25 graus. Em \citeonline{albrecht2009tropical}, a região do Lago Maracaibo foi apontada como o máximo global das observações do TRMM. 


\begin{table}[!h]
\caption{Total de tempestades elétricas observadas entre 1998-2011, para cada período de três meses associados as estações do ano.}
\label{EstacaoQtd}
\centering
\small
\newcommand{\grayline}{\rowcolor[gray]{.88}}
\renewcommand {\tabularxcolumn }[1]{ >{\arraybackslash }m{#1}}
\newcolumntype{W}{>{\centering\arraybackslash}X}
\begin{tabularx}{\textwidth}{W W} %{|p{10cm}|X|X|X|X|X|X|X|X| }
\hline  \hline 
Estação & Número de sistemas \\[1.5pt]  
 \hline
\grayline Verão -- DJF & 46,077 \\[1.5pt]
Outono -- MAM & 36,804\\[1.5pt]
\grayline Inverno -- JJA  & 16,850\\[1.5pt] 
Primavera -- SON & 57,861\\[1.5pt]
\hline 
\end{tabularx}
\end{table}

%Durante DJF a circulação do JBN trazendo umidade da Amazônia é predominante. É quando é ativado os processos de eletrificações de nuvens entre 40$^{\circ}$--20$^{\circ}$ Sul e 70$^{\circ}$--60$^{\circ}$ Norte.

Durante DJF, os máximos de raios são observados em Mato Grosso do Sul; Sul de Mato Grosso; Sudeste Brasileiro, entre costa de Santa Catarina e o Vale do Ribeira em São Paulo, região de fronteira entre São Paulo, Minas Gerais e Rio de Janeiro, aonde localiza-se o Parque Nacional Itatiaia e o Pico das Agulhas Negras; interior de São Paulo; Goias; e na Bacia do Rio Tocantis. Apesar de observarmos o maior número de raios durante a estação de transição entre seca e chuvosa, essas regiões Centrais e Sudeste da AS possuem os processos de eletrificação regulados durante a estação chuvosa. 

Em \citeonline{petersen2002trmm}, é mostrado que mesmo que se tenha observado diminuição na taxa de raios e redução da intensidade convectiva durante o regime de vento de Oeste no experimento TRMM \textit{Large-scale Biosphere Atmosphere Experiment in Amazonia} (LBA), em outras regiões da AS durante o período chuvoso, há um aumento da taxa de raios.

Considerando que o regime de ventos de Leste e Oeste identificado no LBA está associado com as fases ativas e inativas do SMAS conforme descrevem \citeonline{carvalho2002intraseasonal}, pode-se considerar que as máximas taxas de raios apresentadas na figura \ref{txDJF} são moduladas pelas variações na circulação sinóptica associadas com o processos de formação e dissipação da SACZ \cite{petersen2002trmm,albrecht2011,silva2002lba}.

Durante MAM, quando o máximo de chuvas começa a retornar para o Hemisfério Norte, observamos as tempestades elétricas bastante concentradas na região Norte e Nordeste da AS, como mostra a figura \ref{tempestadesMAM}. Neste período, principalmente nas regiões das cidade de Belém, estado do Maranhão, Piauí, Rio Grande do Norte e Paraíba,  ocorrem: os máximos de chuva, os máximos de densidade de raios e os máximos de densidade de tempestades elétricas. Esse sincronismo não é comum.

Ao comparar as figuras \ref{TaxaFlash} e \ref{DensidadeTempestadesSazonal} observa-se que as regiões de máxima densidade espacial de raios não são as regiões de máxima densidade de tempestades elétricas. Os máximos de raios ficam situados em regiões de transição, deslocados dos máximos de sistemas, reforçando a hipótese de \citeonline{williams2002}, em que se espera maior atividade elétrica de nuvem em um ambiente de transição entre seco e úmido.

Por exemplo, a maior área continua da América do Sul com taxas anuais de raios superiores a 20 raios por ano por quilômetro quadrado, como mostra a figura \ref{densidadeRaios}, ocorre na região Sul da AS. Tanto na figura \ref{densidadeTempestade} quanto na figura \ref{DensidadeTempestadesSazonal}, podemos observar um forte gradiente de sistemas nesta região, que marca a transição entre o clima Desértico no Deserto do Atacama e Semi-árido na Argentina para o clima Subtropical úmido, promovendo um ambiente de transição seco/úmido permanente para os sistemas que iniciam-se principalmente na região da Serra de Córdoba na Argentina e se propagam para Noroeste.

\begin{figure}[!ht]
  \centering{
  \subfloat[DJF]{{\includegraphics[height=6.5cm, trim=0 7 0 0, clip]{img/DensidadeTempestades/densEspacial19982011djfTempestadesPolyfill}} \label{tempestadesDJF}}
  \subfloat[MAM]{{\includegraphics[height=6.5cm, trim=0 7 0 0, clip]{img/DensidadeTempestades/densEspacial19982011mamTempestadesPolyfill}} \label{tempestadesMAM}}

  \subfloat[JJA]{{\includegraphics[height=6.5cm, trim=0 7 0 0, clip]{img/DensidadeTempestades/densEspacial19982011jjaTempestadesPolyfill}} \label{tempestadesJJA}}
  \subfloat[SON]{{\includegraphics[height=6.5cm, trim=0 7 0 0, clip]{img/DensidadeTempestades/densEspacial19982011sonTempestadesPolyfill}} \label{tempestadesSON}}
  
  }    
  \caption{Densidade espacial sazonal das tempestades elétricas.}
\label{DensidadeTempestadesSazonal}
\end{figure} 

A partir do estudo das densidades de tempestades elétricas e raios, a figura \ref{eficiencia},  representa as regiões em que as tempestades elétricas são mais eficientes na produção de raios. Foi calculada a taxa de raios por tempestade elétrica por ano por quilômetro quadrado. Os maiores valores desta dimensão que associa-se com eficiência espacial que cada região de 0.25 graus tem em produzir raios, representam os locais em que se tem menor número de sistemas em relação ao número de raios durante os 14 anos de dados.

A região da bacia do Prata é a maior extensão contínua com os maiores valores de eficiência espacial de produção de raios. Porém destacam-se regiões menores como no Vale do Ribeira em São Paulo, Pico das Agulhas Negras em Minas Gerais, região serrana do Rio de Janeiro, parte Sul do Tocantis, parte Leste e Norte do Pará e Leste do estado do Amazonas. Estas regiões podem estar associadas com regiões de tempo severo. Locais em que a topografia ou a circulação local intensifica os sistemas.

Na região do  Parque Nacional Natural Paramillo na Colômbia e no Lago Maracaibo na Venezuela, a taxa de raios por em cada área de tempestade de 0.25 graus mostra valores com a mesma magnitude de regiões na Bacia do Prata, mesmo que o número de raios e de sistemas produzidos ao Norte sejam maiores.

Algumas regiões no pico da Cordilheira dos Andes são bastante eficientes, principalmente na região da cidade de Cochabamba na Bolívia.

\begin{figure}[!h]
\centering{\includegraphics[height=13.5cm]{{img/eficiencia/densEspacial_19982011totalEficienciaPolyfill}}}  
\caption{Eficiencia de tempestade}
\label{eficiencia}
\end{figure}
%\chapter{Morfologia da estrutura 3D da precipitação}

\begin{figure}
  \centering{
  \subfloat[CFAD]{{\includegraphics[scale=1.3]{img/precipitacao3d/cfad10_chuvatotal_total_nuvem}} \label{cfadtotal}} \\
  \subfloat[CCFAD]{{\includegraphics[scale=1.3]{img/precipitacao3d/ccfad_10deg_chuvatotal_total_nuvem}} \label{ccfadtotal}}
  }    
  \caption{Diagramas de Contorno de Frequência por Altitude (CFADs) para tempestades elétricas ordenadas pelo 90° percentil dos índices FTA e FT, para a América do Sul em cada 10$^{\circ}$  $\times$ 10$^{\circ}$. Em cada caixa pode-se verificar a porcentagem (\%) de perfis convectivos, estratiformes e outros, respectivamente, (P) o numero de perfis computados, (L) o número de ocorrência de refletividade no nível de máxima ocorrência e (H) o nível de máxima ocorrência.}
\label{cfadstotal}
\end{figure} 


\chapter{TEMPESTADES ELÉTRICAS SEVERAS}

Conforme descrito em \ref{metodoFtaFt}, as taxas de raios das tempestades elétricas neste trabalho de pesquisa estão associadas aos índices FTA e FT. Nesta seção identifica-se qual desses índices podem melhor associar-se com a intensidade convectiva das tempestades elétricas.
%, ou seja, os sistemas com as maiores taxas de raios por minuto (FT) ou os sistemas com as maiores taxas de raios por minuto por quilômetro quadrado (FTA) da sua extensão. 

Para o estudo de intensidade dos sistemas, foram selecionados apenas as tempestades elétricas as quais possuíram $VT_m$ maior ou igual a 1 minuto e com pelo menos um pixel do campo de visão do PR contido na área do sistema, totalizando 94,733 tempestades elétricas do TRMM. %Como a intensidade convectiva dos sistemas é avaliada com base, principalmente, na morfologia da estrutura tridimensional da precipitação e na taxa de raios, 

As equações \ref{eqFT} e \ref{eqFTA} foram aplicadas nas 94,733 tempestades elétricas selecionadas, e então estudada as distribuições de probabilidades dos índices FTA e FT (figuras \ref{pdfFTAFT} e \ref{cdfFTAFT}). Conforme mostra a figura \ref{pdfFTAFT}, trata-se de distribuições exponenciais de probabilidade. Os valores de FTA e FT para cada quantil da amostragem de tempestades elétricas, tanto para as mais frequentes quanto as mais raras, podem ser verificados por meio da distribuição cumulativa de probabilidade de FTA e FT mostradas na figura \ref{cdfFTAFT}. 

\begin{figure}[!ht]
  \centering
  \includegraphics[height=9cm]{img/FtaFt/pdf_FTA_FT}      
  \caption{Densidade de probabilidade de FTA e FT.} 
   \label{pdfFTAFT} 
\end{figure}

\begin{figure}[!hb]
  \centering 
  \includegraphics[height=9cm]{img/FtaFt/cdf_FTA_FT} 
  \caption{Densidade de probabilidade cumulativa de FTA e FT.}
  \label{cdfFTAFT}
\end{figure}
%\label{seriesFtaFt}

Os sistemas potencialmente severos são selecionados pelo 90\textsuperscript{\underline{o}} percentil das amostras de probabilidades de FTA e FT, que estão associado aos maiores e mais raros valores ocorridos. Portanto, pressupõe-se que as tempestades elétricas as quais provavelmente causaram chuva de granizo, rajadas de ventos com queda de árvores e construções ou tornados estão associadas aos sistemas com valores extremos de FTA ou FT.

O grupo das tempestades elétricas com FTA extremo, possuíram valores entre {29.3--1258.7 $\times$ 10$^{-4}$} raios por minuto por quilômetro quadrado (minuto$^{-1}$ km$^{-2}$), enquanto as tempestades elétricas com extremos de FT possuíram valores entre {47.2--1283.6} raios por minuto (minuto$^{-1}$). 



Observe que o valor mínimo de FT foi de 0.6 raios por minuto. Em \citeonline{cecil2005}, considera-se que, a mínima taxa de raios no tempo para as PFs é de 0.7 raios por minutos. Porém a resolução espacial da projeção do tempo de visada do LIS utilizada nesta tese possui resolução de 0.25$^{\circ}$ $\times$ 0.25$^{\circ}$  \cite{albrecht2009tropical,albrecht2011b}. Ao considerar a velocidade e altura da órbita do satélite, o tempo de observação do LIS em um ponto de 0.25$^{\circ}$ $\times$ 0.25$^{\circ}$ na superfície terrestre pode atingir até 102 segundos na região zenital. Então, as tempestades elétricas que possuíram apenas 1 raio e $VT_m$ de $\simeq$100 segundos, tiveram o mínimo valor de FT de 0.6 raios min$^{-1}$, sendo esta, a mínima taxa de raios no tempo detectável em uma tempestades elétricas do TRMM desta pesquisa.

% Na figura \ref{percetilFtaFt}, temos a série de FTA e FT ordenada, e a linha tracejada vertical corta o 90\% percentil dos índices. 

%\begin{figure}[!ht]
%  \centering
%  \includegraphics[height=6cm]{img/FtaFt/90thFtaFt}	 
%  \caption{90\textsuperscript{\underline{o}} percentil de FTA e FT.}
%  \label{percetilFtaFt}
%\end{figure}


\section{ÁREA E TEMPERATURA DO TOPO DA NUVEM}

Observa-se que os extremos de FTA e FT correspondem a sistemas com tamanhos bem distintos. Conforme é mostrado na figura \ref{size}, verifica-se que as máximas probabilidades de ocorrência de tempestades elétricas associadas com os extremos de FTA, ocorrem em sistemas com área 3 ordens de grandeza menor do que nos extremos de FT.

\begin{figure}[!ht]
  \centering
  \includegraphics[height=9cm,trim=0 0 215 0,clip]{img/tb/TbAreas}   
  \caption{Densidade de probabilidade de extensão em área das tempestades elétricas com extremos de FTA e FT.}
  \label{size}  
\end{figure}


%\begin{figure}[!hb]
%  \centering{  
%  \subfloat[Densidade de probabilidade de extensão em área.] { \includegraphics[height=7.5cm,trim=0 0 215 0,clip]{img/tb/TbAreas} \label{size}} \\
%  \subfloat[Densidade de probabilidade de temperatura de brilho em infravermelho.]{ \includegraphics[height=7.5cm,trim=220 0 0 0,clip]{img/tb/TbAreas} \label{tb}} 
%  }
%  \label{t_tb}
%  \caption{Estudo das frequências de ocorrências de tempestades elétricas selecionas pelo 90\textsuperscript{\underline{o}} percentil dos índices de FT e FTA, por extensão em área e por temperatura de brilho de topo das nuvens.}
%\end{figure}

%As tempestades elétricas com valores extremos de FT são maiores em extensão.
Na figura \ref{areaFTAFTA}, pode-se observar que os sistemas com tamanho entre 10$^2$--10$^3$ km$^2$, não ultrapassam 20 raios por minuto (FT). As tempestades elétricas com FT superior a 100 raios por minuto, possuíram tamanho entre 10$^{4}$--10$^{6}$ km$^2$. Note que há um aumento exponencial de FT conforme aumenta a extensão das tempestades elétricas. No entanto, FTA diminui exponencialmente com o aumento da área das tempestades elétricas.

%Uma tempestade elétrica com 10$^5$ km$^2$, terá maior número de descargas observadas durante o tempo de visada do LIS do que uma com 10$^2$ km$^2$.

\begin{figure}[!ht]
  \centering
  \includegraphics[height=9cm]{img/FtaFt/area_FTA_FT}   
  \caption{Dispersão entre as áreas das tempestades elétricas e os valores de FTA e FT. As linhas horizontais marcam os valores de FTA (cor preta) e FT (cor azul) referente ao 90\textsuperscript{\underline{o}} percentil das respectivas amostragem.}
  \label{areaFTAFTA}  
\end{figure}

\begin{figure}[!hb]
  \centering 
  \includegraphics[height=9cm]{img/FtaFt/volChuva_FTA_FT}
  \caption{Dispersão entre o volume de chuva das tempestades elétricas e os valores de FTA e FT.  As linhas horizontais marcam os valores de FTA (cor preta) e FT (cor azul) referente ao 90\textsuperscript{\underline{o}} percentil das respectivas amostragem.}
  \label{volchuvaFTAFT}
\end{figure}

%\begin{figure}[!hb]
%  \centering 
%  \subfloat[areas versus taxa de raios]{ \includegraphics[height=7.5cm]{img/FtaFt/area_FTA_FT}\label{areaFTAFTA}} \\
%  \subfloat[volume de chuva]{\includegraphics[height=7.5cm]{img/FtaFt/volChuva_FTA_FT}\label{pdfFt}} 
%  \caption{Dispersão referente aos índices FTA e FT.}
%  \label{areaFtaFt}
%\end{figure} 
%Quando a densidade espacial de descargas aumenta muito em uma região com centenas de quilômetros quadrados, em torno de 100 descargas intranuvens para uma nuven-solo, como por exemplo os maiores valores de $Z=IC/CG$ mostrados por \cite{evandro2009} na região de Campo Grande - MS no Brasil, a capacidade do LIS de identificar brilhos transientes provavelmente fica comprometida devido a resolução horizontal da CCD.

Ao normalizar a taxa de raios no tempo por $A_t$, o número de raios fica diluído na extensão do sistema, evidenciando que os maiores valores de FTA correspondem aos sistemas com as maiores densidades espaciais de raios, cuja a área e o número de raios são menores do que nos sistemas com extremos de FT.

Na figura \ref{volchuvaFTAFT}, observa-se que conforme aumenta FT o volume de chuva das tempestades elétricas também aumenta exponencialmente, de maneira semelhante ao aumento de FT com a área (figura \ref{areaFTAFTA}). Para FTA, há um comportamento inverso. Conforme aumenta FTA, o volume de chuva dos sistemas diminui.  

A frequência de ocorrência das temperaturas de brilho associadas a radiância espectral observada no canal 4 do VIRS para todos os pixeis que definiram as áreas dos sistemas, é mostrada na figura \ref{tb}. Observa-se que o maior valor de probabilidade para a curva das tempestades elétricas com índice extremo de FTA, possui temperatura de topo de nuvens aproximadamente 10 K mais frias do que nas tempestades elétricas com extremos de FT, indicando que a convecção nos sistemas ordenados por FTA é mais profunda na maioria das situações.



\begin{figure}[!ht]
  \centering 
  \includegraphics[height=9cm,trim=220 0 0 0,clip]{img/tb/TbAreas}
  \caption{Densidade de probabilidade de temperatura de brilho em infravermelho (VIRS 10.8$\mu$m) do topo das nuvens das tempestades elétricas com extremos de FTA e FT.}
  \label{tb}
\end{figure}

\citeonline{morales2003} ao desenvolver a \textit{Sferics Infrared Rainfall Technique} (SIRT), mostram que as regiões com temperatura de brilho inferior a 215 K e com ocorrência de \textit{sferics} foram as regiões categorizadas como de maior precipitação associada.\sigla{name={SIRT},description={\textit{Sferics Infrared Rainfall Technique} }}

Neste trabalho de pesquisa, ao selecionar as tempestades elétricas com índice extremo de FTA, os maiores valores de probabilidade de ocorrência, conforme é mostrado na figura \ref{tb}, concentram-se em temperaturas de brilho abaixo de 215 K.


Nas figuras \ref{pdffracaoFTA}, \ref{cdffracaoFTA} e \ref{pdffracaoFTA},  \ref{cdffracaoFTA}, apresenta-se o estudo das probabilidades das frações de chuva para as tempestades elétricas com valores extremos de FTA e FT. As curvas denominadas como convectivo, estratiforme e outros, correspondem a fração de perfis do PR classificados como convectivo, estratiforme e outros em relação a toda a área de chuva. A curva denominada na legenda como chuva total corresponde a fração da área de chuva em relação a área total da tempestade elétrica. A curva denominada como varredura do PR, mostra a fração da área da tempestade elétrica que esteve dentro da varredura do PR.


\begin{figure}[!ht]
  \centering
  \includegraphics[height=9cm]{img/FtaFt/fracaoChuva_pdf_topFTA}   
  \caption{Densidade de probabilidade das frações de área de chuva das tempestades elétricas com extremos de FTA, que foram classificadas (2A23) como convectiva, estratiforme e outros, em relação a toda a área de chuva observada pelo PR; da fração de chuva total, que corresponde a fração da área de chuva observada pelo PR em relação a $A_t$ das tempestades elétricas que é definida pelas observações do VIRS e também da fração das tempestades elétricas contida na varredura do PR, mesmo sem chuva válida.  }
  \label{pdffracaoFTA}  
\end{figure}

\begin{figure}[!ht]
  \centering 
  \includegraphics[height=9cm]{img/FtaFt/fracaoChuva_cdf_topFTA}
  \caption{Densidade de probabilidade cumulativa das frações de chuva das tempestades elétricas com valores extremos de FTA.}
  \label{cdffracaoFTA}
\end{figure}


\begin{figure}[!ht]
  \centering
  \includegraphics[height=9cm]{img/FtaFt/fracaoChuva_pdf_topFT}   
  \caption{Densidade de probabilidade das frações de área de chuva das tempestades elétricas com extremos de FT, que foram classificadas (2A23) como convectiva, estratiforme e outros, em relação a toda a área de chuva observada pelo PR; da fração de chuva total, que corresponde a fração da área de chuva observada pelo PR em relação a $A_t$ das tempestades elétricas que é definida pelas observações do VIRS e também da fração das tempestades elétricas contida na varredura do PR, mesmo sem chuva válida.}
  \label{pdffracaoFT}  
\end{figure}

\begin{figure}[!ht]
  \centering 
  \includegraphics[height=9cm]{img/FtaFt/fracaoChuva_cdf_topFT}
  \caption{Densidade de probabilidade cumulativa das frações de chuva das tempestades elétricas com valores extremos de FT.}
  \label{cdffracaoFT}
\end{figure}



Avaliando a densidade de probabilidade da fração convectiva e fração estratiforme das tempestades elétricas, figuras \ref{pdffracaoFTA} e \ref{pdffracaoFT}, verifica-se que para os extremos de FTA as tempestades elétricas são mais frequentemente observadas com 70\% de área convectiva e 30\% de área estratiforme, enquanto que para os extremos de FT, 20\% de fração convectiva e 75\% de fração estratiforme.
%juntamente com a as respectivas distribuições de probabilidade acumulativa,

Verifica-se nas figuras \ref{pdffracaoFT} e \ref{cdffracaoFT} que para a maioria dos sistemas com FT extremo, o PR conseguiu observar 25\% da área das tempestades elétricas. Nas figuras \ref{pdffracaoFTA} e \ref{cdffracaoFTA} das frações de chuva dos sistemas com FTA extremo, o PR observou com maior frequência 100\% da área das tempestades elétricas. Portanto a fração de chuva total dos extremos de FT (figura \ref{pdffracaoFT}) possui um valor de apenas $\simeq$10\% devido a varredura do PR ser menor do que a do VIRS e as tempestades elétricas com os maiores valores de FT abrangerem uma extensão que ultrapassa o alcance do PR. 


Os sistemas selecionados pelo 90\textsuperscript{\underline{o}} percentil do índice FT possuem maior extensão em área e maior volume de chuva. São sistemas com vasta extensão estratiforme conforme descrevem \citeonline{Rasmussen2011}. As regiões das tempestades elétricas com precipitação convectiva, as quais tem potencial de gerar chuva de granizo, frentes de rajada e tornados, ocupam área bem menor do que as áreas com  precipitação estratiforme \cite{Jr2007}.

A maior fração convectiva e menor tamanho das tempestades elétricas com FTA extremos sugerem sistemas em estágio de maturação. Conforme as tempestades elétricas vão entrando em estágio maduro e dissipativo, vão ganhando área de chuva estratiforme e podem começar a se enquadrar no grupo dos extremos de FT. 

%.....
%Para avaliar qual dos índices representaram a maior severidade de tempo, a morfologia da estrutura 3D da precipitação foi estudada por meio dos diagramas CFAD, CCFAD, CFTD e CCFTD. %para os 10\% das amostras de FT e FTA com os maiores valores.
%......

\section{SEVERIDADE COM BASE NA ESTRUTURA 3D DA PRECIPITAÇÃO}

Nesta etapa iremos avaliar a intensidade convectiva com base nos perfis de $Z_c$ do PR, contidos nos sistemas com índices extremos de FTA e FT. 

Nas figuras \ref{ftacfadwithout}, \ref{ftcfadwithout},  \ref{ftacfadwith} e \ref{ftcfadwith} foram calculados os CFADs das tempestades elétricas com índices FTA e FT extremos, para cada região de 10 por 10 graus na superfície sobre a AS. Para localizar a caixa de 10 por 10 graus em que cada sistema esteve contido, foi considerado a latitude e longitude do centro geométrico da área definida por cada sistema. Desta forma, foram obtidas as amostragens de probabilidade de FTA e FT, da mesma maneira que mostrado na figura \ref{pdfFTAFT} ou \ref{cdfFTAFT}, porém para cada região de 10$^{\circ}$ $\times$ 10$^{\circ}$. Assim, o estudo da precipitação tridimensional é feito apenas para as tempestades elétricas com valores de FTA ou FT extremos referentes a cada região, ou seja, referente ao 90\textsuperscript{\underline{o}} percentil de cada  amostragem de FTA e FT regionalizada a cada 10$^{\circ}$ $\times$ 10$^{\circ}$.


\begin{figure}[!ht]
  \centering{  
  \subfloat[]{\includegraphics[height=1.0cm]{img/grids/nucleosRaios/colorbar_virs}\label{barravirs}}\\
  \subfloat[]{\includegraphics[height=6.cm,trim=0 47cm 0 0,clip]{img/grids/nucleosRaios/001_topFTA_25471_0003} \label{nr1}} 
  \subfloat[]{\includegraphics[height=6.cm,trim=0 47cm 0 0,clip]{img/grids/nucleosRaios/002_topFTA_36502_0001} \label{nr2}} \\
  \subfloat[]{\includegraphics[height=6.cm,trim=0 47cm 0 0,clip]{img/grids/nucleosRaios/003_topFTA_03444_0001} \label{nr3}} 
  \subfloat[]{\includegraphics[height=6.cm,trim=0 47cm 0 0,clip]{img/grids/nucleosRaios/004_topFTA_05694_0005} \label{nr4}} 
  }
  \caption{Núcleos de raios das tempestades elétricas. Os pontos na cor  verde são os eventos e os símbolos de positivo na cor preta são os raios. Os pixeis em vermelho são as regiões dos núcleos de raios, definidas a partir da projeção dos eventos em uma grade regular de 0.05$^{\circ}$.} %(0.05$^{\circ}$ $\times$ 0.05$^{\circ}$)
\label{nucleosRaios}
\end{figure}



%----------------------------------------
\begin{sidewaysfigure}%[!H]
\centering
\includegraphics[width=19.5cm]{img/precipitacao3d/severo/percentil/90th/cfad10_semraio_topFTA_percentil}
\caption{CFADs para os extremos de FTA. Porção da precipitação sem raios.}
\label{ftacfadwithout}
\end{sidewaysfigure} 


\begin{sidewaysfigure}%[!H]
\centering
\includegraphics[width=19.5cm]{img/precipitacao3d/severo/percentil/90th/cfad10_semraio_topFT_percentil}
\caption{CFADs para os extremos de FT	. Porção da precipitação sem raios.}
\label{ftcfadwithout}
\end{sidewaysfigure} 
%----------------------------------------

%\begin{figure}[!ht]
%  \centering
%  \includegraphics[height=13.5cm]{img/precipitacao3d/severo/percentil/90th/cfad10_semraio_topFTA_percentil}
% \caption{CFADs para os extremos de FTA. Porção da precipitação sem raios.}
% \label{ftacfadwithout}
%\end{figure} 

\begin{sidewaysfigure}%[!H]
  \centering
  \includegraphics[width=19.5cm]{img/precipitacao3d/severo/percentil/90th/cfad10_comraio_topFTA_percentil}
  \caption{CFADs para os extremos de FTA. Porção da precipitação com raios.}
  \label{ftacfadwith}   
\end{sidewaysfigure} 


\begin{sidewaysfigure}%[!H]
  \centering
  \includegraphics[width=19.5cm]{img/precipitacao3d/severo/percentil/90th/cfad10_comraio_topFT_percentil}
  \caption{CFADs para os extremos de FT. Porção da precipitação com raios.}
  \label{ftcfadwith}   
\end{sidewaysfigure} 

%\begin{figure}[!ht]
%  \centering
%  \includegraphics[height=13.5cm]{img/precipitacao3d/severo/percentil/90th/cfad10_comraio_topFTA_percentil}
%  \caption{CFADs para os extremos de FTA. Porção da precipitação com raios.}
%  \label{ftacfadwith}   
%\end{figure} 

	

As posições geográficas dos eventos do LIS e dos perfis de $Z_c$ válidos do PR, foram projetadas em uma grade regular de 0.05 graus. Os perfis de $Z_c$ projetados em pontos de grade em que tiveram eventos do LIS, definiram as regiões aqui denominadas como precipitação dos núcleos de raios. A figura \ref{nucleosRaios}, mostra as regiões dos núcleos de raios das tempestades elétricas. Observe que na parte superior e inferior das figuras \ref{nr1}, \ref{nr2}, \ref{nr3} e \ref{nr4}, há informações referentes a data e hora em que o sistema foi observado, número de raios/eventos (FL/EV), fração do sistema observado pelo PR, área do sistema (A), semi-eixo maior (a), menor (b), distância focal (2c) e excentricidade (e) de uma elipse ajustada as dimensões do sistema. A barra de cores na figura \ref{barravirs}, corresponde as temperaturas de brilho ($<$258K) associada a radiância espectral em infravermelho observada pelo VIRS.


Os CFADs foram calculados para a precipitação dos núcleos de raios das tempestades elétricas, figuras \ref{ftacfadwith} e \ref{ftcfadwith}, como também para a precipitação fora dos núcleos de raios, figuras \ref{ftacfadwithout} e \ref{ftcfadwithout}.

Note que no canto superior direito de cada CFAD temos alguns valores estatísticos que representam: (\%)  a porcentagem de perfis convectivos, estratiformes e outros, respectivamente; (P) o número de perfis do PR computados; (L) o número de ocorrência de $Z_c$ no nível de altitude de máxima ocorrência; (H) o nível de altitude, em quilômetros, aonde ocorreu o máximo de ocorrências de $Z_c$; (N) o número de tempestades elétricas computadas.

\simbolo{name={\%},description={Nos diagramas, CFAD, CCFAD, CFTD e CCFTD, representam: a porcentagem de perfis convectivos, estratiformes e outros, respectivamente}}
\simbolo{name={P},description={Nos diagramas, CFAD, CCFAD, CFTD e CCFTD, representam: número de perfis do PR computados}}
\simbolo{name={L},description={Nos diagramas, CFAD, CCFAD, CFTD e CCFTD, representam: o número de ocorrência de $Z_c$ no nível de altitude de máxima ocorrência}}
\simbolo{name={H},description={Nos diagramas, CFAD, CCFAD, CFTD e CCFTD, representam: o nível de altitude, em quilômetros, aonde ocorreu o máximo de ocorrências de $Z_c$}}
\simbolo{name={N},description={Nos diagramas, CFAD, CCFAD, CFTD e CCFTD, representam: o número de tempestades elétricas computadas}}

Comparando os CFADs da chuva com e sem raios, representados para os extremos de FTA nas figuras \ref{ftacfadwithout} e \ref{ftacfadwith} e para os extremos de FT, nas figuras \ref{ftacfadwithout} e \ref{ftcfadwithout}, é evidente que a fração das tempestades elétricas sem raios é a porção de menor intensidade convectiva e a fração eletricamente ativa é a região de maior velocidade vertical. Os níveis de contorno de probabilidades dos CFADs da precipitação sem raios possuem suas máximas altitudes aproximadamente 3 quilômetros abaixo das máximas altitudes atingidas pelos contornos dos CFADs da precipitação com raios. A porção sem raios dos sistemas possuíram maior percentual de perfis estratiformes e menores valores de $Z_c$ com os contornos de probabilidades entre 1-10\%, em todos os níveis de altitude.

%\begin{figure}[!ht]
%  \centering
%  \includegraphics[height=13.5cm]{img/precipitacao3d/severo/percentil/90th/cfad10_semraio_topFT_percentil}%
% \caption{CFADs para os extremos de FT. Porção da precipitação sem raios.}
% \label{ftcfadwithout}
%\end{figure} 

%\begin{figure}[!ht]
%  \centering
%   \adjustbox{trim={0\width} {0.435\height} {0\width} {0\height} , clip}%
%   {\includegraphics[width=\textwidth]{img/precipitacao3d/severo/percentil/90th/cCumFad_10deg_semraio_topFTpercentil}}
% \caption{CCFDs para os extremos de FT entre 20S-10N e 90W-30W. Porção da precipitação sem raios.}
% \label{ftccfadwithout}
%\end{figure} 


Se avaliarmos apenas os níveis de contorno com probabilidade entre 2-3.7\% (cor verde), observa-se que os máximos de $Z_c$ associados à chuva na superfície da porção sem raios, figuras \ref{ftacfadwithout} e \ref{ftcfadwithout}, não ultrapassaram 40 dBZ em nenhuma região da AS, enquanto que para a porção de chuvas com raios, figuras \ref{ftacfadwith} e \ref{ftcfadwith}, os valores de $Z_c$ entre 0-2 km de altitude registram valores entre 45-50 dBZ. A fração da precipitação eletricamente ativa possui maior percentual de perfis convectivos e com maiores valores de $Z_c$ associado aos contornos de probabilidade, confirmando a correlação positiva entre a taxa de raios e o volume de chuva \cite{Petersen1998}.

%A convecção é mais ativa nas regiões dos núcleos de raios, aonde a precipitação está associadas com frentes de rajadas, chuvas de granizo e enchentes rápidas. Fora dos núcleos de raios temos a parte da precipitação mais estratiforme, composta por hidrometeoros que não possuem velocidade terminal suficiente para precipitar nos núcleos de raios, e caem mais afastados da região eletricamente ativa.     % Dependendo principalmente das condições de calor umidade e cisalhamento vertical do vento as células 


% \caption{Diagramas de Contorno de Frequência por Altitude (CFADs). Em cada CFAD pode-se verificar: a porcentagem (\%) de perfis convectivos, estratiformes e outros, respectivamente; (P) o numero de perfis do PR computados, (L) o número de ocorrência de refletividade no nível de máxima ocorrência e (H) o nível de máxima ocorrência.}

A figura \ref{ftcfadwithout} mostra que a precipitação sem raios dos extremos de FT na região tropical, entre 20S-10N e 90W-30W, possui banda brilhante marcada entre 4-5 km de altitude, principalmente nos perfis com probabilidade de ocorrência entre 2-5.3\%, nas cores de contorno em verde e amarelo. 

Podemos observar a banda brilhante entre 20S-10N e 90W-30W  dos sistemas com  extremo de FT na porção sem raios de maneira mais elucidativa por meio dos CCFADs da figura \ref{ftccfadwithout}, os quais evidenciam que entre o 12\textsuperscript{\underline{o}} e o 95\textsuperscript{\underline{o}} percentil da amostragem de probabilidade de $Z_c$ por altitude, há uma queda no valor de $Z_c$ logo abaixo de 5 quilômetros de altitude em cada região de 10 por 10 graus. 

No entanto, para a região entre 20S-10N e 90W-30W, ao avaliar os CFADs da figura \ref{ftacfadwithout} ou CCFADs da figura\ref{ftaccfdsubtrop}, que representam a porção sem raios da precipitação tridimensional dos sistemas com índice extremo de FTA, não se observa banda brilhante marcada nos contornos de probabilidade de $Z_c$ por altitude. Há um aumento contínuo de $Z_c$, conforme os níveis de altitude diminuem, sem a diminuição abrupta de $Z_c$ logo abaixo de 5 quilômetros, mostrando que os processos de acreção e colisão coalescência são ativos, indicando maior velocidade vertical no ambiente dos sistemas extremos de FTA do que no ambiente dos sistemas extremos de FT. 

%observa-se que, entre 20S-10N e 90W-30W,  
%a banda brilhante é evidente apenas nas regiões costeiras e oceânicas, nas caixas entre 0-10N  %e 90-80W, entre 10-0S e 40-30W e entre 20-10S e 70-60W.

As chuvas na superfície associadas com a precipitação sem raios das tempestades elétricas entre 20S-10N e 90W-30W, referente aos extremos de FT têm maiores valores de probabilidades com valores de $Z_c$ mais moderados do que quando compara-se com os extremos de FTA, os quais possuem perfis de $Z_c$ com maior aleatoriedade, mas podem atingir valores em dBZ superiores. Observe as diferenças entre as figuras \ref{ftaccfdsubtrop} e figura \ref{ftccfadwithout}. Note como entre 20S-10N e 90W-30W os contornos de probabilidade, principalmente entre 0.3-3.7\% representados pelas cores em azul e verde, são mais alargados na chuva sem raios dos extremos de FTA, do que na chuva sem raios dos extremos de FT. Na chuva sem raios dos sistemas com extremos de FT, os contornos da figura \ref{ftcfadwithout} são mais estreitos, indicando menor aleatoriedade nos valores de $Z_c$ observados.

\begin{sidewaysfigure}%[!H]
  \centering
  \includegraphics[width=19.5cm]{img/precipitacao3d/severo/percentil/90th/cCumFad_10deg_semraio_topFTApercentil}
  \caption{CCFDs para os extremos de FTA. Porção da precipitação sem raios.}
  \label{ftaccfdsubtrop}   
\end{sidewaysfigure} 

\begin{sidewaysfigure}%[!H]
  \centering
  \includegraphics[width=19.5cm]{img/precipitacao3d/severo/percentil/90th/cCumFad_10deg_semraio_topFTpercentil}
  \caption{CCFDs para os extremos de FT. Porção da precipitação sem raios.}
  \label{ftccfadwithout}   
\end{sidewaysfigure} 

%\begin{figure}[!ht]
%  \centering  
%  \adjustbox{trim={.0\width} {.04\height} {0\width} {.565\height},clip}%
%  \centering  
%  \adjustbox{trim={.349\width} {.045\height} {.322\width} {.565\height},clip}%  
%  {\includegraphics[width=27cm] {img/precipitacao3d/severo/percentil/90th/cCumFad_10deg_semraio_topFTApercentil}}
% \caption{CCFDs para os extremos de FTA entre 40-20S e 70-50W. Porção da precipitação sem raios.}
% \label{ftaccfdsubtrop}
%\end{figure} 



Na região entre 40-20S e 70-50W que engloba a Bacia do Prata, a banda brilhante foi menos evidente nos contornos de probabilidade associados a estrutura tridimensional da precipitação fora dos núcleos de raios, tanto para os extremos de FT, figura \ref{ftccfadwithout}, quanto para os extremos de FTA, figura \ref{ftacfadwithout}. 

Entre 40-20S e 70-50W, a porção sem raios da chuva dos extremos de FTA mostra que entre 0-5 km de altitude, a probabilidade de valores inferiores de $Z_c$ em relação as porções sem raios dos extremos de FT é maior. Observe como a mediana das amostras de probabilidades, marcada pela linha de contorno na cor preta no 50\textsuperscript{\underline{o}} percentil do CCFAD em cada caixa de 10 por 10 graus nas figuras \ref{ftccfdsubtrop} e \ref{ftaccfdsubtrop}, indica maior taxa de precipitação entre 0-5 km na porção sem raios das tempestades elétricas com índice FT extremo, figura \ref{ftccfdsubtrop}, mesmo que a estatística na parte superior direita de cada CCFAD indique maior percentual de perfis convectivos para a porção sem raios dos extremos de FTA, figura \ref{ftaccfdsubtrop}.


%, evidenciando que a chuva sem raios dos sistemas com as maiores taxas de raios no tempo é mais severa nesta região.

%\begin{figure}[!ht]
%  \centering  
%  \adjustbox{trim={.349\width} {.045\height} {.%322\width} {.565\height},clip}%
%  {\includegraphics[width=27cm] {img/precipitacao3d/severo/percentil/90th/cCumFad_10deg_semraio_topFTpercentil}}
% \caption{CCFDs para os extremos de FT entre 40-20S e 70-50W. Porção da precipitação sem raios.}
% \label{ftccfdsubtrop}
%\end{figure} 


Porém, a precipitação contida fora dos núcleos de raios dos sistemas extremos selecionados pelo índice FTA, situados entre 40-20S e 70-50W,  e que é explicitada por meio dos CFADs da figura \ref{ftacfadwithout}, revela que a probabilidade entre 0.001-2\%, representados pelas cores de contorno em preto e azul, atingem valores superiores de $Z_c$ do que quando compara-se com os sistemas extremos de FT, na figura \ref{ftcfadwithout}, também entre 40-20S e 70-50W.

Apesar da mediana das probabilidades dos CFADs mostrarem que a precipitação entre 0-5 km de altitude foi mais intensa para os sistemas extremos de FT e localizados entre 40-20S e 70-50W, ao avaliar os contornos de probabilidade cumulativa dos CCFADs na figura \ref{ftaccfdsubtrop}, referente ao estudo da estrutura tridimensional da precipitação fora dos núcleos de raios dos extremos de FTA, observa-se que, acima do 80\textsuperscript{\underline{o}} percentil os valores de $Z_c$ foram superiores em relação aos sistemas com índice extremo de FT, na figura \ref{ftccfdsubtrop}.

Os CFADs referentes as tempestades elétricas selecionadas por FTA possuem contornos de probabilidade em níveis de altitude mais elevados do que os CFADs dos sistemas selecionados por FT, tanto para a porção com raios quanto para a porção sem raios da precipitação dos sistemas.

%A diferença mais notável pode ser observada entre a figura \ref{ftacfadwithout} e \ref{ftcfadwithout} para 0S-10S e 50W-60W, que abrange principalmente o estado do Pará, e parte do Amazônas, Tocantis e Mato Grosso. O CFAD em \ref{ftacfadwithout} define valores de probabilidade em altitude 2 km mais elevada do que em \ref{ftcfadwithout}.


%Nas regiões entre 10N-0S e 70W-80W e entre 20S-40S e 50W-60W, em que \cite{cecil2005} apontam como região das tempestades mais severas na América do Sul, os CFADs em \ref{ftacfadwith} e \ref{ftacfadwithout} possuem contornos de probabilidade aproximadamente 1 km mais elevado do que em \ref{ftcfadwith} e \ref{ftcfadwithout}.

Como o último nível de altitude dos CFADs deste trabalho é limitado por altitudes com até 10\% de L, a maior definição de probabilidades de ocorrência de $Z_c$ em altitude para as tempestades selecionadas pelo índice FTA, indica que a convecção é mais intensa nos extremos de FTA do que nos extremos de FT. Principalmente quando observamos a morfologia da estrutura tridimensional da precipitação dos núcleos de raios, para os extremos do FTA e FT, expressa nos CFADs das figuras \ref{ftacfadwith} e \ref{ftcfadwith}, aonde os perfis de precipitação são classificados majoritariamente como convectivos.

A precipitação é bem mais frequente próxima da superfície, entre 0-3 km de altitude. Acima da região de mistura, a precipitação é mais rara de ocorrer. Em \cite{liu2008}, é mostrado que a densidade espacial de sistemas com no mínimo 20 dBZ em 2 km de altitude é globalmente maior do que os sistemas que atingem 20 dBZ em níveis superiores de altitude.


%A região de 10 por 10 graus, a qual o valor de H marcado no topo direito de cada CFAD, é menor para a precipitação dos núcleos de raios dos sistemas com extremo de FTA, figura \ref{ftacfadwith}, do que para  a precipitação dos núcleos de raios dos sistemas com extremos de FT, figura \ref{ftcfadwith}, e mesmo assim, o CFAD dos extremos de FTA, figura \ref{ftacfadwith}, possuiu maior altitude nos níveis de contorno de probabilidade de $Z_c$, o índice FTA mostra que a chuva  
%esteve associado com maior severidade de tempo do que FT.
%Pois, mesmo que a refletividade mais ocorrente esteja abaixo da região de mistura, a precipitação também é frequente conforme o aumento da altitude, mostrando que nestas regiões, os sistemas com índice FTA extremo têm maior número de ocorrência de chuva 
%mais chuvas na superfície e também maior precipitação acima de 10 km de altitude.
%com bastante representatividade estatística.
%maior quantidade de hidrometeoros na região de mistura e

Por exemplo na região do Panamá, Colômbia e Equador, entre 10N-0S e 70W-80W, o CFAD da figura \ref{ftacfadwith} possui contornos de probabilidade até 16 km de altitude. Na figura \ref{ftcfadwith}, os níveis de contorno param em 15 km.

A precipitação tridimensional observada nos núcleos de raios, explicitada nos contorno dos CFADs a cada 10 graus na figura \ref{ftacfadwith}, referente ao índice FTA, mostram valores de refletividade entre 1-3 dBZ maiores do que na figura \ref{ftcfadwith}, referente ao índice FT, principalmente quando observa-se os contornos de probabilidade de $Z_c$ acima de 5 km de altitude. Para a precipitação entre 1-2 km de altitude os valores são mais semelhantes entre as tempestades elétricas selecionadas por FTA e FT. 

%Porém, nos sistemas extremos de FTA, figura \ref{ftacfadwith}, há um estreitamento da região de contorno com os maiores valores de probabilidade associada a chuva na superfície, entre 3-5\%. Entre 20S-40S e 40-70W, o estreitamente é maior do que as demais regiões mostrando que as chuvas possuem maior probabilidade de estarem associadas com valores de 45 dBZ em \ref{ftacfadwith}.      

As mais baixas probabilidades de $Z_c$ observadas nos CFADs das figuras \ref{ftacfadwith} e \ref{ftcfadwith}, estão associadas com a estrutura tridimensional da precipitação mais severa. Observe os contornos de probabilidade entre 0.001-0.5\%. Estes níveis de contorno revelam os valores de $Z_c$ da precipitação mais rara entre os sistemas com índice extremo de FTA e FT, os quais provavelmente estiveram associados com enchentes rápidas, alta taxa de raios, chuva de granizo, fortes rajadas de vento e até mesmo ocorrência de tornados em algumas regiões. 
% nas figuras \ref{ftacfadwith} e \ref{ftcfadwith}

Os valores maiores valores de $Z_c$ foram registrados na figura \ref{ftacfadwith} entre 20S-40S e 40W-70W, sobre a Bacia do Rio da Prata, que abrange o Sul do Brasil, Paraguai, Uruguai e Argentina. A dinâmica de formação de Sistemas Convectivos de Meso-escala, como é discutido em \cite{Velasco1987} e \cite{Durkee2009}, somados com efeitos de topografia, como por exemplo na região da Serra de Córdoba na Argentina, a qual \cite{Rasmussen2011} mostram grande ocorrência de convecção profunda, promoveram sistemas em que a estrutura tridimensional da precipitação dos núcleos de raios atingiram valores de $Z_c$ superiores a 45 dBZ entre 10-15 km de altitude e chuvas na superfície com $Z_c$ acima de 55 dBZ, como mostram os contornos de probabilidade entre 0.001-0.5\%.

%Os processos de eletrificação dos extremos de FTA pode estar associada com a presença de granizo, cristais de gelo e gotas água super-resfriada, enquanto que para os extremos de FT, 




\subsection{A precipitação dos sistemas severos e o perfil atmosférico de temperatura.}

Os diagramas CCFTD e CFTD, descritos em \ref{chuvaEtemperatura}, são expostos nas figuras \ref{ccftd_fta_com}, \ref{ccftd_ft_com}, \ref{cftd_fta_com} e \ref{cftd_ft_com}, associados as tempestades elétricas com índice extremo de FTA e FT, apenas em suas porções com raios.

A partir dos CCFTDs das figuras \ref{ccftd_fta_com} e \ref{ccftd_ft_com}, iremos avaliar a intensidade convectiva dos sistemas com índice extremo de FTA e FT em determinadas regiões, com base na velocidade de crescimento ou decrescimento dos valores de $Z_{c}$ associados os contornos de probabilidade do 30\textsuperscript{\underline{o}}, 50\textsuperscript{\underline{o}}, 70\textsuperscript{\underline{o}} e 95\textsuperscript{\underline{o}} percentil das amostras de probabilidades expressas nos CFTDs das figuras \ref{cftd_fta_com} e \ref{cftd_ft_com}.	

\begin{sidewaysfigure}%[!H]
\centering
\includegraphics[width=19.5cm]{img/precipitacao3d/severo/percentil/90th/cftd_10deg_comraio_topFTApercentil}
\caption{CFTDs para os extremos de FTA. Porção da precipitação com raios.}
\label{cftd_fta_com}
\end{sidewaysfigure} 

\begin{sidewaysfigure}%[!H]
\centering
\includegraphics[width=19.5cm]{img/precipitacao3d/severo/percentil/90th/ccftd_10deg_comraio_topFTApercentil}
\caption{CCFTDs para os extremos de FTA. Porção da precipitação com raios.}
\label{cftd_fta_com}
\end{sidewaysfigure} 

\begin{sidewaysfigure}%[!H]
\centering
\includegraphics[width=19.5cm]{img/precipitacao3d/severo/percentil/90th/cftd_10deg_comraio_topFTpercentil}
\caption{CFTDs para os extremos de FT. Porção da precipitação com raios.}
\label{cftd_ft_com}
\end{sidewaysfigure} 

\begin{sidewaysfigure}%[!H]
\centering
\includegraphics[width=19.5cm]{img/precipitacao3d/severo/percentil/90th/ccftd_10deg_comraio_topFTpercentil}
\caption{CCFTDs para os extremos de FT. Porção da precipitação com raios.}
\label{ccftd_ft_com}
\end{sidewaysfigure} 

Então, para a região central da Bacia do Rio Amazonas, entre 10-0S e 70-60W, extraí-se as linhas de contorno do CCFTD referentes as probabilidades acumulativas de 30\%, 50\%, 70\% e 95\%. Desta forma obtemos quatro funções $f(x)=y$, \simbolo{name={$f(x)=y$},description={Função de uma variável}} em que $y$ corresponde aos valores de $Z_c$ e $x$ o perfil atmosférico de temperatura. Fazendo a derivada $\dfrac{dy}{dx}$ pode-se avaliar taxa de variação de $Z_c$ por temperatura (dBZ/\textsuperscript{o}C), para diferentes regimes de chuva, das mais frequentes até as mais raras, como mostra a figura \ref{deriv_amazonas}.

\begin{figure}[!ht]
  \centering
  \includegraphics[height=9cm]{img/precipitacao3d/deriv_ccftd/deriv_contornos_cdf_2_1}
  \caption{Taxa de variação de $Z_c$ no perfil de temperatura atmosférico para a região central da Bacia do Rio Amazonas, entre 10-0S e 70-60W.}
  \label{deriv_amazonas}  
\end{figure} 

% nos diferentes quartis do CCFTD dos extremos de FTA,  figura \ref{ccftd_fta_com},  e dos extremos de FT, figura \ref{ccftd_ft_com}. 

Na figura \ref{deriv_amazonas}, observa-se que a taxa de aumento de $Z_c$ em torno de -40 \textsuperscript{o}C e -15 \textsuperscript{o}C, é maior para os extremos de FTA, porém em torno de 0 \textsuperscript{o}C, a taxa de aumento de $Z_c$ é maior para os extremos de FT, mostrando que os hidrometeoros dos sistemas extremos de FTA, crescem em regiões mais frias do que nos extremos de FT.

% agregação e acreção é maior para os sistemas extremos de FTA e na região de derretimento, em torno de 0 \textsuperscript{o}C, o efeito da banda brilhante é mais acentuado para os extremos de FT.  

O aumento do fator de refletividade em torno de 0 \textsuperscript{o}C está associado a mudança do índice de refração da água devido a sua fusão. Já o aumento do fator de refletividade em torno de -40 \textsuperscript{o}C e -15 \textsuperscript{o}C representam o crescimento de hidrometeoros por agregação e acreção \cite{Fabry1995,Takahashi1978}.

Note na figura \ref{deriv_amazonas}, como a precipitação do 95\textsuperscript{\underline{o}} percentil de probabilidade de ocorrência tanto para FTA quanto para FT, é o regime de precipitação mais severa. Pois, há o crescimento de $Z_c$ em torno de -10 \textsuperscript{o}C e -15 \textsuperscript{o}C e não há banda brilhante, indicando precipitação a partir de granizo\footnote{Em \cite{Fabry1995}, este tipo de perfil é discutido como chuva a partir de gelo compacto.}. No 30\textsuperscript{\underline{o}}, 50\textsuperscript{\underline{o}} e 70\textsuperscript{\underline{o}} percentil dos extremos de FT, o efeito da banda brilhante associada ao derretimento é mais evidente do que para os extremos de FTA. 

Quando comparamos a região central da Bacia do Rio Amazonas, com a região central da Bacia do Rio da Prata, entre 30-20S e 60-50W, a microfísica de eletrificação se mostra diferente em cada local. Observa-se que no 50\textsuperscript{\underline{o}} percentil, a taxa de crescimento de $Z_c$ entre -40 \textsuperscript{o}C e -20 \textsuperscript{o}C é maior para a região da Bacia do Prata, figura \ref{deriv_prata}, do que para a região da Bacia Amazônica, figura \ref{deriv_amazonas}, tanto para os sistemas extremos de FTA quando para os sistemas extremos de FT, indicando maior crescimento de flocos de neve na precipitação severa sobre a Bacia do Prata. 


\begin{figure}[!ht]
  \centering
  \includegraphics[height=9cm]{img/precipitacao3d/deriv_ccftd/deriv_contornos_cdf_3_3}
  \caption{Taxa de variação de $Z_c$ no perfil de temperatura atmosférico para a região central da Bacia do Rio da Prata, entre 30-20S e 60-50W.}
  \label{deriv_prata}  
\end{figure} 

Apesar do 95\textsuperscript{\underline{o}} percentil mostrar maiores taxas de dBZ/\textsuperscript{o}C, em -15 \textsuperscript{o}C tanto para FTA quanto FT sobre a Bacia do Rio Amazonas, na figura \ref{deriv_amazonas}, do que sobre a Bacia do Rio da Prata, na figura \ref{deriv_prata}, os contorno de probabilidade acumulativa de 95\% nos CCFTD das figuras \ref{ccftd_fta_com} e \ref{ccftd_ft_com}, em -15 \textsuperscript{o}C, mostram valores de $Z_c$ de aproximadamente 3 dBZ superiores na região da Bacia Platina. Mesmo que o 95\textsuperscript{\underline{o}} percentil mostre maior crescimento de hidrometeoros na região mista sobre a Bacia Amazônica, a precipitação do 95\textsuperscript{\underline{o}} percentil na Bacia do Prata foi mais severa, pois possui maiores valores de $Z_c$.

O aumento abrupto de $Z_c$ associado a fusão da água, entre 30-20S e 60-50W, figura \ref{deriv_prata}, principalmente do 50\textsuperscript{\underline{o}} e 70\textsuperscript{\underline{o}} percentil, ocorrem em -4 \textsuperscript{o}C, enquanto que, entre 10-0S e 70-60W, figura \ref{deriv_amazonas}, o aumento de $Z_c$ ocorre mais próximo de 0 \textsuperscript{o}C, o que indica maior presença de água super-resfriada associada ao processo de derretimento da precipitação entre 30-20S e 60-50W, região da Bacia do Rio da Prata.  

Na região da Bacia do Prata, representada na figura \ref{deriv_prata}, o efeito de banda brilhante também é mais pronunciado para a precipitação com raios dos extremos de FT, o que mostra que as regiões eletricamente ativas da precipitação dos sistemas com índice extremo de FTA é menos estratificada do que nos extremos de FT, em ambas as Bacias Hidrológicas: da Prata e do Amazonas.

\newpage
\section{SEVERIDADE REGIONALIZADA}

Aqui, o estudo da densidade de probabilidade de FTA e FT, conforme mostrado na figura \ref{pdfFTAFT}, foi feito para os sistemas ocorridos em cada região de 2.5 por 2.5 graus de latitude e longitude entre 40N-10S e 90-30W. Verifica-se a distribuição geográfica, dos valores de FTA e FT mais frequentes e mais raros conforme cada localidade.

Buscando identificar quais dos índices, FTA ou FT foi mais sensível para indicar a intensidade convectiva das tempestades elétricas, torna-se interessante verificar quais são as regiões aonde sistemas com os menores valores de FTA e FT são mais frequentes.

Nas figuras \ref{extremosInfFTA} e \ref{extremosInfFT}, temos os valores de FTA e FT  para o 5\textsuperscript{\underline{o}} e 10\textsuperscript{\underline{o}} percentil, das distribuições de probabilidades regionalizadas a cada 2.5 por 2.5 graus.

\begin{figure}
  \subfloat[\textsuperscript{\underline{o}} percentil de FTA]{{\includegraphics[height=6.5cm, trim=0 0 0 0, clip]{img/DistEspacialPercentis/FTA/distEspacialValor005thFta}} \label{5oFta}}\\
  \subfloat[10\textsuperscript{\underline{o}} percentil de FTA]{{\includegraphics[height=6.5cm, trim=0 0 0 0, clip]{img/DistEspacialPercentis/FTA/distEspacialValor010thFta}} \label{10oFta}} 
  \caption{Distribuição espacial dos valores do 5\textsuperscript{\underline{o}} e 10\textsuperscript{\underline{o}} percentil da amostra de probabilidade do índice FTA a cada região de 2.5 por 2.5 graus de latitude e longitude.}
\label{extremosInfFTA}
\end{figure} 

A linha de contorno na cor preta em cada mapa apresentado nesta seção, corresponde ao valor do percentil determinado para a análise regional, porém, é referente a amostragem total exposta na figura \ref{cdfFTAFT}.

No ambiente oceânico e costeiro, as tempestades elétricas mais frequentes devem possuir menores índices de severidade do que no continente, pois na costa e oceano observa-se as maiores probabilidades de ocorrência de chuva quente \cite{Liu2009}. 
%o aquecimento da superfície durante o ciclo diurno é menor e o processo de colisão coalescência é dominante em relação aos processos que envolvem a formação de gelo de nuvem 

Nas figuras \ref{5oFta} e \ref{10oFta}, os contornos com valores de 0.05 $\times$ 10$^{-4}$ e 0.12 $\times$ 10$^{-4}$ raios minutos$^{-1}$
km$^{-2}$ respectivamente, demarcam claramente a divisão entre a convecção oceânica e a continental. O Oceano Verde, conceito associado a convecção durante o regime de ventos de Oeste na estação chuvosa Amazônica, discutido por \citeonline{silva2002lba,williams2002}, é bastante evidente. A região central da Bacia Amazônica possui os  valores de FTA na mesma ordem de magnitude e no mesmo percentil das densidades de probabilidades de FTA regionalizadas das tempestades elétricas oceânicas e costeiras.

Os valores de FT associados ao 5\textsuperscript{\underline{o}} e 10\textsuperscript{\underline{o}} percentil, mostrados nas figuras \ref{5oFt} e \ref{10oFt}, revelam os menores valores de FT no centro do continente, principalmente nas regiões continentais fora da área de atuação da ZCIT e de sistemas transientes subtropicais.

\begin{figure}[!ht]
  \centering{
  \subfloat[5\textsuperscript{\underline{o}} percentil de FT]{{\includegraphics[height=6.5cm, trim=0 0 0 0, clip]{img/DistEspacialPercentis/FT/distEspacialValor005thFt}} \label{5oFt}}\\
  \subfloat[10\textsuperscript{\underline{o}} percentil de FT]{{\includegraphics[height=6.5cm, trim=0 0 0 0, clip]{img/DistEspacialPercentis/FT/distEspacialValor010thFt}} \label{10oFt}}
  }    
  \caption{Distribuição espacial dos valores do 5\textsuperscript{\underline{o}} e 10\textsuperscript{\underline{o}} percentil da amostra de probabilidade do índice FT a cada região de 2.5 por 2.5 graus de latitude e longitude.}
\label{extremosInfFT}
\end{figure} 

Para avaliar a distribuição geográfica dos extremos superiores dos índices FTA e FT, verificou-se os valores do 95\textsuperscript{\underline{o}} e 99\textsuperscript{\underline{o}} percentil das amostragens, os quais são expostos nos mapas das figuras \ref{extremosSupFTA} e \ref{extremosSupFT}.

%: regionalizadas, representados nas cores das figuras \ref{extremosSupFTA} e \ref{extremosSupFT}; e totais (ver figura \ref{seriesFtaFt}), representados pela linha de contorno das figuras \ref{extremosSupFTA} e \ref{extremosSupFT}.

Observa-se na figura \ref{95oFta} que os sistemas com índice FTA  superior à 52.76 $\times$ 10$^{-4}$ raios minutos$^{-1}$
km$^{-2}$, são considerados de severidade extrema, pois correspondem a valores superiores ao valor do 95\textsuperscript{\underline{o}} percentil da amostragem total de FTA da figura \ref{pdfFta}. Porém, em regiões no interior do continente, os valores de FTA do 95\textsuperscript{\underline{o}} percentil das amostragens regionalizadas, atingiram  111.97 $\times$ 10$^{-4}$ raios minutos$^{-1}$ km$^{-2}$.
%, mostrando que nestas regiões, valores de 52.76 $\times$ 10$^{-4}$ raios minutos$^{-1}$ km$^{-2}$ são mais frequentes do que nas regiões em que passa a linha de contorno de 52.76 $\times$ 10$^{-4}$ raios minutos$^{-1}$ km$^{-2}$.
%raios por minutos por quilômetros quadrado

\begin{figure}[!ht]
\centering
{\includegraphics[height=13.5cm, trim=0 0 0 0, clip]{img/DistEspacialPercentis/FTA/distEspacialValor095thFta}} 
\caption{Distribuição espacial dos valores do 95\textsuperscript{\underline{o}} percentil da amostra de probabilidade do índice FTA a cada região de 2.5 por 2.5 graus de latitude e longitude.}
\label{95oFta}
\end{figure} 
  
\begin{figure}[!ht]
\centering  
{\includegraphics[height=13.5cm, trim=0 0 0 0, clip]{img/DistEspacialPercentis/FTA/distEspacialValor099thFta}}
\caption{Distribuição espacial dos valores do  99\textsuperscript{\underline{o}} percentil da amostra de probabilidade do índice FTA a cada região de 2.5 por 2.5 graus de latitude e longitude.}
\label{99oFta}
\end{figure} 


Os valores do 99\textsuperscript{\underline{o}} percentil na figura \ref{99oFta}, mostram que no Leste do estado do Amazonas, no Acre e Tocantis e Sudeste do Peru e Norte da Bolívia, regiões estas que compõem o Oceano Verde, a severidade extrema de FTA possui valores entre 148.93-230.00  $\times$ 10$^{-4}$ raios minutos$^{-1}$ km$^{-2}$,  valores que correspondem aos mais extremos do continente Sul-americano. 

Mesmo que a Floresta Amazônica seja um Oceano Verde para atmosfera durante as fases ativas do SAMS, durante o regime de ventos de Leste na estação chuvosa que associa-se as fases inativas da SAMS e durante a estação de transição seca-úmida (SON), ``o Mar Verde"  ~fica revolto. Mesmo que a Floresta Amazônica dialogue com a precipitação como um oceano, este oceano possui temperatura superficial média na classe das maiores temperaturas superficiais continentais globais e está cercado por um vasto continente. Portanto, tem a capacidades de gerar tempestades elétricas extremamente severas, mostrando que a interação entre a Floresta Amazônica e a atmosfera é bastante diversificada.

As regiões dos maiores valores do 95\textsuperscript{\underline{o}} e 99\textsuperscript{\underline{o}} percentil do índice FTA, os quais são  expostos nas figuras \ref{95oFta} e \ref{99oFta}, são principalmente: a Bacia do Rio da Prata, a região Leste Amazônia e as regiões  do planalto Brasileiro, que se estendem por quase todo o país.

Observa-se que os sistemas mais severos da América do Sul ocorrem associados ao relevo nas regiões entre o Pantanal Mato-grossense e o Planalto Central Brasileiro, entre as Bacias dos Rios: Xingu, Araguaia e Tocantis e também o Planalto Central Brasileiro, entre a Bacia do Rio Paraná e o Planalto Meridional Brasileiro, aonde está localizado os planaltos e chapadas da Bacia do Paraná. Nestas regiões os sistemas severos possuem índice FTA superiores à 80 $\times$ 10$^{-4}$ raios minutos$^{-1}$ km$^{-2}$, como mostram as cores das figuras \ref{95oFta} e \ref{99oFta}. 

Note que para saber aproximadamente o número de raios produzidos pelos sistemas extremos de FTA temos que multiplicar o índice FTA pela área do sistema. Por exemplo, a equação \ref{FTAkm2}, descreve que nas regiões em que os sistemas extremos possuem 100 $\times$ 10$^{-4}$ raios minutos$^{-1}$ km$^{-2}$, um sistemas severo com área de 10$^3$ km$^2$ então possui 10 raios observados pelo LIS em 1 minuto.  

\begin{equation}
100 \times 10^{-4} \left[ \frac{\mathrm{raios}}{\mathrm{minutos}~\mathrm{km}^2} \right]  10^3 [ \mathrm{km}^2 ] = 10 \left[ \frac{\mathrm{raios}}{\mathrm{minutos}}\right]  
\label{FTAkm2}
\end{equation}

Os mapas das figuras \ref{95oFt} e \ref{99oFt}, mostam que nas Bacias: do Rio da Prata principalmente, do Rio Araguaia, Rio Xingu e Rio Tocantis, são locais em que os sistemas possuem os maiores índices de FT tanto no 95\textsuperscript{\underline{o}} quanto no 99\textsuperscript{\underline{o}} percentil.

\begin{figure}[!ht]
\centering
{\includegraphics[height=13.5cm, trim=0 0 0 0, clip]{img/DistEspacialPercentis/FT/distEspacialValor095thFt}} 
\caption{Distribuição espacial dos valores do 95\textsuperscript{\underline{o}} percentil da amostra de probabilidade do índice FT a cada região de 2.5 por 2.5 graus de latitude e longitude.}
\label{95oFt}
\end{figure} 
  
\begin{figure}[!ht]
\centering  
{\includegraphics[height=13.5cm, trim=0 0 0 0, clip]{img/DistEspacialPercentis/FT/distEspacialValor099thFt}}
\caption{Distribuição espacial dos valores do  99\textsuperscript{\underline{o}} percentil da amostra de probabilidade do índice FT a cada região de 2.5 por 2.5 graus de latitude e longitude.}
\label{99oFt}
\end{figure}

Os maiores valores do 95\textsuperscript{\underline{o}} e 99\textsuperscript{\underline{o}} percentil do índice FT, figuras \ref{95oFt} e \ref{99oFt},  ficam situados na região Sul da América do Sul, compatível com a região em que \citeonline{cecil2005}, apontam como o local das tempestades categoria 5, ou seja, das mais severas do globo.
\chapter{CONCLUSÃO}

Cria-se uma metodologia para caracterizar as tempestades elétricas observadas na AS a partir das medidas integradas dos sensores LIS, PR e VIRS a bordo do TRMM, durante o período de janeiro de 1998 a dezembro de 2011. As tempestades elétricas foram definidas por pixeis contíguos com $T_b$ $\leq$ 258 K ($\lambda$ = 10,8 $\mu$m -- VIRS) com pelo menos um raio do LIS. Para tempestades elétricas consideras enormes, foi utilizado o limiar de $T_b$ $\leq$ 221 K, pois as mesmas faziam parte de sistemas frontais ou ZCAS. A partir deste procedimento foi criado um banco de dados de 157~592 tempestades elétricas do TRMM, em que cada sistema é composto por: distribuição de $T_b$ e respetivas latitude e longitudes do VIRS, perfis verticais de $Z_c$ e a classificação convectiva, estratiforme e outros, taxa de precipitação na superfície e respectivas latitudes e longitudes, localização (latitude e longitude) dos eventos, grupos e raios, do LIS e tempo de visada com resolução de 0,25$^{\circ}$ $\times$ 0,25$^{\circ}$.

Com base neste subconjunto de dados do TRMM, o Marco das tempestades elétricas na AS foi avaliado através  do ciclo diurno, ciclo anual, distribuição geográfica de densidades (anual e sazonal) de raios e tempestades elétricas e por fim a densidade geográfica anual de raios por tempestades elétricas.  

Para estudar a severidade das tempestades elétricas foram utilizados dois índices associados ao número de raios dos sistemas: a taxa de raios no tempo -- FT -- e a taxa de raios no tempo normalizado pela área do sistema -- FTA. O FT é comumente utilizado  para prognosticar a ocorrência de granizo e tornados \cite{williams1999,goodman1988,schultz2011,gatlin2010}. Já o FTA foi pensado com o intuito de estudar a eficiência da tempestade elétrica em produzir raios. 

%uma vez que na literatura os extremos são avaliados em termos da taxa de raios no tempo, mas não avaliam a eficiência da produção de raios

Os processos microfísicos que levam as tempestades elétricas a terem mais ou menos raios, foram investigados utilizando os CFADs \cite{yuter1995}. Entretanto, a América do Sul cobre uma vasta extensão territorial que vai do equador até os sub-trópicos e a altura das isotermas podem variar significativamente e como os processos de eletrificação de nuvens  dependem essencialmente da temperatura \cite{Takahashi1978}, faz-se necessário converter a base de altura dos CFADs para temperaturas. Logo criou-se o diagrama CFTD que proporciona uma compreensão a respeito das mudanças de $Z_c$ em função da temperatura do perfil atmosférico, o que auxiliou na identificação de água super-resfriada, cristais de gelo, agregados, \textit{graupel} e granizo, que são responsáveis pela transferência de cargas dentro das nuvens \cite{Takahashi1978,saunders2008}.


\section{MARCO DAS TEMPESTADES ELÉTRICAS}

No o ciclo diurno das tempestades elétricas sobre a AS, observa-se que 40\% dos sistemas ocorreram entre 13h e 17h (HL), mostrando que o aquecimento diurno e o aumento da camada limite planetária no decorrer do dia, aumentam a probabilidade de ocorrência em 5,4 vezes (às 14h HL) em relação ao horário de menor ocorrência, às 9h HL.

No continente observa-se um predomínio da atividade elétrica entre às 13h e 17h HL, enquanto que no oceano existe uma variação dependendo da proximidade com o continente, mas em geral, sobre o oceano, observa-se dois máximos: um por volta das 20h HL e outro entre 4--5h HL. Quando próximo do continente o pico das 20h HL se desloca para às 15h HL.  

Apesar das tempestades elétricas sobre a AS possuírem um ciclo diurno bem definido, entre 0$^{\circ}$--10$^{\circ}$ Norte e 80$^{\circ}$--70$^{\circ}$ Oeste, observou-se a maior probabilidade ($\simeq$0,4\%) de tempestades elétricas noturnas da América do Sul, entre 0h e 0h59min HL. A circulação de vale e montanha associada com a topografia elevada na Colômbia, principalmente a região do Parque Nacional Natural Paramillo, e o Lago Maracaibo na Venezuela, e a atuação da Zona de Convergência Intertropical (ZCIT), promovem condições para o desenvolvimento de tempestades elétricas noturnas de maneira mais eficiente do que as demais regiões da AS \cite{burgesser2012}.

Sobre o oceano costeiro, temos que entre 0$^{\circ}$--10$^{\circ}$ Norte e 90$^{\circ}$--80$^{\circ}$ Oeste (Panamá e Sul da Costa Rica,  Oceano Pacífico que engloba o Parque Nacional da Ilha do Coco na Costa Rica) a  probabilidade de ocorrência de tempestades elétricas da AS, com pico de ocorrência às 4h e outro às 14h. Possivelmente o aquecimento superficial durante o dia e as trocas de energia na forma de calor entre o oceano e a atmosfera explicam esta distribuição bimodal.

A maior atividade horária de tempestades elétricas (0,8\%), ocorreu entre 10$^{\circ}$--0$^{\circ}$ Sul e 70$^{\circ}$--50$^{\circ}$ Oeste e 20$^{\circ}$--10$^{\circ}$ Sul e 60$^{\circ}$--50$^{\circ}$ Oeste, entre às 14h e 16h HL.

Apesar dos SCMs apresentarem um ciclo diurno com atividade noturna \cite{Velasco1987, Durkee2009, machado1998}, entre 30$^{\circ}$--20$^{\circ}$ Sul e 60$^{\circ}$--50$^{\circ}$ Oeste, as tempestades elétricas apresentaram um máximo por volta das 15h HL.


%\section{CICLO ANUAL}

No ciclo anual, a estação de tempestades elétricas na América do Sul se configura entre outubro e março e possui dois picos: janeiro, durante o verão austral; e outubro, período de transição entre a estação seca e chuvosa, quando se observa a maior atividade de tempestades elétricas. 

Distribuições bimodais de atividade elétrica, março e outubro, estão restritas a parte Norte da AS entre 0$^{\circ}$-10$^{\circ}$ Norte e 80$^{\circ}$--70$^{\circ}$ Oeste e entre a Amazônia, definidas pelas regiões 10$^{\circ}$ Sul--0$^{\circ}$ e 80$^{\circ}$--50$^{\circ}$ Oeste. 

A parte Nordeste da AS, entre 0$^{\circ}$--10$^{\circ}$ Norte e 70$^{\circ}$--50$^{\circ}$ Oeste, é marcada por um máximo em agosto, verão do hemisfério Norte. No sul da AS,  entre 40$^{\circ}$--20$^{\circ}$ Sul e 70$^{\circ}$--60$^{\circ}$ Oeste, região da Argentina e Bacia do Prata, foi encontrada a estação de tempestades elétricas mais curta (2 meses), e uma estação quase sem raios durante o inverno austral.

A região com a estação de maior duração de tempestades elétricas,  9 meses (março a novembro), foi sobre a Colômbia e parte Oeste da Venezuela que abrange até o lago Maracaibo.  


%\section{DISTRIBUIÇÃO GEOGRÁFICA}

Referente as distribuição de densidades geográficas, as maiores densidades de tempestades elétricas situam-se na parte Norte e Noroeste da AS, ou seja, na Colômbia associado ao extremo Norte da Cordilheira dos Andes e ao Norte/Noroeste da Floresta Amazônica respectivamente, com valores entre 3,5-4,7 $\times$ 10$^{-4}$ km$^{-2}$. Além disso observa-se uma vasta região com densidades superiores a  2,5 $\times$ 10$^{-4}$ km$^{-2}$, que conta com regiões de topografia elevada como à Noroeste do Lago Titicaca no Peru e algumas regiões do Planalto Brasileiro como sobre a Serra Catarinense e o Parque Nacional das Emas ao Sudoeste de Goias, além de grande parte da região Amazônica, Colômbia, Venezuela e Panamá.  

Em termos de sazonabilidade temos que a primavera apresenta a maior atividade de tempestades elétricas (57~861) seguidas pelo verão (46~077), outono (36~804) e inverno (16~850). 

Durante o verão a máxima atividade é encontrada ao Sul da Amazônia se estendendo pela parte central e Sudeste do Brasil, além da cordilheira dos Andes abrangendo o Peru e Bolívia. No Outono a maior atividade se concentra no litoral do Maranhão e Pará, Sudeste e Noroeste  da Colômbia. No inverno a máxima atividade  fica restrita na região costeira da Colômbia e Panamá. Finalmente na primavera observa-se o máximo se estendendo do Panamá, Colômbia, Venezuela, Sul da Venezuela, Noroeste e centro da Amazônia e Nordeste da Bolívia. 
 
As maiores densidade de raios por ano por quilômetro quadrado sobre a AS são observadas: na Foz do Rio Catatumbo na Venezuela com 371,2 raios ano$^{-1}$ km$^{-2}$ e em Cochabamba na Bolívia com $\simeq$60  raios ano$^{-1}$ km$^{-2}$. Regiões com densidades de raios entre 30-60 raios ano$^{-1}$ km$^{-2}$, correspondem ao Pico das Agulhas Negras na Serra da Mantiqueira entre Minas Gerais e o Rio de Janeiro, Pico da Neblina no Amazonas, parte central da Bacia do Prata, na cordilheira dos Andes do Peru, e extremo Norte da Cordilheira dos Andes sobre a Colômbia. 

Em termos sazonais, os extremos de densidade de raios ficam restritos ao inverno e primavera. Sendo que na primavera temos as maiores densidades correspondentes aos máximos anuais, exceto Pico das Agulhas Negras e parte central da Bacia do Prata. Já no inverno basicamente sobre o extremo Norte da Cordilheira dos Andes na Colômbia e no lago Maracaibo na Venezuela. As regiões do Pico das Agulhas Negras apresenta alta atividade durante o verão e primavera, já  a parte central da Bacia do Prata, apresenta atividade elétrica  no verão, outono e primavera.


Referente as regiões mais eficientes em termos de densidades de raios por tempestades, temos que a região do Lago Maracaibo, na Foz do Rio Catatumbo (pixel com 772 km$^{2}$) se mostra a mais eficiente da AS, atingindo o valor máximo de 11,73 $\times$ 10$^{-2}$ ano$^{-1}$ km$^{-2}$, que representa {114 333} raios ano$^{-1}$, seguida da Bacia do Prata e Serra de Córdoba da Argentina com valores $\simeq$5,5 $\times$ 10$^{-2}$ ano$^{-1}$ km$^{-2}$.


\section{TEMPESTADES ELÉTRICAS SEVERAS}

As tempestades elétricas severas, ou, as mais intensas e raras, foram definidas a partir do limiar de 90\% da distribuição de frequência dos índices FT e FTA, que tiveram pelo menos um perfil vertical do PR com chuva válida e $VT_m$ maior do que 60 segundos  definindo um grupo de 9475 tempestades elétricas extremas de FTA com valores entre 29,3--1258 $\times$ 10$^{-4}$ raios min$^{-1}$ km$^{-2}$, e mais 9475 tempestades elétricas extremas de FT com os valores entre 47,2--1283,6 raios min$^{-1}$.  Estes valores de FT são consistentes com os trabalhos de \citeonline{cecil2005}, que encontraram estes valores nos \textit{top} 0,01\% das PFs (categoria 3, 4, 5).


As tempestades elétricas extremas de FTA tem 72\% de fração convectiva e 32\% de fração estratiforme, enquanto as tempestades elétricas extremas de FT 20\% de fração convectiva e 75\% de fração estratiforme. Estas características indicam que os extremos de FTA correspondem a sistemas novos ou em via de maturação, enquanto os sistemas extremos de FT correspondem a sistemas maduros ou em fase de decaimento \cite{learyHouse1979}. 

O estudo da precipitação tridimensional por meio dos CFADs e CCFADs verifica-se que os perfis com raios possuem maiores valores de  $Z_c$ e são mais profundos na atmosfera. Entre 5--7 km de altitude os valores de $Z_c$ para a precipitação com raios atingem valores entre 5 dBZ e 10 dBZ superiores do que para a precipitação sem raios, tanto para os extremos de FTA quanto FT. As regiões sem raios dos eventos extremos de FT são mais estratiformes do que para os extremos de FTA e apresentam forte assinatura da banda brilhante. Os pixeis sem raios da precipitação dos extremos de FTA não há assinatura da banda brilhante nos sistemas sobre o continente, mas há banda brilhante sobre o oceano. Nas regiões com raios, as tempestades elétricas dos extremos de FTA possuem entre 1-3 dBZ a mais do que os valores de $Z_c$ dos extremos de FT, especialmente acima de 5 km de altitude. Do ponto de vista da intensidade convectiva, ambos FTA e FT, são mais vigorosos do equador para os sub-trópicos. 


Com a criação dos CFTDs e CCFTDs identifica-se que durante o desenvolvimento vertical, a intensidade convectiva pode ser mensurada avaliando a redução ($\simeq$7 dBZ) de $Z_c$ devido ao congelamento dos hidrometeoros acima da isoterma de 0 $^{\circ}$C. A condição de tempo severo, deve estar associada a um maior caminho de temperatura para o congelamento dos hidrometeoros. Com o aumento da intensidade da corrente ascendente há uma aumento da  concentração de gotas pequenas \cite{bigg1953}, o que aumenta a concentração de água super-resfriada e a espessura da camada fria de nuvem e consequentemente expõe os hidrometeoros a um número maior de colisões, gerando maior concentração e diversidade de hidrometeoros\footnote{Gotas de água super-resfriadas, cristais de gelo de diferente formas geométricas, agregados e flocos de neve, \textit{graupel} e granizo com diferentes tamanhos e densidades}, envigorando os processos de eletrificação das nuvens.  

A probabilidade de perfis de $Z_c$ por nível de temperatura (CFTD), apresenta quantis mais largos para os extremos de FTA do que nos extremos de FT. Esse efeito demonstra a maior diversidade de hidrometeoros no ambiente das tempestades elétricas extremas de FTA.    

%Isso implica que nas FTAs temos uma maior camada mista que possibilita maior carregamento ....
%Diferenças entre trópicas e sub-trópicos. Com e sem. 
%Falar das diferenças em termo de temperatura.
%Taxas de variação de $Z_c$ por temperatura.
%Combinando as observações de raios com a análise dos processos microfísicos de crescimentos dos hidrometeoros de nuvem com base na estrutura tridimensional da precipitação, 

A convecção mais intensa sobre a AS consiste em um fenômeno atmosférico capaz de processar a energia potencial convectiva em uma área entre 50-400 km$^{2}$, quando há extremos de FTA, que ocorre nos estágios iniciais das tempestades elétricas. Tempestades elétricas com FT extremo, consistem em um fenômeno  atmosférico com dimensões de $\simeq$50~000 km$^{2}$. As condições de tempo severo que são associada com a intensidade da corrente ascendente, devem ocorrer nas tempestades elétricas extremas de FT, porém quando há novas células convectivas (50-400 km$^{2}$) com FTA extremo, embebidas na extensão das tempestades elétricas maiores. Tempestades elétricas com FT extremo, podem possuir grande número de raios, porém distantes entre si. Nestes casos, temos tempestades grandes e com núcleos de raios pouco eficientes, que provavelmente não estão associadas com fortes correntes ascendentes, ou seja, com condições de tempo severo.


As tempestades elétricas associadas com condições de tempo severo devem estar contidas tanto no grupo dos valores extremos de FT quanto de FTA. A intersecção dos dois grupos de extremos indicará as regiões de máxima severidade. %Nesse sentido temos o Planalto Meridional Brasileiro e no semi-árido Argentino a Leste da Cordilheira dos Andes e da Serra de Córdoba.

Comparando as regiões da Bacia Amazônica (10$^{\circ}$ Sul--0$^{\circ}$ e 70$^{\circ}$--60$^{\circ}$ Oeste) a Bacia do Prata (30$^{\circ}$--20$^{\circ}$ Sul e 60$^{\circ}$-50$^{\circ}$ Oeste), observa-se que para o 95\textsuperscript{\underline{o}} percentil do CCFTD, entre 0 $^{\circ}$C e -15 $^{\circ}$C, as tempestades elétricas extremas de FTA na região Amazônica decrescem 5 dBZ enquanto que as tempestades elétricas extremas de FT decrescem 10 dBZ e na Bacia do Prata, FTA decresce também 5 dBZ, enquanto FT decresce apenas 6 dBZ. A máxima taxa de decrescimento de $Z_c$ em função da temperatura (50\textsuperscript{\underline{o}} percentil) é de $\simeq$-1 dBZ $^{\circ}$C$^{-1}$ para a Amazônia e $\simeq$-0,85 dBZ $^{\circ}$C$^{-1}$ para a Argentina.


As tempestades elétricas com os maiores (95 \textsuperscript{\underline{o}} percentil) valores de FTA, são encontradas sobre uma vasta região da AS, com valores de 52,76 $\times$ 10$^{-4}$ raios min$^{-1}$ km$^{-2}$. Valores de FTA superiores a 84 $\times$ 10$^{-4}$ raios min$^{-1}$ km$^{-2}$ são encontrados ao Sul e Leste da Bacia Amazônica, parte central e Sudeste do Brasil, Sul do Peru, Bolívia, Paraguai, Oeste do Uruguai, Norte e centro da Argentina. Ao considerar o 99\textsuperscript{\underline{o}} percentil da amostragem de FTA, as tempestades elétricas passam a produzir 148,93 raios min$^{-1}$ km$^{-2}$, associadas a regiões de topografia elevada, principalmente entre o Pantanal Mato-grossense e o Planalto Central Brasileiro, entre as Bacias dos Rios Xingu, Araguaia e Tocantis, região do Planalto Meridional Brasileiro aonde está localizado os planaltos e chapadas da Bacia do Paraná e região Leste da Serra de Córdoba.


Os maiores valores do 95\textsuperscript{\underline{o}} ($\geq$ 92,84 raios min$^{-1}$) e 99\textsuperscript{\underline{o}} ($\geq$ 272,28 raios min$^{-1}$) percentil do índice FT ficam situados na região Sul da América do Sul.




\section{IMPLICAÇÕES PARA A ELETRIFICAÇÃO DAS NUVENS}

%hidrometeoros (gotículas de água super-resfriadas e \textit{graupel})

A altura em que a água super-resfriada se congela define a espessura da camada em que ocorre o processo de  acreção na nuvem. Conforme a camada de acreção se estende em um caminho maior de temperatura, a eficiência de carregamento positivo/negativo também aumenta devido uma maior diversidade de processos de eletrificação: colisões entre gotas congeladas com \textit{graupel}, cristais de gelo com \textit{graupel}, cristal de gelo com cristais de gelo, agregados com cristais de gelo,  etc. 

Os resultados adquiridos pelos CFTDs, mostram que o decréscimo de $Z_c$ com a diminuição da temperatura é mais pronunciado próximo do equador (Amazônia) do que no Sul da AS (Argentina), mostrando que apesar dos CFADs da porção tropical da AS possuírem níveis de contornos em altitude entre 1--2 km mais elevadas do que na região subtropical\footnote{Esse efeito deve ao fato da tropopausa ser mais alta na região tropical do que nos sub-trópicos.}, o caminho de temperatura que os hidrometeoros percorrem até congelarem é maior no sub-trópico Sul-americano, causando aumento da probabilidade de carregamento, o que corrobora com resultados anteriores de  \citeonline{zipser2006, Rasmussen2011, cecil2005}, que descrevem que as tempestades mais severas da América do Sul foram encontradas sobre o Planalto Argentino e Bacia do Prata.
 
Com o estudo da morfologia da precipitação 3D convertendo a dimensão de altitude (km) para uma dimensão termodinâmica (temperatura), futuramente, poderão ser estimados os conteúdos de água líquida super-resfriada \cite{sekhon1971, hagen2003}, para determinados níveis de temperatura ao longo da camada de acreção, de modo a estimar a polaridade dos centros de cargas dominantes nas nuvens de tempestades elétricas a partir dos resultados de pesquisas de laboratório que mostraram que a polaridade do carregamento depende do conteúdo de água líquida e da temperatura \cite{Takahashi1978,Saunders1999,saunders2008}.




%\apendice
%\include{capitulos/cronogramaProrroga}
%\chapter{ARTIGO SUBMETIDO}

Este artigo foi submetido em 2012, porém recusado, pela revista \textit{Atmospheric Research} edição especial para a \textit{International Conference on Atmospheric Electricity (ICAE)} realizada em 2011 no Rio de Janeiro. Participei da ICAE 2011 apresentando dois trabalhos na forma de pôster.

\includepdf[pages=1, scale=1]{/home/evandro/Dropbox/Tese/submission/atmosres/atmosres-d-12-00171.pdf}
\includepdf[pages=-]{/home/evandro/Dropbox/Tese/submission/ATMOSRES-D-12-00171.pdf}

%\chapter{TRABALHOS APRESENTADOS EM CONFERÊNCIAS}

Foram 5 trabalhos apresentados em 3 conferências internacionais: 2 trabalhos apresentados como primeiro autor, na \textit{International Conference on Atmospheric Electricity (ICAE)} no Rio de Janeiro em 2011;  1 trabalho apresentado como primeiro autor na \textit{16th International Conference on Clouds and Precipitation (ICCP)} realizada em Leipzig, Alemanha em 2012; 2 trabalhos apresentados como primeiro autor, na \textit{International Conference on Atmospheric Electricity (ICAE)} em Norman, OK, Estados Unidos em 2014.

Irei listar apenas os dois trabalhos mais relevante os quais foram apresentados e discutidos mais recentemente na ICAE 2014 e estão sendo aprimorados para a submissão de um artigo para revista. 

\includepdf[pages=-]{/home/evandro/Dropbox/Tese/TrabalhosApresentados/icae2014/AnselmoMorales_oral.pdf}
\includepdf[pages=-]{/home/evandro/Dropbox/Tese/TrabalhosApresentados/icae2014/AnselmoMorales_poster.pdf}









\cleardoublepage
\phantomsection
\renewcommand{\bibname}{REFER\^ENCIAS}
\renewcommand{\arraystretch}{2}
\bibliography{referencias}
%\addcontentsline{toc}{chapter}{Indice Remissívo}
\cleardoublepage
\phantomsection
%\addcontentsline{toc}{chapter}{ÍNDICE REMISSIVO}
\printindex
\end{document}